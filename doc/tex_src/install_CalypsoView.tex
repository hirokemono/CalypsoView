\newpage
\section{Installation}
CalypsoView has a MacOS version and Linux implimentaion. Installation procedure and look and feel are different between these implimentaion.

\subsection{MacOS}
CalypsoView has a binary application bundle in the package. This application needs simply to drag and drop into  
\verb|/Application| folder (see Figure \ref{fig:mac_install}).
%
\begin{figure}[htbp]
\begin{center}
\includegraphics*[width=80mm]{images/install_mac}
\end{center}
\caption{Folder of the MacOS binary}
\label{fig:mac_install}
\end{figure}
%

\subsubsection{Compiler Requirements}
Source code of CalypsoView\_Cocoa is written in C and Objective-C. To hack and build the program, Xcode is required.

\subsubsection{Library Requirements}
\label{sec:requirements_mac}
Calypso requires the following libraries for MacOS version.
\begin{itemize}
\item Xcode (https://developer.apple.com)
\item zlib (https://www.zlib.net)
\end{itemize}
zlib and other foundations are pre-installed in MacOS. Consequently, program should work witout installing any other libraries.

\subsection{Linux}
\subsubsection{Compiler Requirements}
Source code of CalypsoView\_GLFW is written in C. Fortran90 is also used to make a C source code including shader program. Consequently, Fortran compiler is required. GCC, the GNU Compiler Collection (\url{https://gcc.gnu.org}) includes gfortran compiler in the most of Linux distributions. For MacOS, any fortran compiler needs to be installed because Xcode does not have fortran compiler.

\subsubsection{Library Requirements}
\label{sec:requirements}
\begin{itemize}
\item GNU make
\item zlib (https://www.zlib.net)
\item libpng (http://www.libpng.org/pub/png/libpng.html)
\item gtk+3  (https://developer.gnome.org/gtk3/stable/)
\item OpenGL (https://www.opengl.org)
\item GLFW   (https://www.glfw.org)
\end{itemize}
Linux and MacOS use GNU make as a default 'make' command, but some system (e.g. BSD or SOLARIS) does not use GNU make as default. \verb|configure| command searches and set correct GNU make command.

In the most of Linux distributions have these libraries as a default except for {\color{red} GLFW}, but they should have a package to install GLFW.

\subsubsection{Installation of required softwares for Ubuntu Linux}
GCC, the GNU Compiler Collection (\url{https://gcc.gnu.org}) is already installed in the most of Linux distributions. However, packages for development are need to be installed. For Ubuntu 20, for example, the required compilers and  libraries can be installed by using \verb|apt| command as following::
%
\begin{verbatim}
% sudo apt install pkg-config
% sudo apt install git
% sudo apt install gfortran
% sudo apt install zlib1g
% sudo apt install zlib1g-dev
% sudo apt install libpng-dev
% sudo apt install libgtk-3-dev
% sudo apt install libglfw3-dev
\end{verbatim}
%

\subsection{Known problems}
\subsubsection*{OpenGL on MacOS}
Apple set OpenGL as a deprecated library. OpenGL is still supported, but some alternative needs to be considered.

\subsection{Directories}

The top directory of Calypso (ex. \verb|[CALYPSO_HOME]|) contains the following directories.
\begin{verbatim}
% cd [CALYPSO_HOME]
% ls
CMakeLists.txt	Makefile.in	configure.in	examples
INSTALL		bin		doc		src
LICENSE		configure	doxygen		work

\end{verbatim}

\begin{description}
\item{\verb bin:      } directory for executable files
\item{\verb cmake:    } directory for cmake configurations
\item{\verb cmake:    } directory for document generated by doxygen
\item{\verb doc:      } documentations
\item{\verb examples: } examples
\item{\verb src:      } source files
\item{\verb work:     } work directory. Compile is done in this directory.
\end{description}

\subsection{Doxygen}
Doxygen (\url{http://www.doxygen.org}) is an powerful document generation tool from source files. We only save a configuration file in this directory because thousands of html files generated by doxygen. The documents for source codes are generated by the following command:
% 
\begin{verbatim}
% cd [CALYPSO_HOME]/doxygen
% doxygen ./Doxyfile_CALYPSO
\end{verbatim}
%
The html documents can see by opening \verb|[CALYPSO_HOME]/doxygen/html/index.html|.  Automatically generated documentation is also available on the CIG website at \url{http://www.geodynamics.org/cig/software/calypso/}.

\subsection{Install using {\tt configure} command }
\subsubsection{Configuration using {\tt configure} command }
Calypso uses the configure script for configuration to install. The simplest way to install programs is the following process in the top directory of Calypso.
% 
\begin{verbatim}
%pwd
[CALYPSO_HOME]
% ./configure
...
% make
...
% make install
\end{verbatim}
%
After the installation, object modules can be deleted by the following command;
% 
\begin{verbatim}
% make clean
\end{verbatim}
%

{./configure} generates a Makefile in the current directory.  Available options for {\tt configure} can be checked using the {\tt ./configure --help} command. The following options are available in the {\tt configure} command.
%
{\small
\begin{verbatim}
Optional Features:
--disable-option-checking  ignore unrecognized --enable/--with options
--disable-FEATURE       do not include FEATURE (same as --enable-FEATURE=no)
--enable-FEATURE[=ARG]  include FEATURE [ARG=yes]
--enable-silent-rules   less verbose build output (undo: "make V=1")
--disable-silent-rules  verbose build output (undo: "make V=0")
--enable-cocoa          Use Cocoa framework
--enable-dependency-tracking
do not reject slow dependency extractors
--disable-dependency-tracking
speeds up one-time build

Optional Packages:
--with-PACKAGE[=ARG]    use PACKAGE [ARG=yes]
--without-PACKAGE       do not use PACKAGE (same as --with-PACKAGE=no)
--with-zlib=DIR root directory path of zlib installation defaults to
/usr/local or /usr if not found in /usr/local
--without-zlib to disable zlib usage completely
--with-x                use the X Window System

Some influential environment variables:
CC          C compiler command
CFLAGS      C compiler flags
LDFLAGS     linker flags, e.g. -L<lib dir> if you have libraries in a
nonstandard directory <lib dir>
LIBS        libraries to pass to the linker, e.g. -l<library>
CPPFLAGS    (Objective) C/C++ preprocessor flags, e.g. -I<include dir> if
you have headers in a nonstandard directory <include dir>
PKG_CONFIG  path to pkg-config utility
CPP         C preprocessor
ZLIB_CFLAGS C compiler flags for ZLIB, overriding pkg-config
ZLIB_LIBS   linker flags for ZLIB, overriding pkg-config
PNG_CFLAGS  C compiler flags for PNG, overriding pkg-config
PNG_LIBS    linker flags for PNG, overriding pkg-config
GL_CFLAGS   C compiler flags for GL, overriding pkg-config
GL_LIBS     linker flags for GL, overriding pkg-config
GTK3_CFLAGS C compiler flags for GTK3, overriding pkg-config
GTK3_LIBS   linker flags for GTK3, overriding pkg-config
GLFW_CFLAGS C compiler flags for GLFW, overriding pkg-config
GLFW_LIBS   linker flags for GLFW, overriding pkg-config

Use these variables to override the choices made by `configure' or to help
it to find libraries and programs with nonstandard names/locations.
\end{verbatim}
}

At the end of the configuration, The following message can use to check if libraries can be refered correctly:

{\small
\begin{verbatim}
-----   Configuration summary   -------------------------------

host:        "x86_64-apple-darwin16.7.0"

Use Cocoa...           no
Use X Window...       yes

Use zlib ...          yes
Use PNG output...     yes

Use GTK3+...          yes
Use GLFW...           yes

---------------------------------------------------------------
\end{verbatim}
}


\subsubsection{Compile}
Compile is performed using the {\tt make} command. The Makefile in the top directory is used to generate another Makefile in the {\tt work} directory, which is automatically used to complete the compilation. The object file and libraries are compiled in the {\tt work} directory. Finally, the executive files are assembled in {\tt bin} directory. You should find the following programs in the {\tt bin} directory.
%
\begin{description}
\item{\verb kemoviewer_GLFW:    }\\
 Viewer program
\end{description}
%
The following library files are also made in {\tt work} directory.
%
\begin{description}
\item{\verb libcalypsoview.a:    } CalypsoView library
\end{description}
%

\subsubsection{Clean}
The object and fortran module files in {\tt work} directory is deleted by typing
\begin{verbatim}
% make clean
\end{verbatim}
This command deletes files with the extension {\tt .o}, {\tt .mod}, {\tt .par}, {\tt .diag}, and {\tt ~}.

\subsubsection{Distclean}
To revert the files and directory to the original package, use make distclean as
\begin{verbatim}
% make distclean
\end{verbatim}

\subsubsection{Install}
 The executive files are copied to the install directory \verb|$(INSTDIR)/bin|. The install directory \verb|$(INSTDIR)| is defined in Makefile, and can also set by  \verb|${--prefix}| option for \verb|configure| command. Alternatively, you can use the programs in \verb|${SRCDIR}/bin| directory without running \verb|make install|. If directory \verb|${PREFIX}| does not exist, \verb|make install | creates  \verb|${PREFIX}|,  \verb|${PREFIX}/lib|,  \verb|${PREFIX}/bin|, and  \verb|${PREFIX}/include| directories. No files are installed in \verb|${PREFIX}/lib| and \verb|${PREFIX}/include|.

\subsubsection{Construct dependecies (only for developper)}
C source files need dependency among include files. Consequently, list of dependency of source files are saved in the file \verb|Makefile.depends| in each directory. When you modify the source files with changing the module usage,  \verb|Makefile.depends| files need to be updated. To update the  \verb|Makefile.depends|files, use the  \verb|make| command at the \verb|[CALYPSO_HOME]| directory as \\
%
\begin{verbatim}
% make depends
\end{verbatim}

The dependency is generated by the gcc with \verb|-MM -w -DDEPENDENCY_CHECK| option. Consequently, the dependencies need to be generated by the environment with gcc or compatible compiler. After generating the dependency, you can transfer the modified package and build without using gcc.

\subsection{Install without using configure}
\label{section:no_configure}
It is possible to compile Calypso without using the \verb|configure| command. To do this, you need to edit the \verb|Makefile|. First, copy \verb|Makefile| from template \verb|Makefile.in| as
%
\begin{verbatim}
% cp Makefile.in Makefile
\end{verbatim}
In Makefile, the following variables should be defined.
%
\begin{description}
\item{\verb|SHELL|}    Name of shell command.
\item{\verb|SRCDIR|}   Directory of this Makefile. 
\item{\verb|INSTDIR|}  Install directory.
\item{\verb|MPICHDIR|} Directory names for MPI implementation. If you set fortran90 compiler name for MPI programs in \verb|MPIF90|, you do not need to define this valuable.
\item{\verb|F90|} Command name of local Fortran 90 compiler to make a C source file including GLSL shader sources.
\item{\verb|AR|}  Command name for archive program (ex. \verb|ar|) to generate libraries. If you need some options for archive command, options are also included in this valuable.
\item{\verb|RANLIB|} Command name for \verb|ranlib| to generate index to the contents of an archive. If system does not have \verb|ranlib|, set \verb|true| in this valuable. \verb|true| command does not do anything for libraries.
\item{}
\item{\verb|OPTFLAGS|}  Optimization flags for C compiler
\item{\verb|BLAS_LIBS|} Library lists for BLAS  (ex.      \verb|-lblas|)
\item{\verb|ZLIB_CFLAGS|} Option flags for zlib  (ex.     \verb|-I/usr/include|)
\item{\verb|ZLIB_LIB|}   Library lists for zlib (ex.      \verb|-L/usr/lib -lz|)
\item{\verb|PNG_CFLAGS|} Option flags for libpng  (ex.    \verb|`pkg-config --cflags libpng`|)
\item{\verb|PNG_LIBS|}   Library lists for libpng (ex.    \verb|`pkg-config --libs libpng`|)
\item{\verb|X_CFLAGS|} Option flags for X window  (ex.    \verb`|pkg-config --cflags x11`|)
\item{\verb|X_LIBS|}   Library lists for X window (ex.    \verb|`pkg-config --libs x11`|)
\item{\verb|OPENGL_INC|} Option flags for OpenGL  (ex.    \verb|`pkg-config --cflags glfw3`|)
\item{\verb|OPENGL_LIBS|}   Library lists for OpenGL (ex. \verb|`pkg-config --libs gl`|)
\item{\verb|GTK3_CFLAGS|} Option flags for gtk+-3  (ex.   \verb|`pkg-config --cflags gtk+-3.0`|)
\item{\verb|GTK3_LIBS|}   Library lists for gtk+-3 (ex.   \verb|pkg-config --libs gtk+-3.0`|)
\item{\verb|GLFW_CFLAGS|} Option flags for GLFW  (ex.     \verb|`pkg-config --cflags glfw3`|)
\item{\verb|GLFW_LIBS|}   Library lists for GLFW (ex.     \verb|`pkg-config --libs glfw3`|)
\end{description}
%

\subsection{Install using cmake}
CMake is a cross-platform, open-source build system. CMake can be downloaded from \url{http://www.cmake.org}. The following procedure is required to install.
%
\begin{enumerate}
\item Create working directory (you can also use \verb|[CALYPSO_HOME]/work|).
\item Generate Makefile and working directories by {\tt cmake} command.
\item Compile programs by {\tt make} command.
\end{enumerate}
%
In this section, \verb|[CALYPSO\_HOME]/work| is used as the working directory.
Options for CMake can be checked by \verb|cmake -i [CALYPSO_HOME]| command at \verb|[CALYPSO_HOME]| \\
\verb|/work|. There are a number of options can be found, but the following valuables are important settings for installation:
%
\begin{itemize}
\item Install directory
\begin{description}
\item{\verb|CMAKE_INSTALL_PREFIX|}  \\
Install directory
\end{description}

\item Compiler settings
\begin{description}
\item{\verb|CMAKE_Fortran_COMPILER|} \\
Fortran90 compiler.
\item{\verb|CMAKE_c_COMPILER|} C compiler.

\item{\verb|CMAKE_Fortran_FLAGS|} \\
Optimization flags for Fortran90 compiler.
\item{\verb|CMAKE_c_FLAGS|} \\
Optimization flags for C compiler.
\end{description}

\item Manual settings for optional features
\begin{description}
\item{\verb|CMAKE_LIBRARY_PATH|}   \\
CMake library  search paths. This directory is used to search FFTW3 library.
\item{\verb|CMAKE_INCLUDE_PATH|}   \\
CMake include search paths. This directory is used to search include file for FFTW3.
\end{description}
%
\end{itemize}
%
The easiest example of using CMake on Mac OS X with gcc9 is the following:
\begin{verbatim}
% cd build
% cmake ~/CALYPSO/ -DCMAKE_Fortran_COMPILER=/opt/local/bin/gfortran-mp-9 \
? -DCMAKE_c_COMPILER=/opt/local/bin/gcc-mp-9 \
? -DCMAKE_Fortran_FLAGS="-O3 -g" -DCMAKE_c_FLAGS="-O3"
\end{verbatim}
%
After configuration, compile and install are started by
\begin{verbatim}
% make
...
% make install
\end{verbatim}
%

After running \verb|make| command, execute files are built in \verb|[CALYPSO_HOME]/work/bin| directory.

\section{Start the program}
\subsection{Mac OS}
The program will start by double clicking the application icon. Viewer window and menu window will open as shown in Figure \ref{fig:start_mac}.
%
\begin{figure}[htbp]
\begin{center}
\includegraphics*[width=120mm]{images/Desktop_101}
\end{center}
\caption{Windows of CalypsoView at starting.}
\label{fig:start_mac}
\end{figure}
%

\subsection{Linux}
The program is started by input the command in terminal as 
\begin{verbatim}
% make
...
% [BINDIR]/CalypsoView_GLFW
\end{verbatim}
%
The viewer window and menu window will be displayed as shown in Figure \ref{fig:start_linux}.

%
\begin{figure}[htbp]
\begin{center}
\includegraphics*[width=120mm]{images/Desktop_1}
\end{center}
\caption{Windows of CalypsoView at starting.}
\label{fig:start_linux}
\end{figure}
%



