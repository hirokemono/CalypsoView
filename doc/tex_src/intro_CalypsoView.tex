\section{Introduction}
\label{section:introduction}
CalypsoView is a date viewer program for sections and isosurfaces generated by Calypso. Calypso is a program package for magnetohydrodynamics (MHD) simulations in a rotating spherical shell for geodynamo problems.  This program is intended to run on a desktop computer with a single process.

To make the program simple and small as posible, CalypsoView can only visualize cross section and isosurface results and saves image data. To visualize the data with whole volume, please use more powerful visualizaion program such as ParaView or VizIt.

This user guide provides instructions for the configuration and execution of CalypsoView.

\section{History}
\label{sec:history}
Calypso has its origins in two earlier projects. One is a dynamo simulation code written by Hiroaki Matsui in 1990's using a spectral method. This code solves for the poloidal and toroidal spectral coefficients, like Calypso, but it calculates the nonlinear terms in the spectral domain using a parallelization for SMP architectures. The other project is the thermal convection version of GeoFEM, which is Finite Element Method (FEM) platform for massively parallel computational environment, originally written by Hiroshi Okuda in 2000. Under GeoFEM Project, Lee Chen developed cross sectioning, iso-surfacing, and volume rendering modules for data visualization for parallel computations. 

In GeoFEM project, Yoshitaka Wada developed GppView, which is a mesh and surface viewer for FEM mesh data with GeoFEM format and cross sections and isosurfaces obtained by sectioning module from GeoFEM. In 2012, Hiroaki Matsui has developed the data viewer with a same features as GPPView because of the halting of the development of GPPView.

CalypsoView Ver. 0.1 supports the following features and capabilities
%
\begin{itemize}
\item Visualize cross sections and isosurface data from Calypso.
\item Visualize on map using Aitoff projection for contour map.
\item CalypsoView can display up to 10 sectioning and isosurface data in total.
\item Ouptput sequential image files with rotating along with $x-$, $y-$, or $z-$axis (or movie file for Mac OS version)
\item Ouptput time sequential image files (or movie file for Mac OS version)
\item Attach image data as a texture using a spherical coordinate.
\end{itemize}
%

CalypsoView DOES NOT SUPPORT the following features and capabilities.
%
\begin{itemize}
\item Input result data with whole domain and construct isosurface or cross sections.
\end{itemize}
%


\section{Acknowledgements}
\label{section:acknowledgements}
Calypso was primarily developed by Dr. Hiroaki Matsui in collaboration with Prof. Bruce Buffett at the University of California, Berkeley. The following NSF grants supported the development of Calypso, 
%
\begin{itemize}
\item B.A. Buffett, NSF EAR-0509893; Models of sub-grid scale turbulence in the Earth's core and the geodynamo; 2005 - 2007.
\item B.A. Buffett and D. Lathrop,  NSF EAR-0652882; CSEDI Collaborative Research: Integrating numerical and experimental geodynamo models, 2007 - 2009
\item B.A. Buffett, NSF EAR-1045277; Development and application of turbulence models in numerical geodynamo simulations ;  2010 - 2012
\end{itemize}
%

\section{Citation}
\label{section:citation}

Computational Infrastructure for Geodynamics (CIG) and the Calypso developers are making the source code to Calypso available to researchers in the hope that it will aid their research and teaching. A number of individuals have contributed a significant amount of time and energy into the development of Calypso. We request that you cite the appropriate papers and make acknowledgements as necessary. The Calypso development team asks that you cite the following papers:

Matsui, H., E. King, and B.A. Buffett, Multi-scale convection in a geodynamo simulation with uniform heat flux along the outer boundary, {\it Geochemistry, Geophysics, Geosystems}, {\bf 15}, 3212 -- 3225, 2014.
