% Template article for preprint document class `elsart'
% SP 2001/01/05

\documentclass[12pt]{article}
\usepackage{graphicx,color,amssymb}
\usepackage{times,textpos}
\usepackage{geometry}
\usepackage{epstopdf,hyperref,float}
%\usepackage{epstopdf,hyperref,float,slashbox}
\usepackage[titletoc,toc,title]{appendix}

\hypersetup{
    colorlinks=true,
    citecolor=blue,
    linkcolor=blue,
    urlcolor=blue,
}
% Use the option doublespacing or reviewcopy to obtain double line spacing
% \documentclass[doublespacing]{elsart}

% The amssymb package provides various useful mathematical symbols
%\usepackage{amssymb}
\begin{document}
%%%%%%%%%%%%%%%%%%%%%%%%%%%%%%
%%% START OF CIG MANUAL COVER TEMPLATE %%%
%%%%%%%%%%%%%%%%%%%%%%%%%%%%%%
% This should be pasted at the start of manuals and appropriate strings entered at locations indicated with FILL.
% Be sure the TeX file includes the following packages.
% \usepackage{graphicx}
% \usepackage{times}
% \usepackage{textpos}

\definecolor{dark_grey}{gray}{0.3}
\definecolor{purple}{RGB}{163,0,107}

\newgeometry{vmargin={20mm,20mm},hmargin={20mm,10mm}}

%LINE 1%
{
\renewcommand{\familydefault}{\sfdefault}

\pagenumbering{gobble}
\begin{center}
\resizebox{\textwidth}{!}{\textcolor{dark_grey}{\fontfamily{\sfdefault}\selectfont
COMPUTATIONAL INFRASTRUCTURE FOR GEODYNAMICS (CIG)
}}

\hrule

%LINE 2%
\color{dark_grey}
\rule{\textwidth}{2pt}

%LINE 3%
\color{dark_grey}
% FILL: additional organizations
% e.g.: {\Large Organization 1\\Organization 2}
{\Large }
\end{center}

%COLOR AND CODENAME BLOCK%
\begin{center}
\resizebox{\textwidth}{!}{\colorbox
% FILL: color of code name text box
% e.g. blue
{purple}{\fontfamily{\rmdefault}\selectfont \textcolor{white} {
% FILL: name of the code
% You may want to add \hspace to both sides of the codename to better center it, such as:
% \newcommand{\codename}{\hspace{0.1in}CodeName\hspace{0.1in}}
\hspace{0.1in}Calypso\hspace{0.05in}
}}}
\end{center}

%MAIN PICTURE%
\begin{textblock*}{0in}(1.0in,0.55in)
% FILL: image height
% e.g. height=6.5in
\begin{center}
\vspace{.1in}
\includegraphics[height=4.2in]
% FILL: image file name
% e.g. cover_image.png
{Images/Interior2.pdf}
\end{center}
\end{textblock*}

%USER MANUAL%
\color{dark_grey}
\hfill{\Huge \fontfamily{\sfdefault}\selectfont User Manual \\
% FILL: manual version
% e.g. 1.0
\raggedleft \huge \fontfamily{\sfdefault}\selectfont Version {2.0}\\}

%AUTHOR(S) & WEBSITE%
\null
\vfill
\color{dark_grey}
\Large \hfill {\raggedleft \fontfamily{\sfdefault}\selectfont
% FILL: author list
% e.g. Author One\\Author Two\\Author Three\\
% be sure to have a newline (\\) after the final author
Hiroaki Matsui \\
}
{\fontfamily{\sfdefault}\selectfont www.geodynamics.org}

%\hrule

%LINE%
\color{dark_grey}
\rule{\textwidth}{2pt}

}

\pagebreak
\restoregeometry
\pagenumbering{arabic}

%%%%%%%%%%%%%%%%%%%%%%%%%%%%%%
%%%   END OF CIG MANUAL COVER TEMPLATE    %%%
%%%%%%%%%%%%%%%%%%%%%%%%%%%%%%

%******************************macros*********************************
                                   %-----$B;z2<$2(B
\newcommand{\para}{\hspace*{\parindent}}
                                   %-----fraction \by{}{}
\newcommand{\by}[2]{\frac{\displaystyle #1}{\displaystyle #2}}
                                   %-----vector (arrow) \hvec{}
\newcommand{\hvec}[1]{\vec{\mathstrut #1}}
                                   %-----time derivative \tdot{}
\newcommand{\tdot}[1]{\stackrel{\cdot}{#1}}
\newcommand{\tdots}[1]{\stackrel{\cdot \cdot}{#1}}
                                   %-----Bold vector
\newcommand{\bvec}[1]{ \mbox{\boldmath$#1$} }
                                   %-----vector operation ( by text)
\newcommand{\tgrad}{ \mbox{grad} \; }
\newcommand{\tdiv}{ \mbox{div} \; }
\newcommand{\trot}{ \mbox{rot} \; }
%
                                   %-----vector operation ( by nabla)
\newcommand{\bgrad}{ \nabla }
\newcommand{\bdiv}{ \nabla \cdot }
\newcommand{\brot}{ \nabla \times }
                                   %-----partial \rd
\newcommand{\rd}{\partial}
                                   %-----l(l+1)
\newcommand{\llone}[1]{ #1 \left( #1 +1 \right) }
                                   %-----s
%\newcommand{\koube}{ \left[ \left\{ \left( \frac{1}{q_{\perp}^{2}} - 
%  \frac{1}{q_{\parallel}^{2}} \right) \frac{k_{\parallel}p_{\parallel}}
%  {\gamma} + \frac{1}{q_{\parallel}^{2} \frac{{k_{\parallel}p_{\parallel 0}}
%  {\gamma} - \by{1}{q_{\perp}^{2}} \omega \right} f_{l}(p_{\perp},p_{\parallel}%)  
%+ \by{1}{q_{\perp}^{2}} \left( \omega - k_{\parallel} \frac{p_{\parallel}}
%  {\gamma} \right) f_{l-1}(p_{\perp},p_{\parallel}) \right]}
  
				   

%\maketitle
%\input{captions.txt}
%
%

\section*{Preface}
Calypso is a program package of magnetohydrodynamics (MHD) simulations in a rotating spherical shell for geodynamo problems. This package consists of the simulation program, preprocessing program, post processing program to generate field data for visualization programs, and several small utilities. The simulation program runs on parallel computing systems using MPI and OpenMP parallelization.

\newpage
\tableofcontents

%
\newpage
\section{Introduction}
\label{section:introduction}
Calypso is a program package for magnetohydrodynamics (MHD) simulations in a rotating spherical shell for geodynamo problems. This package consists of the simulation program, preprocessing program, post processing program to generate field data for visualization programs, and several small utilities. The simulation program runs on parallel computing systems using MPI and OpenMP parallelization.

Calypso solves the equations that govern convection and magnetic-field generation in a rotating spherical shell. Flow is driven by thermal or compositional buoyancy in a Boussinesq fluid. Calypso also support various boundary conditions (e.g. fixed temperature, heat flux, composition, and compositional flux), and permits a conductive and rotatable inner core. Results are written as spherical harmonics coefficients, Gauss coefficients for the region outside of the fluid shell, and field data in Cartesian coordinate for easily visualization with a number of visualization programs.

This user guide describes the essentials of the magnetohydrodynamics theory and equations behind Calypso, and provides instructions for the configuration and execution of Calypso.

\section{History}
\label{sec:history}
Calypso has its origins in two earlier projects. One is a dynamo simulation code written by Hiroaki Matsui in 1990's using a spectral method. This code solves for the poloidal and toroidal spectral coefficients, like Calypso, but it calculates the nonlinear terms in the spectral domain using a parallelization for SMP architectures. The other project is the thermal convection version of GeoFEM, which is Finite Element Method (FEM) platform for massively parallel computational environment, originally written by Hiroshi Okuda in 2000. Under GeoFEM Project, Lee Chen developed cross sectioning, iso-surfacing, and volume rendering modules for data visualization for parallel computations.. 

Hiroaki Matsui was responsible for adding routines to GeoFEM to perform magnetohydrodynamics simulation in a rotating frame. In 2002 this code successfully performed dynamo simulations in a rotating spherical shell using insulating magnetic boundary conditions.  The following year Matsui implemented a subgrid scale (SGS) model in the FEM dynamo model in collaboration with Bruce Buffett. A module to solve for double diffusive convection was added to the FEM dynamo model by Hiroaki Matsui in 2009.

Progress in understanding the role of subgrid scale models in magnetohydrodynamic simulations relies on quantitative estimates for the transfer of energy between spatial scales. This information is most easily obtained from a spherical harmonic expansion of the simulation results, even when the simulation is performed by FEM. Hiroaki Matsui implemented the spherical harmonic transform in 2007 using a combination of MPI and OpenMP, and later included the spherical harmonic transform routines into his old dynamo code to create Calypso. Additional software in the program package for visualization is based on data formats from the FEM model. In addition, the control parameter file format is adapted from the input formats used in GeoFEM.

Calypso Ver. 1.0 supports the following features and capabilities
%
\begin{itemize}
\item Magnetohydrodynamics simulation for a Boussinesq fluid in a rotating spherical shell.
\item Convection driven by thermal and compositional buoyancy.
\item Temperature or heat flux is fixed at boundaries
\item Composition or compositional flux is fixed at boundaries
\item Non-slip or free-slip boundary conditions
\item Outside of the fluid shell is electrically insulated or pseudo vacuum boundary.
\item A conductive inner core with the same conductivity as the surrounding fluid
\item A rotating inner core driven by the magnetic and viscous torques.
\end{itemize}
%
%
\subsection{Updates for Ver 1.1}
In Version 1.1, a number of bug fixes and additional comments for Doxygen are completed. The following large bugs are fixed:
%
\begin{itemize}
\item \verb|configure| command is updated to find appropriate GNU make command. (see Section \ref{sec:requirements})
\item Label for radial grid type in the file \verb|ctl_sph_shell| \verb|raidal_grid_type_ctl| is changed to \verb|radial_grid_type_ctl|. If the old name is used in the control file, program \verb|gen_sph_grid| will crash.
\end{itemize}
%

And, the following features are implemented
\begin{itemize}
\item New ordering is used for spherical harmonics data to reduce communication time. The old version of spectrum indexing data, which is generated by \verb|gen_sph_grids| in Ver. 1.0 is also supported in Ver. 1.1.
\item Evaluation of Coriolis term is updated. Now, Adams-Gaunt integrals are evaluated in the initialization process in the simulation program \verb|sph_mhd|, so the data file for Adams-Gaunt integrals which is made by \verb|gen_sph_grids|  is not required.
\item Add a program \verb|sph_add_initial_field|. to modify existed initial field data. This program is used to modify or add new fields in spectrum data. (See Section \ref{sec:add_initial_field}.)
\item Heat and composition source terms are implemented. These source terms are fixed with time, and defined as spectrum data. The source terms are defined by using initial field generation program \\ \verb|sph_initial_field| or \verb|sph_add_initial_field|. (See section \ref{sec:sph_initial_field} and  \ref{sec:add_initial_field}.)
\item The boundary conditions for temperature and composition can be defined by using spherical harmonics coefficients. (i.e. inhomogeneous boundary conditions can be applied.) These boundary conditions are defined by using single external data file. (See Section \ref{sec:boundary_file})
\end{itemize}

%
\subsection{Updates for Ver 1.2}
In Version 1.2, the following features are implemented:
\begin{itemize}
\item To reduce the number of calculation, Legendre transform is calculated with taking account to the symmetry with respect to the equator. Time for Legendre transform is approximately half of that in Ver 1.1.
\item BLAS library can be used for the Legendre transform optionally.
\item Cross sectioning and isosurfacing module are newly implemented. These modules are re-written by Fortran90 from the parallel sectioning modules in GeoFEM by Lee Chen in C, and some features are added for visualizations of geodynamo simulations. See section \ref{section:PSF} and \ref{section:ISO}.
\item Initial data assemble program \verb|assemble_mhd| is parallelized. This program can perform with any number of MPI processes, but we recommend to run the program with {\bf one} process or the same number of processes as the number of subdomains for the target configuration which is defined by \verb|num_new_domain_ctl|. See section \ref{sec:add_initial_field}.
\item The time and time step information in the restart data can be modifield by  \verb|assemble_mhd|. See section \ref{sec:add_initial_field}
\end{itemize}

\subsection{Updates for Ver 2.0}
In Version 2.0, there are a number of changes as;
\begin{itemize}
\item Start using Fortran structures to reduce global instances.
\item Using MPI-IO for data IO to generate a single date file from MPI processes
\item Include interface to zlib library (\url{https://www.zlib.net}) for data IO with data compression. zlib is pre-installed in MacOS and most of Linux distributions.
\item Calypso now support various format of data IO (ascii, binary, gzipped ascii, and gzipped binary) through MPI-IO.
\item Include spherical harmonics index generator into simulation program. Consequently, we can start program without preprocessing.
\item Modules to generate longitudinal average data is included into the simulation program.
\item Number of array information is not required to define "array" block in control files.
\item Block layout of the control file is changed for spatial resolution and spherical harmonic data IO.
\end{itemize}






\section{Acknowledgements}
\label{section:acknowledgements}
Calypso was primarily developed by Dr. Hiroaki Matsui in collaboration with Prof. Bruce Buffett at the University of California, Berkeley. The following NSF grants supported the development of Calypso, 
%
\begin{itemize}
\item B.A. Buffett, NSF EAR-0509893; Models of sub-grid scale turbulence in the Earthユs core and the geodynamo; 2005 - 2007.
\item B.A. Buffett and D. Lathrop,  NSF EAR-0652882; CSEDI Collaborative Research: Integrating numerical and experimental geodynamo models, 2007 - 2009
\item B.A. Buffett, NSF EAR-1045277; Development and application of turbulence models in numerical geodynamo simulations ;  2010 - 2012
\end{itemize}
%

\section{Citation}
\label{section:citation}

Computational Infrastructure for Geodynamics (CIG) and the Calypso developers are making the source code to Calypso available to researchers in the hope that it will aid their research and teaching. A number of individuals have contributed a significant amount of time and energy into the development of Calypso. We request that you cite the appropriate papers and make acknowledgements as necessary. The Calypso development team asks that you cite the following papers:

Matsui, H., E. King, and B.A. Buffett, Multi-scale convection in a geodynamo simulation with uniform heat flux along the outer boundary, {\it Geochemistry, Geophysics, Geosystems}, {\bf 15}, 3212 -- 3225, 2014.


\newpage
\section{Model of Simulation}
\subsection{Governing equations}
%
\begin{figure}[htbp]
\begin{center}
\includegraphics*[width=130mm]{images/Spherical_shell}
\end{center}
\caption{Rotating spherical shell modeled on the Earth's outer core.}
\label{fig:shell}
\end{figure}
%
This model performs a magnetohydrodynamics (MHD) simulation in a rotating spherical shell modeled on the Earth's outer core (see Figure \ref{fig:shell}). We consider a spherical shell from the inner core boundary (ICB) to the core mantle Boundary (CMB) in a rotating frame which constantly rotates with angular velocity $\bvec{\Omega} = \Omega \hat{z}$. The fluid shell is filled with a conductive fluid with constant diffusivities (kinematic viscosity $\nu$, magnetic diffusivity $\eta$, thermal diffusivity $\kappa_{T}$, and compositional diffusivity $\kappa_{C}$). The inner core ($0 < r < r_{i}$) is solid, and may be considered an electrical insulator  or may have the same conductivity as the outer core. We assume that the region outside of the core is an electrical insulator. The rotating spherical shell is filled with Boussinesq modeled fluid. The governing equations of the MHD dynamo problem are the following,
%
\begin{eqnarray}
\frac{\partial \bvec{u}}{\partial t} + \left(\bvec{\omega} \times \bvec{u}\right)
 & = & - \nabla \left(P+\frac{1}{2}u^{2} \right) -\nu \nabla \times \nabla \times \bvec{u}
\nonumber \\
 & &  - 2 \Omega \left(\hat{z} \times \bvec{u} \right)
     + \left( \frac{\rho}{\rho_{0}} \bvec{g} \right)
     + \frac{1}{\rho_{0}} \left(\bvec{J} \times \bvec{B} \right),
\nonumber \\
 \frac{\partial \bvec{B}}{\partial t}
 & = & -\eta \nabla \times \nabla \times \bvec{B}
       + \nabla \times \left(\bvec{u} \times \bvec{B} \right),
\nonumber \\
\frac{\partial T}{\partial t} + \left(\bvec{u} \cdot \nabla \right) T
 & = & \kappa_{T} \nabla^{2} T + q_{T},
\nonumber \\
\frac{\partial C}{\partial t} + \left(\bvec{u} \cdot \nabla \right) C
 & = & \kappa_{C} \nabla^{2} C + q_{C},
\nonumber \\
\nabla \cdot \bvec{u} & = & \nabla \cdot \bvec{B} = 0,
\label{eq:conservation}
\nonumber \\
\bvec{\omega} & = & \nabla \times \bvec{u},
\nonumber
\end{eqnarray}
%
and
\begin{eqnarray}
\bvec{J} & = & \frac{1}{\mu_{0}} \nabla \times \bvec{B},
\nonumber
\end{eqnarray}
%
where, $\bvec{u}$, $\bvec{\omega}$, $P$, \bvec{B}, \bvec{J}, $T$, $C$, $q_{T}$, and $q_{C}$ are the velocity, vorticity, pressure, magnetic field, current density, temperature, compositional variation, heat source, and source of light element, respectively. Coefficients in the governing equations are the kinetic viscosity $\nu$, thermal diffusivity $\kappa_{T}$, compositional diffusivity $\kappa_{C}$, and magnetic diffusivity $\eta$. The density $\rho$ is written as a function of $T$, $C$, average density $\rho_{0}$, thermal expansion $\alpha_{T}$, and density ratio of light element to main composition $\alpha_{C}$,
%
\begin{eqnarray}
\rho & = & \rho_{0} \left[1 - \alpha_{T} \left( T - T_{0} \right)
                               - \alpha_{C} \left( C - C_{0} \right) \right]
\nonumber
\end{eqnarray}
%
In Calypso, the vorticity equation and divergence of the momentum equation are used for solving $\bvec{u}$, $\bvec{\omega}$, and $P$ as,
\begin{eqnarray}
\frac{\partial \bvec{\omega}}{\partial t} + \nabla \times \left(\bvec{\omega} \times \bvec{u}\right)
 & = & -\nu \nabla \times \nabla \times \bvec{\omega}
     - 2 \Omega \nabla \times \left(\hat{z} \times \bvec{u} \right)
\nonumber \\
 & & + \nabla \times \left( \frac{\rho}{\rho_{0}} \bvec{g} \right)
     + \frac{1}{\rho_{0}} \nabla \times \left(\bvec{J} \times \bvec{B} \right),
\nonumber
\end{eqnarray}
%
and
\begin{eqnarray}
\nabla \cdot \left(\bvec{\omega} \times \bvec{u}\right)
 & = & -\nabla^{2} \left(P+\frac{1}{2}u^{2} \right) - 2 \Omega \nabla \cdot \left(\hat{z} \times \bvec{u} \right)
\nonumber \\
 & & + \nabla \cdot \left( \frac{\rho}{\rho_{0}} \bvec{g} \right)
     + \frac{1}{\rho_{0}} \nabla \cdot \left(\bvec{J} \times \bvec{B} \right)
\nonumber.
\end{eqnarray}


\subsection{Spherical harmonics expansion}
In Calypso, fields are expanded into spherical harmonics. A scalar field (for example, temperature $T(r, \theta, \phi)$) is expanded as
%
\begin{eqnarray}
T(r, \theta, \phi) &=& \sum_{l=0}^{L} \sum_{m=-l}^{l} T_{l}^{m}(r) Y_{l}^{m}(\theta,\phi),
\nonumber
\end{eqnarray}
where  $Y_{l}^{m}$ are the spherical harmonics. Solenoidal fields (e.g. velocity $\bvec{u}$, vorticity $\bvec{\omega}$, magnetic field $\bvec{B}$, and current density $\bvec{J}$) are decomposed into poloidal and toroidal components. For example, the magnetic field is described as 
\begin{eqnarray}
\bvec{B}(r, \theta, \phi) & = & \sum_{l=1}^{L} \sum_{m=-l}^{l} 
\left( \bvec{B}_{Sl}^{\ m} + \bvec{B}_{Tl}^{\ m} \right),\nonumber
\end{eqnarray}
where
\begin{eqnarray}
\bvec{B}_{Sl}^{\ m}(r, \theta, \phi) & = & \nabla \times \nabla \times \left( B_{Sl}^{\ m}(r) Y_{l}^{m}(\theta,\phi) \hat{r} \right),
\nonumber \\
\bvec{B}_{Tl}^{\ m}(r, \theta, \phi) & = & \nabla \times \left( B_{Tl}^{\ m}(r) Y_{l}^{m}(\theta,\phi) \hat{r} \right).
\nonumber
\end{eqnarray}

The spherical harmonics are defined as real functions. $P_{l}^{m} \cos \left( m\phi \right)$ is assigned for positive $m$, $P_{l}^{m} \sin \left( m\phi \right)$ is assigned for negative $m$, where $P_{l}^{m}$ are Legendre polynomials. Because Schmidt quasi normalization is used for the Legendre polynomials $P_{l}^{m}$, the orthogonality relation for the spherical harmonics is 
%
\begin{eqnarray}
\int Y_{l}^{m} Y_{l'}^{m'} \sin \theta d\theta d\phi &=& 4\pi \frac{1}{2l+1} \delta_{ll'}\delta_{mm'},
\nonumber
\end{eqnarray}
%
where, $\delta_{ll'}$ is Kronecker delta.

\subsection{Evaluation of Coriolis term}
The curl of the Coriolis force $-2\Omega \nabla \times \left(\hat{z} \times \bvec{u} \right)$ is evaluated in the spectrum space using the triple products of the spherical harmonics. These 3j-symbols (or Gaunt integral $G_{Lll'}^{Mmm'}$ and Elsasser integral $E_{Lll'}^{Mmm'}$) are written as
%
\begin{eqnarray}
G_{Lll'}^{Mmm'} & = & \int Y_{L}^{M} Y_{l}^{m} Y_{l'}^{m'}
\sin\theta d\theta d\phi,
\nonumber \\
E_{Lll'}^{Mmm'} & = & \int Y_{L}^{M} \left (
   \frac{\partial Y_{l}^{m}}{\partial \theta} \frac{\partial Y_{l'}^{m'}}{\partial\phi}
 - \frac{\partial Y_{l}^{m}}{\partial \phi} \frac{\partial Y_{l'}^{m'}}{\partial \theta}
 \nonumber
\right) d\theta d\phi.
\nonumber
\end{eqnarray}
%
The Gaunt integral $1/(4\pi) G_{Lll'}^{Mmm'}$ and Elsasser integral $1/(4\pi) E_{Lll'}^{Mmm'}$ for the Coriolis terms are evaluated in the simulation program.


\subsection{Boundary conditions}
Calypso currently supports the following boundary conditions for velocity $\bvec{u}$, magnetic field $\bvec{B}$, temperature $T$, and composition variation $C$. These boundary conditions are defined in the control file \verb|control_MHD|.

\subsubsection{Non-slip boundary}
The velocity $\bvec{u}$ is set to be 0 at the boundary. For poloidal and toroidal coefficients of velocity, $U_{Sl}^{\ m}(r)$ and $U_{Tl}^{\ m}(r)$, the boundary condition can be described as
%
\begin{eqnarray}
U_{Sl}^{\ m}(r) & = & \frac{\partial U_{Sl}^{\ m}}{\partial r} = 0,
\nonumber
\end{eqnarray}
%
and 
%
\begin{eqnarray}
U_{Tl}^{\ m}(r) & = & 0.
\nonumber
\end{eqnarray}
%
\subsubsection{Free-slip boundary}
For a free slip boundary, shear stress and radial flow vanish at the boundary. The boundary condition for poloidal and toroidal coefficients are described as
%
\begin{eqnarray}
U_{Sl}^{\ m}(r) = \frac{\partial^2}{\partial r^2} \left( \frac{1}{r} U_{Sl}^{\ m}(r) \right) & = & 0,
\nonumber
\end{eqnarray}
%
and 
%
\begin{eqnarray}
\frac{\partial}{\partial r} \left( \frac{1}{r^2} U_{Tl}^{\ m}(r) \right) & = & 0.
\nonumber
\end{eqnarray}
%
\subsubsection{Fixed rotation rate}
If the boundary rotates with a rotation vector $\bvec{\Omega}_{b} = \left(\Omega_{bx}, \Omega_{by}, \Omega_{bz}\right)$, the boundary conditions for poloidal and toroidal coefficients are described as
%
\begin{eqnarray}
U_{Sl}^{\ m}(r) & = & \frac{\partial U_{Sl}^{\ m}}{\partial r} = 0,
\nonumber \\
U_{T1}^{\ 1s}(r) & = & r^{2} \Omega_{by},
\nonumber \\
U_{T1}^{\ 0}(r) & = &  r^{2} \Omega_{bz},
\nonumber \\
U_{T1}^{\ 1c}(r) & = & r^{2} \Omega_{bx},
\nonumber
\end{eqnarray}
%
and 
%
\begin{eqnarray}
U_{Tl}^{\ m}(r) & = & 0 \mbox{ for } l > 2.
\nonumber
\end{eqnarray}
%
\subsubsection{Fixed homogenous temperature}
When a constant temperature $T_{b}$ is is applied, the spherical harmonic coefficients are
%
\begin{eqnarray}
T_{0}^{0}(r) & = &  T_{b},
\nonumber
\end{eqnarray}
%
and 
%
\begin{eqnarray}
T_{l}^{m}(r) & = & 0 \mbox{ for } l > 1.
\nonumber
\end{eqnarray}
%
\subsubsection{Fixed homogenous heat flux}
A constant heat flux is imposed by setting the radial temperature gradient to $F_{Tb}$. The spherical harmonic coefficients are
%
\begin{eqnarray}
\frac{\partial T_{0}^{0}}{\partial r} & = &  F_{Tb},
\nonumber
\end{eqnarray}
%
and 
%
\begin{eqnarray}
\frac{\partial T_{l}^{m}}{\partial r} & = & 0 \mbox{ for } l > 1.
\nonumber
\end{eqnarray}
%
\subsubsection{Fixed composition}
When a constant composition $C_{b}$ is applied, the spherical harmonic coefficients are 
%
\begin{eqnarray}
C_{0}^{0}(r) & = & C_{b},
\nonumber
\end{eqnarray}
%
and 
%
\begin{eqnarray}
C_{l}^{m}(r) & = & 0 \mbox{ for } l > 1.
\nonumber
\end{eqnarray}
%
\subsubsection{Fixed composition flux}
A constant composition flux is imposed by setting the radial composition gradient to $F_{Cb}$. The spherical harmonic coefficients are
%
\begin{eqnarray}
\frac{\partial C_{0}^{0}}{\partial r} & = &  F_{Cb},
\nonumber
\end{eqnarray}
%
and 
%
\begin{eqnarray}
\frac{\partial C_{l}^{m}}{\partial r} & = & 0 \mbox{ for } l > 1.
\nonumber
\end{eqnarray}
%
\subsubsection{Connection to the magnetic potential field}
If the regions outside the fluid shell are assumed to be electrical insulators, current density vanishes in the electric insulator
 %
\begin{eqnarray}
\bvec{J}_{ext} &= & 0,
\nonumber
\end{eqnarray}
%
where the suffix ${}_{ext}$ indicates fields outside of the fluid shell. At the boundaries of the fluid shell, the magnetic field $\bvec{B}_{fluid}$, current density $\bvec{J}_{fluid}$ , and electric field $\bvec{E}_{fluid}$ in the conductive fluid satisfy:
 %
\begin{eqnarray}
\left (\bvec{B}_{fluid} - \bvec{B}_{ext} \right)  = 0,
\nonumber \\
\left (\bvec{J}_{fluid} - \bvec{J}_{ext} \right)  \cdot \hat{r}   = 0,
\nonumber
\end{eqnarray}
and 
\begin{eqnarray}
\left (\bvec{E}_{fluid} - \bvec{E}_{ext} \right) \times \hat{r}  = 0,
\nonumber
\end{eqnarray}
%
where, $\hat{r}$ is the radial unit vector (i.e. normal vector for the spherical shell boundaries). 
Consequently, radial current density $\bvec{J} $ vanishes at the boundary as
  %
\begin{eqnarray}
\bvec{J} \cdot \hat{r}  = 0
 \mbox{ at } r = r_{i}, r_{o}
\nonumber
\end{eqnarray}
%

In an electrical insulator the magnetic field can be described as a potential field
 %
\begin{eqnarray}
\bvec{B}_{ext} = - \nabla W_{ext},
\nonumber
\end{eqnarray}
%
where $W_{ext}$ is the magnetic potential. The boundary conditions can be satisfied by  connecting the magnetic field in the fluid shell at boundaries to the potential fields. The magnetic field is connected to the potential field in an electrical insulator. At CMB ($r = r_{o}$), the boundary condition can be described by the poloidal and toroidal coefficients of the magnetic field as
%
\begin{eqnarray}
\frac{l}{r} B_{Sl}^{\ m}(r) & = & - \frac{\partial B_{Sl}^{\ m}}{\partial r},
\nonumber
\end{eqnarray}
%
and 
%
\begin{eqnarray}
B_{Tl}^{\ m}(r) & = & 0.
\nonumber
\end{eqnarray}
%

If the inner core is also assumed to be an insulator, the magnetic boundary conditions for ICB ($r = r_{i}$) can be described as
%
\begin{eqnarray}
\frac{l+1}{r} B_{Sl}^{\ m}(r) & = & \frac{\partial B_{Sl}^{\ m}}{\partial r},
\nonumber
\end{eqnarray}
%
and 
%
\begin{eqnarray}
B_{Tl}^{\ m}(r) & = & 0.
\nonumber
\end{eqnarray}
%

\subsubsection{Magnetic boundary condition for center}
If the inner core has the same conductivity as the outer core, we solve the induction equation for the inner core as for the outer core with the boundary conditions for the center. The poloidal and toroidal coefficients at center are set to 
%
\begin{eqnarray}
B_{Sl}^{\ m}(0) &=& B_{Tl}^{\ m}(0) = 0.
\nonumber
\end{eqnarray}
%

\subsubsection{Pseudo-vacuum magnetic boundary condition}
Under the pseudo-vacuum boundary condition, the magnetic field has only a radial component at the boundaries. Considering the conservation of the magnetic field, the magnetic boundary condition will be
%
\begin{eqnarray}
\frac{\partial}{\partial r}\left(r^{2} B_{r} \right) =  B_{\theta} = B_{\phi} = 0
 \mbox{ at } r = r_{i}, r_{o}.
\nonumber
\end{eqnarray}
%
The present boundary condition is also described by using the poloidal and toroidal coefficients as
%
\begin{eqnarray}
\frac{\partial B_{Sl}^{\ m}}{\partial r} & = &  B_{Tl}^{\ m}(r) = 0
 \mbox{ at } r = r_{i}, r_{o}.
\nonumber
\end{eqnarray}



\newpage
\section{Installation}


\subsection{Compiler Requirements}
Most of source code of Calypso are written in Fortran2003. Consequently, Fortran compiler with supporting fortran 2003 is required. We can obtain a number of information about Fortran from \url{http://fortranwiki.org/}, and you can also find a table of the supported features of Fortran 2003 standard at \url{http://fortranwiki.org/fortran/show/Fortran+2003+status}. In addition, C compiler is optionally required to us zlib support for compressed data IO. 

GCC, the GNU Compiler Collection (\url{https://gcc.gnu.org}) includes gfortran compiler in the most of Linux distributions. For MacOS, any fortran compiler needs to be installed because Xcode does not have fortran compiler. The easiest way is installing GCC by using a package manager such as macports (\url{https://www.macports.org}) or homebrew (\url{https://brew.sh/index}).


\subsection{Library Requirements}
\label{sec:requirements}
Calypso requires the following libraries.
\begin{itemize}
\item GNU make
\item MPI libraries (OpenMPI, MPICH, etc)
\item FFTPACK Ver 5.1D (\url{https://people.sc.fsu.edu/~jburkardt/f_src/fftpack5.1d/fftpack5.1d.html}). The source files for FFTPACK are included in {\tt src/EXTERNAL\_libs} directory.
\end{itemize}
Linux and Max OS X use GNU make as a default 'make' command, but some system (e.g. BSD or SOLARIS) does not use GNU make as default. \verb|configure| command searches and set correct GNU make command. MPI library such as OpenMPI (\url{https://www.open-mpi.org}) or MPICH (\url{https://www.mpich.org}) can be installed by the most of package manager.

In addition, the following environment and libraries can be used (optional).
\begin{itemize}
\item OpenMP 
\item BLAS
\item zlib (https://www.zlib.net)
\item FFTW version 3 (\url{http://www.fftw.org}) including Fortran wrapper
\item PARALLEL HDF5 (\url{https://support.hdfgroup.org/HDF5/PHDF5})  including Fortran wrapper.
\end{itemize}
Note: Calypso does NOT use MPI and OpenMP features in FFTW3. 

In the most of platforms, the Fourier transform by FFTW is faster than that by FFTPACK. 

Zlib is used for compressed data IO. Zlib is installed in most of UNIX platforms.

HDF5 is used for field data output with XDMF format instead of VTK format. The comparison of field data format is described in section ref{sec:VTK}. 

OpenMP is used for the parallelization under the shared memory. Better choice to use both MPI and OpenMP parallelization (so-called Hybrid parallelization) or only using MPI (so-called flat MPI) is depends on the computational platform and compiler. For example, flat MPI has much better performance on Linux cluster with Intel Xeon processors and with Intel fortran compiler, but Hybrid model has better performance on Hitachi SR24000 with Power 8 processors.

\subsection{Known problems}
\subsubsection*{FFTPACK and Intel compiler}
FFTPACK fails to compile with Intel fortran using the {\tt `-warn all'} option. Currently the {\tt `-warn all'} option is excluded by Makefile when FFTPACK is compiled.

\subsubsection*{Homebrew's FFTW3 on Mac OS X}
Calypso uses Fortran wrappers in FFTW3. If FFTW3 is installed using Homebrew for Mac OS X (\url{http://mxcl.github.com/homebrew/}), the required fortran wrappers are not installed. In this case, please install FFTW3 with Fortran wrappers with another package manager (Macports (\url{http://www.macports.org}, for example), build FFTW3 by yourself including the Fortran wrapper, or turn off FFTW3 features in Calypso.

\subsubsection*{XL fortran}
In XL fortran, preprocessor options is not specified by \verb|-D...|, but \verb|-Wf, '-D...'|. Please edit preprocessor macro option \verb|F90CPPFLAGS| in \verb|work/Makefile| by an editor.

\subsubsection*{Cross compiler support}
{\tt configure} command in Calypso does not support cross compilation. If you want to compile with a cross compiler, please set the variables in Makefile manually (see section \ref{section:no_configure})

\subsection{Directories}

The top directory of Calypso (ex. \verb|[CALYPSO_HOME]|) contains the following directories.
\begin{verbatim}
% cd [CALYPSO_HOME]
% ls
CMakeLists.txt	Makefile.in	configure.in	examples
INSTALL		bin		doc		src
LICENSE		configure	doxygen		work

\end{verbatim}

\begin{description}
\item{\verb bin:      } directory for executable files
\item{\verb cmake:    } directory for cmake configurations
\item{\verb cmake:    } directory for document generated by doxygen
\item{\verb doc:      } documentations
\item{\verb examples: } examples
\item{\verb src:      } source files
\item{\verb work:     } work directory. Compile is done in this directory.
\end{description}

\subsection{Doxygen}
Doxygen (\url{http://www.doxygen.org}) is an powerful document generation tool from source files. We only save a configuration file in this directory because thousands of html files generated by doxygen. The documents for source codes are generated by the following command:
% 
\begin{verbatim}
% cd [CALYPSO_HOME]/doxygen
% doxygen ./Doxyfile_CALYPSO
\end{verbatim}
%
The html documents can see by opening \verb|[CALYPSO_HOME]/doxygen/html/index.html|.  Automatically generated documentation is also available on the CIG website at \url{http://www.geodynamics.org/cig/software/calypso/}.

\subsection{Install using {\tt configure} command }
\subsubsection{Configuration using {\tt configure} command }
Calypso uses the configure script for configuration to install. The simplest way to install programs is the following process in the top directory of Calypso.
% 
\begin{verbatim}
%pwd
[CALYPSO_HOME]
% ./configure
...
% make
...
% make install
\end{verbatim}
%
After the installation, object modules can be deleted by the following command;
% 
\begin{verbatim}
% make clean
\end{verbatim}
%

{./configure} generates a Makefile in the current directory.  Available options for {\tt configure} can be checked using the {\tt ./configure --help} command. The following options are available in the {\tt configure} command.
%
{\small
\begin{verbatim}
Optional Features:
  --disable-option-checking  ignore unrecognized --enable/--with options
  --disable-FEATURE       do not include FEATURE (same as --enable-FEATURE=no)
  --enable-FEATURE[=ARG]  include FEATURE [ARG=yes]
  --enable-fftw3          Use fftw3 library 
 Optional Packages:
  --with-PACKAGE[=ARG]    use PACKAGE [ARG=yes]
  --without-PACKAGE       do not use PACKAGE (same as --with-PACKAGE=no)
  --with-hdf5=yes/no/PATH full path of h5pcc for parallel HDF5 configuration
  --with-blas=<lib>       use BLAS library <lib>
  --with-zlib=DIR root directory path of zlib installation defaults to
                    /usr/local or /usr if not found in /usr/local
  --without-zlib to disable zlib usage completely

Some influential environment variables:
  CC          C compiler command
  CFLAGS      C compiler flags
  LDFLAGS     linker flags, e.g. -L<lib dir> if you have libraries in a
              nonstandard directory <lib dir>
  LIBS        libraries to pass to the linker, e.g. -l<library>
  CPPFLAGS    (Objective) C/C++ preprocessor flags, e.g. -I<include dir> if
              you have headers in a nonstandard directory <include dir>
  FC          Fortran compiler command
  FCFLAGS     Fortran compiler flags
  MPICC       MPI C compiler command
  MPIFC       MPI Fortran compiler command
  PKG_CONFIG  path to pkg-config utility
  CPP         C preprocessor
  FFTW3_CFLAGS
              C compiler flags for FFTW3, overriding pkg-config
  FFTW3_LIBS  linker flags for FFTW3, overriding pkg-config

\end{verbatim}
}
%
An example of usage of the configure command is the following;
\begin{verbatim}
% ./configure --prefix='/Users/matsui/local' \
? CFLAGS='-O -Wall -g' FCFLAGS='-O -Wall -g' \
? PKG_CONFIG_PATH='/Users/matsui/local/lib/pkgconfig' \
? --with-blas=yes --enable-fftw3 --with-zlib=/usr/local \
? --with-hdf5='/Users/matsui/local/bin/h5pcc'

\end{verbatim}

\subsubsection{Compile}
Compile is performed using the {\tt make} command. The Makefile in the top directory is used to generate another Makefile in the {\tt work} directory, which is automatically used to complete the compilation. The object file and libraries are compiled in the {\tt work} directory. Finally, the executive files are assembled in {\tt bin} directory. You should find the following programs in the {\tt bin} directory.
%
\begin{description}
\item{\verb gen_sph_grids:    }\\
 Preprocessing program for data transfer for spherical harmonics transform
\item{\verb check_sph_grids:    }\\
 Check program for data communication for spherical harmonics transform
\item{\verb sph_mhd:          }\\
 Simulation program
\item{\verb sph_initial_field: }\\
 Example program to generate initial field
\item{\verb sph_add_initial_field: }\\
 Example program to add initial field in existing spectum data
\item{\verb sph_snapshot:     }\\
 Data transfer from spectrum data to field data
\item{\verb sph_dynamobench:  }\\
 Data processing for dynamo benchmark test by Christensen {\it et. al.} (2002)
\item{\verb assemble_sph:     }\\
 Data transfer program to change number of subdomains.
\item{\verb sectioning:     }\\
 Generate cross section and isosurface from field data and FEM mesh data.
\item{\verb field_to_VTK:   }\\
 Data transfer program from field and FEM mesh data to VTK format.
\item{\verb psf_to_vtk:     }\\
 Data transfer program from section and isosurface data to VTK format.
\item{\verb t_ave_sph_mean_square:     }\\
 Time averaging program for the mean square data.
\item{\verb t_ave_picked_sph_coefs:     }\\
 Time averaging program for the picked spectrum data.
\item{\verb t_ave_nusselt:     }\\
Time averaging program for the Nusselt number data.
\item{\verb check_sph_grids:   }\\
                   Check program for tests.
\item{\verb make_f90depends:  }\\
 Program to generate dependency of the source code ({\tt make} command uses to generate {\tt work/Makefile})
\end{description}
%
The following library files are also made in {\tt work} directory.
%
\begin{description}
\item{\verb libcalypso.a:    } Calypso library
\item{\verb libcalypso_c.a:    } Calypso library from C sources
\item{\verb libfftpack.5d.a: } FFTPACK 5.1 library
\end{description}
%

\subsubsection{Clean}
The object and fortran module files in {\tt work} directory is deleted by typing
\begin{verbatim}
% make clean
\end{verbatim}
This command deletes files with the extension {\tt .o}, {\tt .mod}, {\tt .par}, {\tt .diag}, and {\tt ~}.

\subsubsection{Distclean}
To revert the files and directory to the original package, use make distclean as
\begin{verbatim}
% make distclean
\end{verbatim}

\subsubsection{Install}
 The executive files are copied to the install directory \verb|$(INSTDIR)/bin|. The install directory \verb|$(INSTDIR)| is defined in Makefile, and can also set by  \verb|${--prefix}| option for \verb|configure| command. Alternatively, you can use the programs in \verb|${SRCDIR}/bin| directory without running \verb|make install|. If directory \verb|${PREFIX}| does not exist, \verb|make install | creates  \verb|${PREFIX}|,  \verb|${PREFIX}/lib|,  \verb|${PREFIX}/bin|, and  \verb|${PREFIX}/include| directories. No files are installed in \verb|${PREFIX}/lib| and \verb|${PREFIX}/include|.

\subsubsection{Construct dependecies (only for developper)}
Fortran90 routines need to be build after modules which are used in the routines. C source files also need dependency among include files. Consequently, list of dependency of source files are saved in the file \verb|Makefile.depends| in each directory. When you modify the source files with changing the module usage,  \verb|Makefile.depends| files need to be updated. To update the  \verb|Makefile.depends|files, use the  \verb|make| command at the \verb|[CALYPSO_HOME]| directory as \\
%
\begin{verbatim}
% make depends
\end{verbatim}

This process generate dependencies of the Fortran modules by program \verb|make_f90depends|. For C source files, the dependency is generated by the gcc with \verb|-MM -w -DDEPENDENCY_CHECK| option. Consequently, the dependencies need to be generated by the environment with gcc or compatible compiler. After generating the dependency, you can transfer the modified package and build without using gcc.

\subsection{Install without using configure}
\label{section:no_configure}
It is possible to compile Calypso without using the \verb|configure| command. To do this, you need to edit the \verb|Makefile|. First, copy \verb|Makefile| from template \verb|Makefile.in| as
%
\begin{verbatim}
% cp Makefile.in Makefile
\end{verbatim}
In Makefile, the following variables should be defined.
%
\begin{description}
\item{\verb|SHELL|}    Name of shell command.
\item{\verb|SRCDIR|}   Directory of this Makefile. 
\item{\verb|INSTDIR|}  Install directory.
\item{\verb|MPICHDIR|} Directory names for MPI implementation. If you set fortran90 compiler name for MPI programs in \verb|MPIF90|, you do not need to define this valuable.
\item{\verb|MPICHINCDIR|} Directory names for include files for MPI implementation. If you set fortran90 compiler name for MPI programs in \verb|MPIF90|, you do not need to define this valuable.
\item{\verb|MPILIBS|}   Library names for MPI implementation. If you set fortran90 compiler name for MPI programs in \verb|MPIF90|, you do not need to define this valuable.
\item{\verb|F90_LOCAL|} Command name of local Fortran 90 compiler to compile module dependency listing program.
\item{\verb|MPIF90|} Command name of Fortran90 compiler and linker for MPI programs. If command does not have MPI implementation, you need to define the definition of MPI libraries \verb|MPICHDIR|, \verb|MPICHINCDIR|, and \verb|MPILIBS|.
\item{\verb|AR|}     Command name for archive program (ex. \verb|ar|) to generate libraries. If you need some options for archive command, options are also included in this valuable.
\item{\verb|RANLIB|} Command name for \verb|ranlib| to generate index to the contents of an archive. If system does not have \verb|ranlib|, set \verb|true| in this valuable. \verb|true| command does not do anything for libraries.
\item{}
\item{\verb|F90OPTFLAGS|}  Optimization flags for Fortran90 compiler (including OpenMP flags)
\item{\verb|BLAS_LIBS|} Library lists for BLAS  (ex.  \verb|-lblas|)
\item{\verb|ZLIB_CFLAGS|} Option flags for FFTW3  (ex.  \verb|-I/usr/include|)
\item{\verb|ZLIB_LIB|}   Library lists for FFTW3 (ex. \verb|-L/usr/lib -lz|)
\item{\verb|FFTW3_CFLAGS|} Option flags for FFTW3  (ex.  \verb|-I/usr/local/include|)
\item{\verb|FFTW3_LIBS|}   Library lists for FFTW3 (ex. \verb|-L/usr/local/lib -lfftw3 -lm|)
\item{\verb|HDF5_FFLAGS|}  Option flags to compile with HDF5. This setting can be found by using hfd5 command \verb|h5pfc -show|.

\item{\verb|HDF5_LDFLAGS|}    Option flags  to link with  HDF5. This setting can be found by using hfd5 command \verb|h5pfc -show|.

\item{\verb|HDF5_FLIBS|}   Library lists for HDF5. This setting can be found by using hfd5 command \verb|h5pfc -show|.

\end{description}
%

\subsection{Install using cmake}
CMake is a cross-platform, open-source build system. CMake can be downloaded from \url{http://www.cmake.org}. The following procedure is required to install.
%
\begin{enumerate}
\item Create working directory (you can also use \verb|[CALYPSO_HOME]/work|).
\item Generate Makefile and working directories by {\tt cmake} command.
\item Compile programs by {\tt make} command.
\end{enumerate}
%
In this section, \verb|[CALYPSO\_HOME]/work| is used as the working directory.
Options for CMake can be checked by \verb|cmake -i [CALYPSO_HOME]| command at \verb|[CALYPSO_HOME]| \\
\verb|/work|. There are a number of options can be found, but the following valuables are important settings for installation:
%
\begin{itemize}
\item Install directory
\begin{description}
\item{\verb|CMAKE_INSTALL_PREFIX|}  \\
Install directory
\end{description}

\item Compiler settings
\begin{description}
\item{\verb|CMAKE_Fortran_COMPILER|} \\
Fortran90 compiler.
\item{\verb|CMAKE_c_COMPILER|} C compiler.

\item{\verb|CMAKE_Fortran_FLAGS|} \\
Optimization flags for Fortran90 compiler.
\item{\verb|CMAKE_c_FLAGS|} \\
Optimization flags for C compiler.
\end{description}

\item Option settings
\begin{description}
\item{\verb|CMAKE_DISABLE_FIND_PACKAGE_OpenMP_Fortran|} \\
OpenMP is not used if 'yes' is set in this valuable.
\item{\verb|CMAKE_DISABLE_FIND_PACKAGE_BLAS|}  \\
BLAS library is not linked if 'yes' is set in this valuable.
\item{\verb|CMAKE_DISABLE_FIND_PACKAGE_FFTW|}  \\
FFTW3 library is not linked if 'yes' is set in this valuable.
\item{\verb|CMAKE_DISABLE_FIND_PACKAGE_ZLIB|}  \\
Zlib library is not linked if 'yes' is set in this valuable.
\item{\verb|CMAKE_DISABLE_FIND_PACKAGE_HDF5|}  \\
HDF5 library is not linked if 'yes' is set in this valuable.
\end{description}

\item Manual settings for optional features
\begin{description}
\item{\verb|CMAKE_LIBRARY_PATH|}   \\
CMake library  search paths. This directory is used to search FFTW3 library.
\item{\verb|CMAKE_INCLUDE_PATH|}   \\
CMake include search paths. This directory is used to search include file for FFTW3.
\item{\verb|HDF5_INCLUDE_DIRS|}  \\
Include file directories to compile with HDF5. This setting can be found by using hfd5 command \verb|h5pfc -show|.
\item{\verb|HDF5_LIBRARY_DIRS|}    \\
Location of HDF5 library. This setting can be found by using hfd5 command \verb|h5pfc -show|.
\item{\verb|HDF5_LIBRARIES|}   \\
Library lists for HDF5. This setting can be found by using hfd5 command \verb|h5pfc -show|.
\end{description}
%
\end{itemize}
%
The easiest example of using CMake on Mac OS X with gcc9 is the following:
 \begin{verbatim}
% cd build
% cmake ~/CALYPSO/ -DCMAKE_Fortran_COMPILER=/opt/local/bin/gfortran-mp-9 \
? -DCMAKE_c_COMPILER=/opt/local/bin/gcc-mp-9 \
? -DCMAKE_Fortran_FLAGS="-O3 -g" -DCMAKE_c_FLAGS="-O3"
\end{verbatim}
%
After configuration, compile and install are started by
 \begin{verbatim}
% make
...
% make install
\end{verbatim}
%

After running \verb|make| command, execute files are built in \verb|[CALYPSO_HOME]/work/bin| directory.


\section{Simulation procedure}
Calypso consists of programs shown in Table \ref{table:controls}.
%
\begin{table}[htp]
\caption{List of program and required control file name}
\begin{center}
\begin{tabular}{|c|c|c|}
\hline
Program & Control file name & Type \\ \hline
\verb|gen_sph_grids|          & \verb|control_sph_shell| &  Parallel    \\
\verb|check_sph_grids|        & \verb|control_sph_shell| &  Parallel    \\ \hline
\verb|sph_mhd|                & \verb|control_MHD| &        Parallel  \\ \hline
\verb|sph_initial_field|      & \verb|control_MHD| &        Parallel  \\
\verb|sph_add_initial_field|  & \verb|control_MHD| &        Parallel  \\
\verb|assemble_sph|           & \verb|control_assemble_sph| & Parallel  \\ \hline
\verb|sph_snapshot|           & \verb|control_snapshot| &   Parallel  \\
\verb|sph_dynamobench|        & \verb|control_snapshot| &   Parallel  \\ \hline
\verb|sectioning|             & \verb|control_viz| &        Parallel  \\
\verb|field_to_VTK|           & \verb|control_viz| &        Parallel  \\
\verb|psf_to_vtk|             & N/A & Serial  \\ \hline
% \verb|sph_ene_check|          & N/A & Serial  \\
\verb|t_ave_sph_mean_square|  & N/A & Serial  \\
\verb|t_ave_picked_sph_coefs| & N/A & Serial  \\
\verb|t_ave_nusselt|          & N/A & Serial  \\ \hline
% \verb|make_f90depends|        & N/A & Serial  \\ \hline
\end{tabular}
\end{center}
\label{table:controls}
\end{table}
%

Because the serial programs do not use MPI, they are simply invoked by
%
\begin{verbatim}
% [program]
\end{verbatim}
%
Parallel programs must be invoked using MPI commands. On a Linux cluster using MPICH, parallel programs are invoked with 
%
\begin{verbatim}
% mpirun -np [# of processes] [program]
\end{verbatim}
%

This command will vary depending on the MPI implementation installed on the machine.  Please consult with your sysadmin for details.

%
To perform simulations by Calypso, the following processes are required.
%
\begin{enumerate}
\item Generate grids and spherical harmonics indexing information by \\
\verb|gen_sph_grids| (if necessary).
\item Make initial fields by \verb|sph_initial_field| (if necessary). 
\item Perform the simulation by \verb|sph_mhd|.
\item Convert the parallel spectra data by \verb|assemble_sph| to continue with changing number of processes (if necessary).
\item Data analysis by \verb|sph_snapshot|, \verb|sph_snapshot|, or \verb|sph_dynamobench|.
\item Update initial fields by \verb|sph_add_initial_field| for more simulations (if necessary).
\item Evaluate time averages by \verb|t_ave_sph_mean_square|, \verb|t_ave_picked_sph_coefs|, or \verb|t_ave_nusselt| (if necessary).
\end{enumerate}
%
The simulation program \verb|sph_mhd| requires an indexing file for spherical transform.  \verb|sph_mhd| generates spectrum data and monitoring data, and field data in Cartesian coordinate as outputs. The data transform programs (\verb|sph_snapshot| and \verb|sph_zm_snapshot|) generate outputs data  from parallel spectra data. The flow of data is shown in Figure \ref{fig:flow_0}. 
%
\begin{figure}[H]
\begin{center}
\includegraphics*[width=130mm]{images/flow_0}
\end{center}
\caption{Data flow of the simulation. Simulations require index data for spherical harmonics transform, initial spectra (optional) data, and FEM mesh data. Simulation program also outputs spectra data, monitoring data and  field data in Cartesian coordinate. Data transform program generates output data for simulation program from spectra data.}
\label{fig:flow_0}
\end{figure}
%

Each program needs one control file, the name of which is defined by the program.
 (Standard input is not supported by Fortran 90 so Calypso uses control files.)
The appropriate control file names are shown in the Table \ref{table:controls}. 
The following rules are used in the control files. An example of a control file is shown in Figure \ref{fig:control_example}.
%
\begin{itemize}
\item Lines starting with `\verb|#|' or `\verb|!|' are treated as a comment lines and ignored. 
\item All control files consist of blocks which start with `\verb|begin [name]|' and end with `\verb|end [name]|'.
\item The item name is shown first and the associated value/data is second.
\item The order of items and blocks can be changed.
\item If an item consists of multiple data, these should be listed in one line.
\item If an item does not belong in the block it is ignored.
\item Some blocks can read from external file. The external file name is defined by `\verb|file [block name] [external file name]|'. The following blocks can be excluded as a external file:
	\begin{itemize}
	\item \hyperref[href_t:spherical_shell_ctl]{\tt spherical\_shell\_ctl}
	\item \hyperref[href_t:cross_section_ctl]{\tt cross\_section\_ctl}
	\item \hyperref[href_t:isosurface_ctl]{\tt isosurface\_ctl}
	\end{itemize}
%
\item An array block starts with `\verb|begin array [name]|' and ends with `\verb|end array [name]|'.
\item In Fortran program, character `\verb|/|' is recognized as an end of character valuable if text with `\verb|/|' ({\it e.g.} file prefix including file paths) is not enclosed by {\tt '} or {\tt "}.
\item Calypso's control file input is limited to 255 characters for each line.
\end{itemize}
%
\begin{figure}[htbp]
\begin{center}
%
\begin{verbatim}
begin MHD_control
  begin data_files_def
    debug_flag_ctl            'OFF'
    num_subdomain_ctl            4
!
    sph_file_prefix             'sph_lm31r48c_4/in'
    sph_file_fmt_ctl           'merged'
  end data_files_def
!
  begin spherical_shell_ctl
    begin num_grid_sph
      truncation_level_ctl     4
      ngrid_meridonal_ctl     12
      ngrid_zonal_ctl         24
!
      radial_grid_type_ctl   explicit
      array r_layer
        r_layer    1  0.5384615384615
        r_layer    2  0.5384615384615
        r_layer    3  1.038461538462
        r_layer    4  1.538461538462
      end array r_layer
!
    end num_grid_sph
  end spherical_shell_ctl
end MHD_control
\end{verbatim}
%
\caption{Example of Control file}
\label{fig:control_example}
\end{center}
\end{figure}
%

\newpage
\section{Examples} \label{section:examples}
Several examples are provided in the \verb|examples| directory. There are three subdirectories as examples. README files are also provided to perform these examples in each subdirectory.
%
\begin{description}
\item{\tt assemble\_sph}    Examples for assembling program of spectrum data. (see section \ref{section:assemble_sph})
\item{\tt dynamo\_benchmark} Examples for dynamo benchmark by Christensen {\it et. al.} (2001)
\item{\tt heat\_composition\_source} Examples for the heat and composition diffusion problem including source term )
\item{\tt heterogineous\_temp} Examples for the heat and composition diffusion problem including thermal and compositional heterogeneity at boundaries.)
\item{\tt spherical\_shell} Examples for preprocessing program (see Section \ref{section:gen_sph_grid})
\end{description}
%

\subsection{Examples for preprocessing program}
Four examples illustrate the use of the preprocessing program. The examples include
%
\begin{description}
\item{\tt Chebyshev\_points} \label{href_t:gen_Chebyshev} Example to generate indexing data using Chebyshev collocation points
\item{\tt equidistance} \label{href_t:gen_equidistance} Example to generate indexing data with equi-distance grid
\item{\tt explicitly\_defined} \label{href_t:gen_explicit} Example to generate indexing data with explicitly defined radial points
\item{\tt with\_inner\_core} \label{href_t:gen_w_innercore} Example to generate indexing data including inner core and external of the fluid shell.
\end{description}

The program \verb|gen_sph_grids| generate spherical harmonics indexing file under the directory defined by the file \verb|control_sph_shell|.
%
\subsection{Examples of dynamo benchmark}
\label{section:dynamobench}
There are four examples for simulations using dynamo benchmark test as following.
%
\begin{description}
\item{\tt Case\_0} Example of dynamo benchmark case 0 (Thermally driven convection without magnetic field)
\item{\tt Case\_1} Example of dynamo benchmark case 1 (Dynamo model with co-rotating and electrically insulated inner core)
\item{\tt Case\_2} Example of dynamo benchmark case 2 (Dynamo model with rotatable and conductive inner core)
\item{\tt Compositional\_case\_1} Example of dynamo benchmark case 1 using compositional variation instead of temperature
\end{description}
%
The process of the simulation in these examples is the same using 4 MPI processes:
%
\begin{enumerate}
\item Change to the directory for Benchmark Case 1 (for example) \\
{\small
\begin{verbatim}
[username]$ cd [CALYPSO_DIR]/examples/dynamo_benchmark/dynamobench_case1
\end{verbatim}
}

\item  Create the grid files for the simulation  \\
{\small
\begin{verbatim}
[dynamobench_case_1]$ [CALYPSO_DIR]/bin/gen_sph_grids
\end{verbatim}
}

\item  Create initial field (Benchmark Case 1 only, see section \ref{sec:sph_initial_field}) \\
{\small
\begin{verbatim}
[dynamobench_case_1]$ [CALYPSO_DIR]/bin/sph_initial_field
\end{verbatim}
}

\item  Run simulation program
{\small
\begin{verbatim}
[dynamobench_case_1]$ mpirun -np 4 [CALYPSO_DIR]/bin/sph_mhd
\end{verbatim}
}

\item  To continue the simulation, change the parameter \verb|rst_ctl| in \verb|control_MHD| from \verb|dynamo_benchmark_1| to \verb|start_from_rst_file| and continue simulation by repeating step 2.

\item  To check the results for dynamo benchmark, run 
{\small
\begin{verbatim}
[dynamobench_case_1]$ mpirun -np 4 [CALYPSO_DIR]/bin/sph_dynamobench
\end{verbatim}
}
\end{enumerate}
Each example has the following input and data outputs.
%
\subsubsection{Data files and directories for Case 0}
\label{section:bench_case0}
%
\begin{description}
\item{\tt control\_sph\_shell}	Control file for spherical shell preprocessing
\item{\tt  control\_MHD}		Control file for simulation
\item{\tt control\_snapshot}	Control file for postprocessing
\item{\tt sph\_lm31r48c\_4} 	Spherical shell indexing data directory
\item{\tt rst\_4}				Spectr data directory for restarting
\item{\tt field}				Field data directory for for visualization
\item{\tt setions}			Cross section data directory for for visualization
\end{description}
%
\subsubsection{Data files and directories for Case 1}
\label{section:bench_case1}
%
\begin{description}
\item{\tt control\_sph\_shell}	Control file for spherical shell preprocessing
\item{\tt  control\_MHD}		Control file for simulation
\item{\tt control\_snapshot}	Control file for postprocessing
\item{\tt control\_psf\_CMB}	Control file for section at CMB (See Section \ref{section:PSF})
\item{\tt control\_psf\_eq}		Control file for section at equatorial plane (See Section \ref{section:PSF})
\item{\tt control\_psf\_z0.3}	Control file for section at z = 0.3 (See Section \ref{section:PSF})
\item{\tt control\_psf\_s0.55}	Control file for cylindrical surface at s = 0.55 (See Section \ref{section:PSF})
\item{\tt control\_iso\_temp}	Control file for isosurface of temperature (See Section \ref{section:ISO})
\item{\tt sph\_lm31r48c\_4} 	Spherical shell indexing data directory
\item{\tt rst\_4}				Spectr data directory for restarting
\item{\tt field}				Field data directory for for visualization
\item{\tt setions}			Cross section data directory for for visualization (See Section \ref{section:PSF})
\item{\tt isourfaces}			Isosurface data directory for for visualization (See Section \ref{section:ISO})
\end{description}

 After running the program, the following files are written.
\begin{description}
\item{\tt sph\_pwr\_volume\_s.dat}   Mean square data over the fluid shell.
\end{description}

%
\subsubsection{Data files and directories for Case 2}
\label{section:bench_case2}
%
\begin{description}
\item{\tt control\_sph\_shell}	Control file for spherical shell preprocessing
\item{\tt control\_MHD}		Control file for simulation
\item{\tt control\_snapshot}	Control file for postprocessing
\item{\tt control\_psf\_CMB}	Control file for section at CMB (See Section \ref{section:PSF})
\item{\tt control\_psf\_ICB}		Control file for section at ICB (See Section \ref{section:PSF})
\item{\tt control\_psf\_eq}		Control file for section at equatorial plane (See Section \ref{section:PSF})
\item{\tt control\_psf\_z0.3}	Control file for section at z = 0.3  (See Section \ref{section:PSF})
\item{\tt control\_psf\_s0.55}	Control file for cylindrical surface at s = 0.55 (See Section \ref{section:PSF})
\item{\tt sph\_lm31r48c\_4} 	Spherical shell indexing data directory
\item{\tt rst\_4}				Spectr data directory for restarting
\item{\tt field}				Field data directory for for visualization
\item{\tt setions}			Cross section data directory for for visualization (See Section \ref{section:PSF})
\end{description}

 After running the program, the following files are written.
\begin{description}
\item{\tt sph\_pwr\_volume\_s.dat}   Mean square data over the fluid shell.
\end{description}
%
\subsubsection{Data files and directories for Compositional Case 1}
\label{section:bench_case1C}
%
\begin{description}
\item{\tt const\_sph\_initial\_spectr.f90}	Source code to generate initial field (need )
\item{\tt control\_sph\_shell}	Control file for spherical shell preprocessing
\item{\tt control\_MHD}		Control file for simulation
\item{\tt control\_snapshot}	Control file for postprocessing
\item{\tt sph\_lm31r48c\_4} 	Spherical shell indexing data directory
\item{\tt rst\_4}				Spectr data directory for restarting
\item{\tt field}				Field data directory for for visualization
\end{description}
%

\subsection{Example of data assembling program}
An example for spectrum data assembling program is provided in \verb|assemble_sph| directory. This example uses simulation results of dynamo benchmark case 1.
First, copy data from dynamo benchmark case 1 by using shell script as
%
\begin{verbatim}
[assemble_sph]$ sh copy_from_case1.sh
\end{verbatim}

Then, construct new domain decomposition data as 
%
\begin{verbatim}
[sph_lm31r48c_4]$ cd sph_lm31r48c_2
[sph_lm31r48c_2]$ mpirun -np 2 [CALYPSO_DIR]/bin/gen_sph_grids
[sph_lm31r48c_2]$ cd ../
\end{verbatim}

Finally restart data for new configuration is generated by \verb|assemble_sph| in \verb|2doamins| directory.
\begin{verbatim}
[sph_lm31r48c_2]$ mpirun -np 2 [CALYPSO_DIR]/bin/assemble_sph
\end{verbatim}


\subsection{Example of treatment of heat and composition source term}
This example solves heat and composition diffusion with including source terms. In this example, only temperature and composition are solved by
%
\begin{eqnarray}
\frac{\partial T}{\partial t} =  \kappa_{T} \nabla^{2} T + q_{T},
\nonumber \\
\frac{\partial C}{\partial t} = \kappa_{C} \nabla^{2} C + q_{C},
\nonumber
\end{eqnarray}
%
 In the present example, diffusivities are fixed to be $\kappa_{T} = \kappa_{C} = 1$. Heat and composition sources are given as
$q_{T} = \frac{2}{r}$ and $q_{C} = 1.0$, respectively. The source terms are given in the initial field data. The procedure of the simulation is the same as for the dynamo benchmark Case 1. However, initial field generation program \verb|sph_initial_field| is required to build by the following process:

\begin{enumerate}
\item Copy source file \verb|const_sph_initial_spectr.f90| to \\
 \verb|[CALYPSO_DIR]/src/programs/data_utilities/INITIAL_FIELD|.
\begin{verbatim}
$[sph_initial_field]$ INITIAL_FIELD
\end{verbatim}

\item Build initial field generation program again.
\begin{verbatim}
[sph_initial_field]$ cd [CALYPSO_DIR]/work
[work]$ make
\end{verbatim}

\item Return to the example directory
\begin{verbatim}
[work]$ cd [CALYPSO]/examples/heat_composition_source
\end{verbatim}

\end{enumerate}
After building \verb|sph_initial_field|, the procedure is the same as for the dynamo benchmarks. Aftrer the simulation, $Y_{0}^{0}$ component of temperature and composition as a function of radius and time is written in \verb|picked_mode.dat|.


 \subsection{Example of thermal and compositional boundary conditions by external file}
 Heterogeneous boundary are input using an external file. An example to set thermal and compositional boundary conditions is given in \verb|heterogineous_temp| directory. As in the heat source example, only the diffusion problem is solved in this example. In file \verb|bc_spectr.btx|, temperature boundary conditions are defined for $Y_{0}^{0}$, $Y_{1}^{1s}$, $Y_{1}^{1c}$, and ,$ Y_{2}^{2c}$ component, and compositional boundary is defined for $Y_{0}^{0}$, $Y_{2}^{2s}$, and $Y_{2}^{2c}$ components. The radial profile of these spherical harmonics coefficients are written in \verb|picked_mode.dat|.



%

\newpage
\section{Preprocessing program ({\tt gen\_sph\_grid})}
\label{section:gen_sph_grid}
%
The following programs are built by compile.
%
\begin{verbatim}
\end{verbatim}
%
\begin{figure}[htbp]
\begin{center}
\includegraphics*[width=130mm]{images/flow_1}
\end{center}
\caption{Generated files by preprocessing program in Data flow.}
\label{fig:gen_sph_grid}
\end{figure}
%
This program generates index table and a communication table for parallel spherical harmonics, table of integrals for Coriolis term, and FEM mesh information to generate visualization data (see Figure \ref{fig:gen_sph_grid}). This program needs control file for input. This program can perform with {\bf any} number of MPI processes less than the number of subdomains, but is required to run with the SAME number of subdomains to generate MERGED data . The output files include the indexing tables. 

%
\begin{table}[htp]
\caption{List of files for {\tt gen\_sph\_grid} }
\begin{center} 
\begin{tabular}{|c|c|c|c|}
\hline
Files & extension & Parallelization & I/O \\ \hline \hline
Control file & \verb|control_sph_grid| & Single & Input \\ \hline
Index for $(r,j)$ & \verb|[sph_prefix].[rj_extension]| & - & Output \\
Index for $(r,l,m)$ & \verb|[sph_prefix].[rlm_extension]| & - & Output \\
Index for $(r,t,m)$ & \verb|[sph_prefix].[rtm_extension]| & - & Output \\
Index for $(r,t,p)$ & \verb|[sph_prefix].[rtp_extension]| & - & Output \\ \hline
FEM mesh & \verb|[sph_prefix].[fem_extension]| & - & Output \\ \hline
Radial point list & \verb|radial_info.dat| & Single & Output \\ \hline
\end{tabular}
\end{center}
\label{table:gen_sph_grid}
\end{table}
%
%
\begin{table}[htp]
\caption{data format flag {[\tt sph\_file\_fmt\_ctl]} and extensions.}
\begin{center} 
\begin{tabular}{|c||c|c|c|c|}
\hline
  \multicolumn{5}{|c|}{Distributed files} \\ \hline
  \verb|[sph_file_fmt_ctl]| &  \verb|ascii| & \verb|binary| & \verb|gzip| & \verb|gzip_bin| \\ \hline
\verb|[rj_extension]|  & \verb|[#].rj|  & \verb|[#].brj| & \verb|[#].rj.gz|  & \verb|[#].brj.gz| \\
\verb|[rlm_extension]| & \verb|[#].rlm| & \verb|[#].blm| & \verb|[#].rlm.gz| & \verb|[#].blm.gz| \\
\verb|[rtm_extension]| & \verb|[#].rtm| & \verb|[#].btm| & \verb|[#].rtm.gz| & \verb|[#].btm.gz| \\
\verb|[rtp_extension]| & \verb|[#].rtp| & \verb|[#].btp| & \verb|[#].rtp.gz| & \verb|[#].btp.gz| \\ \hline
\verb|[fem_extension]| & \verb|[#].gfm| & \verb|[#].gfb| & \verb|[#].gfm.gz| & \verb|[#].gfb.gz| \\ \hline \hline
  \multicolumn{5}{|c|}{Single file}  \\ \hline
  \verb|[sph_file_fmt_ctl]| & \verb|merged| & \verb|merged_bin| & \verb|merged_gz| & \verb|merged_bin_gz| \\ \hline
\verb|[rj_extension]|   & \verb|.rj|  & \verb|.brj| & \verb|.rj.gz|  & \verb|.brj.gz| \\
\verb|[rlm_extension]| & \verb|.rlm| & \verb|.blm| & \verb|.rlm.gz| & \verb|.blm.gz| \\
\verb|[rtm_extension]| & \verb|.rtm| & \verb|.btm| & \verb|.rtm.gz| & \verb|.btm.gz| \\
\verb|[rtp_extension]| & \verb|.rtp| & \verb|.btp| & \verb|.rtp.gz| & \verb|.btp.gz| \\ \hline
\verb|[fem_extension]| & \verb|.gfm| & \verb|.gfb| & \verb|.gfm.gz| & \verb|.gfb.gz| \\ \hline \hline
  \multicolumn{5}{c}{{\tt [\#]} is the domain or process number} \\
\end{tabular}
\end{center}
\label{table:mesh_format}
\end{table}
%

My favorite (recommend) data format is \verb|merged_gz|, because we can reduce the number of files in one directory and check the data by using \verb|gunzip -c| command.
%

\subsection{Position of radial grid}
The preprocessing program sets the radial grid spacing, either by a list in the control file or by setting an equidistant grid or Chebyshev collocation points.

In equidistance grid, radial grids are defined by
%
\begin{eqnarray}
r(k) & = & r_{i} + \left(r_{o}-r_{i} \right) \frac{k-k_{ICB}}{N},
\nonumber
\end{eqnarray}
%
where, $k_{ICB}$ is the grid points number at ICB. The radial grid set from the closest points of minimum radius defined by \hyperref[href_i:Min_radius_ctl]{\tt [Min\_radius\_ctl]} in control file to the closest points of the maximum radius defined by \hyperref[href_i:Max_radius_ctl]{\tt [Max\_radius\_ctl]} in control file, and radial grid number for the innermost points is set to $k = 1$.

In Chebyshev collocation points, radial grids in the fluid shell are defined by
%
\begin{eqnarray}
r(k) & = & r_{i} + \frac{\left(r_{o}-r_{i} \right)}{2} \left[ \frac{1}{2} - \cos \left(\pi \frac{ k-k_{ICB}}{N} \right) \right],
\nonumber
\end{eqnarray}
%
For the inner core ($r<r_{i}$), grid points is defined by
%
\begin{eqnarray}
r(k) & = & r_{i} - \frac{\left(r_{o}-r_{i} \right)}{2} \left[ \frac{1}{2} - \cos \left(\pi \frac{ k-k_{ICB}}{N} \right) \right],
\nonumber
\end{eqnarray}
%
and, grid points in the external of the shell ($r>r_{o}$) is defined by
%
\begin{eqnarray}
r(k) & = & r_{o} + \frac{\left(r_{o}-r_{i} \right)}{2} \left[ \frac{1}{2} - \cos \left(\pi \frac{ k-k_{CMB}}{N} \right) \right],
\nonumber
\end{eqnarray}
%
where, $k_{CMB}$ is the grid point number at CMB.

\subsection{Control file (\tt{control\_sph\_shell})}
Control files for Calypso consists of blocks starting and ending with \verb|begin| and \verb|end|, respectively. Entities with more than one components are defined between \verb|begin array| and \verb|end array| flags. The number of components of an array must be defined at \verb|begin array| line. If blocks to be defined in an external file, the external file name is defined by \verb|file| flag. 

\label{section:control_sph_shell}
Control file ({\tt control\_sph\_shell}) consists the following items. Detailed description for each item can be checked by clicking each item.
\\
%
\verb|spherical_shell_ctl|
\label{href_i:spherical_shell_ctl}
\\
\\
%
Block {\tt MHD\_control} (Top block of the control file)
	\begin{itemize}
	\item Block \hyperref[href_t:data_files_def]{\tt data\_files\_def}
%
		\begin{itemize} \label{href_i:data_files_def2}
		\item \hyperref[href_t:num_subdomain_ctl]{\tt num\_subdomain\_ctl    [Num\_PE]}
		\item \hyperref[href_t:sph_file_prefix]{\tt sph\_file\_prefix    [sph\_prefix]}
		\item \hyperref[href_t:sph_file_fmt_ctl]{\tt sph\_file\_fmt\_ctl    [sph\_format]}
		\end{itemize}
%
	\item File \hyperref[href_t:spherical_shell_ctl]{\tt spherical\_shell\_ctl    [resolution\_control]}
	\item or Block \hyperref[href_t:spherical_shell_ctl]{\tt spherical\_shell\_ctl}
%
		\begin{itemize}
		\item (Block \hyperref[href_t:FEM_mesh_ctl]{\tt FEM\_mesh\_ctl})
		\begin{itemize} \label{href_i:FEM_mesh_ctl}
		\item (\hyperref[href_t:FEM_mesh_output_switch]{\tt FEM\_mesh\_output\_switch [ON or OFF]})
		\end{itemize}
%
		\item Block \hyperref[href_t:num_domain_ctl]{\tt num\_domain\_ctl}
			\begin{itemize} \label{href_i:num_domain_ctl}
			\item \hyperref[href_t:num_radial_domain_ctl]{\tt num\_radial\_domain\_ctl [Ndomain]}
			\item \hyperref[href_t:num_horizontal_domain_ctl]{\tt num\_horizontal\_domain\_ctl [Ndomain]}
%
            \item {\color{magenta} Array \hyperref[href_t:num_domain_sph_grid]
				{\tt num\_domain\_sph\_grid    [Direction]    [Ndomain]} \\
				(Depricated)}
			\item {\color{magenta} Array \hyperref[href_t:num_domain_legendre]
				{\tt num\_domain\_legendre    [Direction]    [Ndomain]} \\
				(Depricated)}
			\item {\color{magenta} Array \hyperref[href_t:num_domain_spectr]
				{\tt num\_domain\_spectr      [Direction]    [Ndomain]} \\
				(Depricated)}
			\end{itemize}
%
		\item Block \hyperref[href_t:num_grid_sph]{\tt num\_grid\_sph}
			\begin{itemize} \label{href_i:num_grid_sph}
	        \item \hyperref[href_t:truncation_level_ctl]{\tt truncation\_level\_ctl	[Lmax]}
			\item \hyperref[href_t:ngrid_meridonal_ctl]{\tt ngrid\_meridonal\_ctl [Ntheta]}
			\item \hyperref[href_t:ngrid_zonal_ctl]{\tt ngrid\_zonal\_ctl [Nphi]}
%
			\item \hyperref[href_t:radial_grid_type_ctl]{\tt radial\_grid\_type\_ctl} \\
				\verb|[explicit, Chebyshev, or equi_distance]| \label{href_i:radial_grid_type_ctl}
%
			\item \hyperref[href_t:num_fluid_grid_ctl]{\tt num\_fluid\_grid\_ctl  [Nr\_shell]}
			\item \hyperref[href_t:fluid_core_size_ctl]{\tt fluid\_core\_size\_ctl  [Length]}
			\item \hyperref[href_t:ICB_to_CMB_ratio_ctl]{\tt ICB\_to\_CMB\_ratio\_ctl  [R\_ratio]}
			\item \hyperref[href_t:Min_radius_ctl]{\tt Min\_radius\_ctl  [Rmin]}    
				\label{href_i:Min_radius_ctl}
			\item \hyperref[href_t:Max_radius_ctl]{\tt Max\_radius\_ctl  [Rmax]}    
				\label{href_i:Max_radius_ctl}
%
\\
			\item Array \hyperref[href_t:r_layer]{\tt r\_layer  [Layer \#]  [Radius]}    
%
			\item Array \hyperref[href_t:boundaries_ctl]{\tt boundaries\_ctl  [Boundary\_name]  [Layer \#]}    
			\end{itemize}
		\end{itemize}
	\end{itemize}

If \verb|num_radial_domain_ctl| and \verb|num_horizontal_domain_ctl| are defined, the following arrays \verb|num_domain_sph_grid|, \verb|num_domain_legendre|, and \verb|num_domain_spectr| are not necessary. \\
(see \hyperref[href_t:gen_w_innercore]{example} \verb|spherical_shell/with_inner_core|)

The external file for resoultion and parallelization information \verb|[resolution_control]| needs the following contorl blocks:
%
	\begin{itemize}
	\item Block \hyperref[href_i:spherical_shell_ctl]{\tt spherical\_shell\_ctl}
		\begin{itemize}
		\item Block \hyperref[href_i:FEM_mesh_ctl]{\tt FEM\_mesh\_ctl}
		\item Block \hyperref[href_i:num_domain_ctl]{\tt num\_domain\_ctl}
		\item Block \hyperref[href_i:num_grid_sph]{\tt num\_grid\_sph}
		\end{itemize}
	\end{itemize}
%
\subsection{Spectrum index data}
\verb|gen_sph_grid| generates indexing table of the spherical transform. To perform spherical harmonics transform with distributed memory computers, data communication table is also included in these files. Calypso needs four indexing data for the spherical transform.
%
\begin{description}
\item{\verb|[sph_prefix].[rj_extension]|} Indexing table for spectrum data $f(r,l,m)$ to calculate linear terms. In program,  spherical harmonics modes $(l,m)$ is indexed by $j = l(l+1) + m$. The spectrum data are decomposed by spherical harmonics modes $j$. Data communication table for Legendre transform is included. The data also have the radial index of the ICB and CMB. Extension \verb|[rj_extension]| is listed in Table \ref{table:mesh_format}.
\item{\verb|[sph_prefix].[rlm_extension]|} Indexing table for spectrum data $f(r,l,m)$ for Legendre transform. The spectrum data are decomposed by radial direction $r$ and spherical harmonics order $m$. Data communication table to caricurate liner terms is included. Extension \verb|[rlm_extension]| is listed in Table \ref{table:mesh_format}.
\item{\verb|[sph_prefix].[rtm_extension]|} Indexing table for data $f(r,\theta,m)$ for Legendre transform. The data are decomposed by radial direction $r$ and spherical harmonics order $m$. Data communication table for backward Fourier transform is included. Extension \verb|[rtm_extension]| is listed in Table \ref{table:mesh_format}.
\item{\verb|[sph_prefix].[rtp_extension]|} Indexing table for data $f(r,\theta,m)$ for Fourier transform and field data $f(r,\theta,\phi)$. The data are decomposed by radial direction $r$ and meridional direction $\theta$. Data communication table for forward Legendre transform is included. Extension \verb|[rtp_extension]| is listed in Table \ref{table:mesh_format}.
\end{description}
%

\subsection{Finite element mesh data (optional)}
Calypso generates field data for visualization with XDMF or VTK format. To generate field data file, the preprocessing program generates FEM mesh data for each subdomain of spherical grid $(r,\theta,\phi)$ under the Cartesian coordinate $(x,y,z)$. The mesh data file is written based on GeoFEM (\url{http://geofem.tokyo.rist.or.jp}) mesh data format, which consists of each subdomain mesh and communication table among overlapped nodes. The extension of the mesh file is listed in Table \ref{table:mesh_format}. This mesh data is only used in the programs \hyperref[sec:sectioning]{\tt sectioning} and \verb|field_to_VTK|.

\subsection{Radial grid data}
The preprocessing program generates radius of each layer in \verb|radial_info.dat| if \verb|radial_grid_type_ctl| is set to \verb|Chebyshev| or \verb|equi_distance|. This file consists of blocks \verb|array r_layer| and \verb|array boundaries_ctl| for control file. This data may be useful if you want to modify radial grid spacing by yourself.

\subsection{How to define spatial resolution and parallelization?}
  Calypso uses spherical harmonics expansion method and in horizontal discretization and finite difference methods in the radial direction. In the spherical harmonics expansion methods, nonlinear terms are solved in the grid space while time integration and diffusion terms are solved in the spectrum space. We need to set truncation degree $l_{max}$ of the spherical harmonics and number of grids in the three direction $(N_{r}, N_{\theta}, N_{\phi})$ in the preprocessing program. The following condition is required (or recommended) for $l_{max}$ and $(N_{r}, N_{\theta}, N_{\phi})$. $l_{max}$ is defined by \verb|truncation_level_ctl|, and $N_{r}$ for the fluid shell (outer core) is defined by  \verb|num_fluid_grid_ctl|.  $N_{\theta}$ and $N_{\phi}$ is defined by \verb|ngrid_meridonal_ctl| and \verb|ngrid_zonal_ctl|, respectively.
%
\begin{itemize}
\item $N_{\phi} = 2 N_{\theta}$.
\item $N_{\theta}$ must be more than $l_{max}+1$, but
\item To eliminate aliasing in the spherical transform, $N_{\theta} \ge 1.5 \left( l_{max}+1 \right)$ is highly recommended.
\item $N_{\phi}$ should consists of products among power of 2, power of 3, and power of 5.
\end{itemize}
%
%
\begin{figure}[htbp]
\begin{center}
\includegraphics*[width=130mm]{images/parallelize}
\end{center}
\caption{Parallelization and data communication in Calypso in the case using 9 (3x3) processors. Data are decomposed in radial and meridional direction for nonlinear term evaluations, decomposed in radial and harmonic order for Legendre transform, and decomposed in spherical harmonics for linear calculations.}
\label{fig:parallelization}
\end{figure}
%
Calypso is parallelized 2-dimensionally and direction of the parallelization is changed in the operations in the spherical transform (See Figure \ref{fig:parallelization}). Two dimensional parallelization delivers many parallelize configuration. Here is the approach  how to find the best configuration:
%
\begin{itemize}
\item Maximum parallelization level in horizontal direction is $\left( l_{max} + 1 \right)  /2$,  and $N_{r}+1$ is the maximum level in radial direction.
\item Decompose number of radial points $N_{r}+1$ and truncation degree $\left( l_{max} + 1 \right) / 2 $ into prime numbers.
\item Decide number of MPI processes from the prime numbers.
\item Choose the number of decomposition in the radial and horizontal direction as close as possible.
\end{itemize}
% 
Here is an example for the case with $(N_{r}, l_{max}) = (89, 95)$. The maximum number of parallelization is $90 \times 48  = 4320$ processes.  $N_{r}+1$ and $\left( l_{max} + 1 \right)  /2$ can be decomposed into $90 = 2 \times 3^2 \times 5$ and $48 = 2^4 \times 3 $. Now, if 160 processes run is intended, $160 = 10 \times 16$ is the closest number of decompositions. Comparing with the prime numbers of the spatial resolution, radial and horizontal decomposition will be 10 and 16, respectively.

\newpage
\section{Simulation program ({\tt sph\_mhd})}
\label{section:sph_mhd}
%

The name of the simulation program is {\tt sph\_mhd}. This program requires {\tt control\_MHD} as a Control file. This program performs with the indexing file for spherical harmonics and Coriolis term integration file generated by the preprocessing program {\tt gen\_sph\_grid}.
%
\begin{figure}[htbp]
\begin{center}
\includegraphics*[width=130mm]{images/flow_2}
\end{center}
\caption{Data flow for the simulation program.}
\label{fig:flow_2}
\end{figure}
%
Data files for this program are listed in Table \ref{table:sph_mhd}. Indexing data for spherical harmonics which starting with \verb|[sph_prefix]| are obtained by the preprocessing program \verb|gen_sph_grid|. If these indexing data files do not exist, the spherical harmonics indexing data files are also generated by using information in \verb|spherical_shell_ctl| block. The boundary condition data file \verb|[boundary_data_name]| is optionally required if boundary conditions for temperature and composition are not homogenous.
%
\begin{table}[htp]
\caption{List of files for simulation {\tt sph\_mhd} }
\begin{center} 
\begin{tabular}{|c|c|c|}
\hline
 name & Parallelization & I/O \\ \hline \hline
\verb|control_MHD| & Serial & Input \\ \hline
\verb|[sph_prefix].[rj_extension]|  & - & Input / (Output) \\
\verb|[sph_prefix].[rlm_extension]| & - & Input / (Output) \\
\verb|[sph_prefix].[rtm_extension]| & - & Input / (Output) \\
\verb|[sph_prefix].[rtp_extension]| & - & Input / (Output) \\ \hline
\verb|[sph_prefix].[fem_extension]| & - & (Input / Output) \\ \hline
\verb|[boundary_data_name]| & Single & Input \\ \hline
\verb|[rst_prefix].[step #].[rst_extension]| &  - & Input/Output  \\ \hline
\verb|[vol_pwr_prefix]_s.dat| & Single & Output \\ \hline
\verb|[vol_pwr_prefix]_l.dat| & Single & Output \\
\verb|[vol_pwr_prefix]_m.dat| & Single & Output \\
\verb|[vol_pwr_prefix]_lm.dat| & Single & Output \\
\verb|[vol_ave_prefix].dat| & Single & Output \\ \hline
\verb|[layer_pwr_prefix]_s.dat| & Single & Output \\
\verb|[layer_pwr_prefix]_l.dat| & Single & Output \\
\verb|[layer_pwr_prefix]_m.dat| & Single & Output \\
\verb|[layer_pwr_prefix]_lm.dat| & Single & Output \\ \hline
\verb|[gauss_coef_prefix].dat| & Single & Output   \\
\verb|[picked_sph_prefix].dat| & Single & Output   \\ \hline
\verb|[nusselt_number_prefix].dat| & Single & Output   \\ \hline
\verb|[fld_prefix].[step#].[domain#].[extension]| &  - & Output  \\ \hline
\verb|[section_prefix].[step#].[extension]| &  Single & Output  \\
\verb|[isosurface_prefix].[step#].[extension]| &  Single & Output  \\ \hline
\end{tabular}
\end{center}
(Output): Marked files are generated if files do not exist.
\label{table:sph_mhd}
\end{table}
%
%
\begin{table}[htp]
\caption{data format flag {[\tt restart\_file\_fmt\_ctl]} and extensions for the restart file.}
\begin{center} 
\begin{tabular}{|c||c|c|c|c|}
\hline
  \multicolumn{5}{|c|}{Distributed files} \\ \hline
  \verb|[sph_file_fmt_ctl]| &  \verb|ascii| & \verb|binary| & \verb|gzip| & \verb|gzip_bin| \\ \hline
\verb|[rst_extension]| & \verb|[#].fst| & \verb|[#].fsb| & \verb|[#].fst.gz| & \verb|[#].fsb.gz| \\ \hline \hline
  \multicolumn{5}{|c|}{Single file}  \\ \hline
  \verb|[sph_file_fmt_ctl]| & \verb|merged| & \verb|merged_bin| & \verb|merged_gz| & \verb|merged_bin_gz| \\ \hline
\verb|[rst_extension]| & \verb|.fst| & \verb|.fsb| & \verb|.fst.gz| & \verb|.fsb.gz| \\ \hline \hline
  \multicolumn{5}{c}{{\tt [\#]} is the domain or process number} \\
\end{tabular}
\end{center}
\label{table:restart_format}
\end{table}
%
%
\newpage
\subsection{Control file}
The format of the control file \verb|control_MHD| is described below. The detail of each block is described in section \ref{section:def_control}. You can jump to detailed description by clicking each item. \\
\\
%
Block \verb|MHD_control|  (Top block of the control file)
\label{href_i:MHD_control}
%
\begin{itemize}
\item Block \hyperref[href_t:data_files_def]{\tt data\_files\_def}
	\label{href_i:data_files_def}
%
	\begin{itemize}
	\item \hyperref[href_t:num_subdomain_ctl]
			{\tt num\_subdomain\_ctl    [Num\_PE]}
	\item \hyperref[href_t:num_smp_ctl]
			{\tt num\_smp\_ctl    [Num\_Threads]}
	\item \hyperref[href_t:sph_file_prefix]
			{\tt sph\_file\_prefix    [sph\_prefix]}
	\item \hyperref[href_t:boundary_data_file_name]
		{\tt boundary\_data\_file\_name    [boundary\_data\_name]}
%
	\item \hyperref[href_t:restart_file_prefix]
		{\tt restart\_file\_prefix    [rst\_prefix])}
	\item \hyperref[href_t:field_file_prefix]
			{\tt field\_file\_prefix    [fld\_prefix]}
%
	\item \hyperref[href_t:sph_file_fmt_ctl]
			{\tt sph\_file\_fmt\_ctl    [sph\_format]}
	\item \hyperref[href_t:restart_file_fmt_ctl]
			{\tt restart\_file\_fmt\_ctl    [rst\_format]}
	\item \hyperref[href_t:field_file_fmt_ctl]
			{\tt field\_file\_fmt\_ctl    [fld\_format]}
	\end{itemize}
%
\item (File or Block \hyperref[href_i:spherical_shell_ctl]
			{\tt spherical\_shell\_ctl        [resolution\_control]})
	\begin{itemize}
	\item (Block \hyperref[href_i:FEM_mesh_ctl]
        {\tt FEM\_mesh\_ctl} See Section \ref{section:gen_sph_grid})
	\item (Block \hyperref[href_i:num_domain_ctl]
		{\tt num\_domain\_ctl} See Section \ref{section:gen_sph_grid})
	\item (Block \hyperref[href_i:num_grid_sph]
		{\tt num\_grid\_sph} See Section \ref{section:gen_sph_grid})
	\end{itemize}
%
\item Block \verb|model|
	\begin{itemize}
	\item Block \hyperref[href_t:phys_values_ctl]{\tt phys\_values\_ctl}
		\begin{itemize} \label{href_i:phys_values_ctl}
		\item Array \hyperref[href_t:nod_value_ctl]
			{\tt nod\_value\_ctl    [Field]  [Viz\_flag]  [Monitor\_flag]}
		\end{itemize}
%
	\item Block \hyperref[href_t:time_evolution_ctl]{\tt time\_evolution\_ctl}
		\begin{itemize} \label{href_i:time_evolution_ctl}
		\item Array \hyperref[href_t:time_evo_ctl]
			{\tt time\_evo\_ctl    [Field]}
		\end{itemize}
%
	\item Block \hyperref[href_t:boundary_condition]{\tt boundary\_condition}
		\begin{itemize} \label{href_i:boundary_condition}
		\item Array \hyperref[href_t:bc_temperature]
            {\tt bc\_temperature       [Group]  [Type]  [Value]}
		\item Array \hyperref[href_t:bc_velocity]
			{\tt bc\_velocity          [Group]  [Type]  [Value]}
		\item Array \hyperref[href_t:bc_composition]
			{\tt bc\_composition       [Group]  [Type]  [Value]}
		\item Array \hyperref[href_t:bc_magnetic_field]
			{\tt bc\_magnetic\_field    [Group]  [Type]  [Value]}
		\end{itemize}
%
	\item Block \hyperref[href_t:forces_define]{\tt forces\_define}
		\begin{itemize} \label{href_i:forces_define}
		\item Array \hyperref[href_t:force_ctl]{\tt force\_ctl    [Force]}
		\end{itemize}
%
	\item Block \hyperref[href_t:dimensionless_ctl]{\tt dimensionless\_ctl}
		\begin{itemize} \label{href_i:dimensionless_ctl}
		\item Array \hyperref[href_t:dimless_ctl]{\tt dimless\_ctl    [Name]  [Value]}
		\end{itemize}
%
	\item Block \hyperref[href_t:coefficients_ctl]{\tt coefficients\_ctl}
		\begin{itemize} \label{href_i:coefficients_ctl}
		\item Block \hyperref[href_t:thermal]{\tt thermal}
			\begin{itemize} \label{href_i:thermal}
			\item Array \hyperref[href_t:coef_4_termal_ctl]
					{\tt coef\_4\_termal\_ctl          [Name] [Power]}
			\item Array \hyperref[href_t:coef_4_t_diffuse_ctl]
					{\tt coef\_4\_t\_diffuse\_ctl      [Name] [Power]}
			\item Array \hyperref[href_t:coef_4_heat_source_ctl]
					{\tt coef\_4\_heat\_source\_ctl    [Name] [Power]}
			\end{itemize}
%
		\item Block \hyperref[href_t:momentum]{\tt momentum}
			\begin{itemize} \label{href_i:momentum}
			\item Array \hyperref[href_t:coef_4_velocity_ctl]
				{\tt coef\_4\_velocity\_ctl          [Name] [Power]}
			\item Array \hyperref[href_t:coef_4_press_ctl]
				{\tt coef\_4\_press\_ctl             [Name] [Power]}
			\item Array \hyperref[href_t:coef_4_v_diffuse_ctl]
                {\tt coef\_4\_v\_diffuse\_ctl         [Name] [Power]}
			\item Array \hyperref[href_t:coef_4_buoyancy_ctl]
				{\tt coef\_4\_buoyancy\_ctl          [Name] [Power]}
			\item Array \hyperref[href_t:coef_4_Coriolis_ctl]
				{\tt coef\_4\_Coriolis\_ctl          [Name] [Power]}
			\item Array \hyperref[href_t:coef_4_Lorentz_ctl]
				{\tt coef\_4\_Lorentz\_ctl           [Name] [Power]}
			\item Array \hyperref[href_t:coef_4_composit_buoyancy_ctl]
				{\tt coef\_4\_composit\_buoyancy\_ctl [Name] [Power]}
			\end{itemize}
%
		\item Block \hyperref[href_t:induction]{\tt induction}
			\begin{itemize} \label{href_i:induction}
			\item Array \hyperref[href_t:coef_4_magnetic_ctl]
				{\tt coef\_4\_magnetic\_ctl   [Name] [Power]}
			\item Array \hyperref[href_t:coef_4_m_diffuse_ctl]
				{\tt coef\_4\_m\_diffuse\_ctl [Name] [Power]}
			\item Array \hyperref[href_t:coef_4_induction_ctl]
				{\tt coef\_4\_induction\_ctl  [Name] [Power]}
			\end{itemize}
%
		\item Block \hyperref[href_t:composition]{\tt composition}
			\begin{itemize} \label{href_i:composition}
			\item Array \hyperref[href_t:coef_4_composition_ctl]
				{\tt coef\_4\_composition\_ctl         [Name] [Power]}
			\item Array \hyperref[href_t:coef_4_c_diffuse_ctl]
				{\tt coef\_4\_c\_diffuse\_ctl          [Name] [Power]}
			\item Array \hyperref[href_t:coef_4_composition_source_ctl]
				{\tt coef\_4\_composition\_source\_ctl [Name] [Power]}
			\end{itemize}
		\end{itemize}
%
%	\item Block \hyperref[href_t:gravity_define]{\tt gravity\_define}
%		\begin{itemize} \label{href_i:gravity_define}
%		\item \verb||
%				\hyperref[href_t:gravity_type_ctl]{\tt gravity\_type\_ctl    [Name]}
%		\end{itemize}
%
%	\item Block \hyperref[href_t:Coriolis_define]{\tt Coriolis_define}
%		\begin{itemize} \label{href_i:Coriolis_define}
%		\item Array \hyperref[href_t:rotation_vec]
%				{\tt rotation\_vec    [Direction] [Value]}
%		\end{itemize}
%
	\item Block \hyperref[href_t:temperature_define]{\tt temperature\_define}
		\begin{itemize} \label{href_i:temperature_define}
		\item \verb||
				\hyperref[href_t:ref_temp_ctl]{\tt ref\_temp\_ctl        [REFERENCE\_TEMP]}
		\item Block \hyperref[href_t:low_temp_ctl]{\tt low\_temp\_ctl}
			\begin{itemize}
			\item \hyperref[href_t:depth]      {\tt depth        [RADIUS]}
			\item \hyperref[href_t:temperature]{\tt temperature  [TEMPERATURE]}
			\end{itemize}
%
		\item Block \hyperref[href_t:high_temp_ctl]{\tt high\_temp\_ctl}
			\begin{itemize}
			\item \hyperref[href_t:depth]      {\tt depth        [RADIUS]}
			\item \hyperref[href_t:temperature]{\tt temperature  [TEMPERATURE]}
			\end{itemize}
		\end{itemize}
	\end{itemize}
%
\item Block \verb|control|
	\begin{itemize}
	\item Block \hyperref[href_t:time_step_ctl]{\tt time\_step\_ctl}
		\begin{itemize} \label{href_i:time_step_ctl}
		\item \hyperref[href_t:elapsed_time_ctl]
			{\tt elapsed\_time\_ctl        [ELAPSED\_TIME]}
		\item \hyperref[href_t:i_step_init_ctl]
			{\tt i\_step\_init\_ctl        [ISTEP\_START]}
		\item \hyperref[href_t:i_step_finish_ctl]
			{\tt i\_step\_finish\_ctl      [ISTEP\_FINISH]}
		\item \hyperref[href_t:i_step_check_ctl]
			{\tt i\_step\_check\_ctl       [ISTEP\_MONITOR]}
		\item \hyperref[href_t:i_step_rst_ctl]
			{\tt i\_step\_rst\_ctl         [ISTEP\_RESTART]}
		\item \hyperref[href_t:i_step_field_ctl]
			{\tt i\_step\_field\_ctl       [ISTEP\_FIELD]}
		\item \hyperref[href_t:i_step_sectioning_ctl]
			{\tt i\_step\_sectioning\_ctl  [ISTEP\_SECTION]}
		\item \hyperref[href_t:i_step_isosurface_ctl]
			{\tt i\_step\_isosurface\_ctl  [ISTEP\_ISOSURFACE]}
		\item \hyperref[href_t:dt_ctl]
			{\tt dt\_ctl                   [DELTA\_TIME]}
		\item \hyperref[href_t:time_init_ctl]
			{\tt time\_init\_ctl           [INITIAL\_TIME]}
		\end{itemize}
%
	\item Block \hyperref[href_t:restart_file_ctl]{\tt restart\_file\_ctl}
		\begin{itemize} \label{href_i:restart_file_ctl}
		\item \hyperref[href_t:rst_ctl]{\tt rst\_ctl      [INITIAL\_TYPE]}
		\end{itemize}
%
	\item Block \verb||
    		\hyperref[href_t:time_loop_ctl]{\tt time\_loop\_ctl}
		\begin{itemize} \label{href_i:time_loop_ctl}
		\item \hyperref[href_t:scheme_ctl]{\tt scheme\_ctl              [EVOLUTION\_SCHEME]}
		\item \hyperref[href_t:coef_imp_v_ctl]{\tt coef\_imp\_v\_ctl    [COEF\_INP\_U]}
		\item \hyperref[href_t:coef_imp_t_ctl]{\tt coef\_imp\_t\_ctl    [COEF\_INP\_T]}
		\item \hyperref[href_t:coef_imp_b_ctl]{\tt coef\_imp\_b\_ctl    [COEF\_INP\_B]}
		\item \hyperref[href_t:coef_imp_c_ctl]{\tt coef\_imp\_c\_ctl    [COEF\_INP\_C]}
		\item \hyperref[href_t:FFT_library_ctl]{\tt FFT\_library\_ctl   [FFT\_Name]}
		\item \hyperref[href_t:Legendre_trans_loop_ctl]
			{\tt Legendre\_trans\_loop\_ctl [Leg\_Loop]}
		\end{itemize}
%
	\end{itemize}
%
\item Block \hyperref[href_t:sph_monitor_ctl]{\tt sph\_monitor\_ctl}
	\begin{itemize} \label{href_i:sph_monitor_ctl}
	\item \hyperref[href_t:volume_average_prefix]
			{\tt volume\_average\_prefix        [vol\_ave\_prefix]}
	\item \hyperref[href_t:volume_pwr_spectr_prefix]
			{\tt volume\_pwr\_spectr\_prefix    [vol\_pwr\_prefix]}
	\item \hyperref[href_t:nusselt_number_prefix]
			{\tt nusselt\_number\_prefix        [nusselt\_number\_prefix]}
%
	\item Array \hyperref[href_t:volume_spectrum_ctl]{\tt volume\_spectrum\_ctl}
		\begin{itemize}
		\item Block \verb|volume_spectrum_ctl|
			\begin{itemize}
			\item \hyperref[href_t:volume_average_prefix]
				{\tt volume\_average\_prefix      [vol\_ave\_prefix]}
			\item \hyperref[href_t:volume_pwr_spectr_prefix]
				{\tt volume\_pwr\_spectr\_prefix  [vol\_pwr\_prefix]}
			\item \hyperref[href_t:inner_radius_ctl]
				{\tt inner\_radius\_ctl           [radius]}
			\item \hyperref[href_t:outer_radius_ctl]
				{\tt outer\_radius\_ctl           [radius]}
			\end{itemize}
		\end{itemize}
%
	\item Block \hyperref[href_t:layered_spectrum_ctl]{\tt layered\_spectrum\_ctl}
		\begin{itemize}
		\item \hyperref[href_t:layered_pwr_spectr_prefix]
				{\tt layered\_pwr\_spectr\_prefix         [layer\_pwr\_prefix]}
			\item Array \hyperref[href_t:spectr_layer_ctl]
				{\tt spectr\_layer\_ctl [Layer \#] }
		\end{itemize}
%
	\item Block \hyperref[href_t:gauss_coefficient_ctl]{\tt gauss\_coefficient\_ctl}
		\begin{itemize}
		\item \hyperref[href_t:gauss_coefs_prefix]
			{\tt gauss\_coefs\_prefix                [gauss\_coef\_prefix]}
		\item \hyperref[href_t:gauss_coefs_radius_ctl]
            {\tt gauss\_coefs\_radius\_ctl           [gauss\_coef\_radius]}
			\item Array \hyperref[href_t:pick_gauss_coefs_ctl]
                		{\tt pick\_gauss\_coefs\_ctl  [Degree]   [Order]}
		\item Array \hyperref[href_t:pick_gauss_coef_degree_ctl]
                    {\tt pick\_gauss\_coef\_degree\_ctl  [Degree]}
		\item Array \hyperref[href_t:pick_gauss_coef_order_ctl]
					{\tt pick\_gauss\_coef\_order\_ctl   [Order]}
		\end{itemize}
%
	\item Block \hyperref[href_t:pickup_spectr_ctl]{\tt pickup\_spectr\_ctl}
		\begin{itemize}
		\item \hyperref[href_t:picked_sph_prefix]
			{\tt picked\_sph\_prefix                    [picked\_sph\_prefix]|}
		\item Array \hyperref[href_t:pick_layer_ctl]
					{\tt pick\_layer\_ctl               [Layer \#]}
		\item Array \hyperref[href_t:pick_sph_spectr_ctl]
					{\tt pick\_sph\_spectr\_ctl         [Degree]  [Order]}
		\item Array \hyperref[href_t:pick_sph_degree_ctl]
					{\tt pick\_sph\_degree\_ctl         [Degree]}
		\item Array \hyperref[href_t:pick_sph_order_ctl]
					{\tt pick\_sph\_order\_ctl          [Order]}
		\end{itemize}
%
	\item Block \hyperref[href_t:mid_equator_monitor_ctl]
				{\tt mid\_equator\_monitor\_ctl}
		\begin{itemize}
		\item \hyperref[href_t:nphi_mid_eq_ctl]
				{\tt nphi\_mid\_eq\_ctl   [Nphi\_mid\_equator]}
				\label{href_i:nphi_mid_eq_ctl}
		\end{itemize}
	\end{itemize}
%
\item Block \hyperref[href_t:visual_control]{\tt visual\_control}
    \begin{itemize} \label{href_i:visual_control}
	\item \hyperref[href_t:i_step_sectioning_ctl]
        {\tt i\_step\_sectioning\_ctl  [ISTEP\_SECTION]}
    \item Array \hyperref[href_t:cross_section_ctl]{\tt cross\_section\_ctl}
		\begin{itemize}
        \item File or Block {\tt cross\_section\_ctl} \\
                            {\tt [section\_control\_file]} \\
								(See section \ref{section:section_control})
		\end{itemize}
%
    \item \hyperref[href_t:i_step_isosurface_ctl]
		{\tt i\_step\_isosurface\_ctl  [ISTEP\_ISOSURFACE]}
    \item Array \hyperref[href_t:isosurface_ctl]{\tt isosurface\_ctl}
		\begin{itemize}
		\item File or Block {\tt isosurface\_ctl} \\
                            {\tt [isosurface\_control\_file]} \\
								(See section \ref{section:isosurface_control})
		\end{itemize}
    \end{itemize}
%
\item Block \hyperref[href_t:dynamo_vizs_control]{\tt dynamo\_vizs\_control}
	\begin{itemize} \label{href_i:dynamo_vizs_control}
		\item File or Block \hyperref[href_t:zonal_mean_section_ctl]
							{\tt zonal\_mean\_section\_ctl} \\
							{\tt [zonal\_mean\_section\_control\_file]} \\
								(See section \ref{section:section_control})
		\item File or Block \hyperref[href_t:zonal_RMS_section_ctl]
							{\tt zonal\_RMS\_section\_ctl} \\
                            {\tt [zonal\_RMS\_section\_control\_file]} \\
                                (See section \ref{section:section_control})
%
		\item Block \hyperref[href_t:crustal_filtering_ctl]{\tt crustal\_filtering\_ctl}
			\begin{itemize}
				\item \hyperref[href_t:crustal_filtering_ctl]
						{\tt truncation\_degree\_ctl          [Degree]}
			\end{itemize}
	\end{itemize}
\end{itemize}
%
\verb|spherical_shell_ctl| block is required if spherical harmonics indexing files are not exist.
%
\subsection{Spectrum data for restarting}
Spectrum data is used for restarting data and generating field data by Data transform program \verb|sph_snapshot|, \verb|sph_zm_snapshot|, or \verb|sph_dynamobench|. This file is saved for each subdomain (MPI processes), then \verb|[step #]| and \verb|[domain #]| are added in the file name. The \verb|[step #]| is calculated by \verb|time step| / \verb|[ISTEP_RESTART]|. Data format is defined by \verb|[restart_file_fmt_ctl]| as shown in Table \ref{table:restart_format}.

\subsection{Thermal and compositional boundary condition data file}\label{sec:boundary_file}
Thermal and compositional heterogeneity at boundaries are defined by a external file named  \verb|[boundary_data_name]|. In this file, temperature, composition, heat flux, or compositional flux at ICB or CMB can be defined by spherical harmonics coefficients. To use boundary conditions in \verb|[boundary_data_name]|, file name is defined by \verb|boundary_data_file_name| column in control file, and boundary condition type \verb|[type]| is set to \verb|fixed_file| or \verb|fixed_flux_file| in \verb|bc_temperature| or \verb|bc_composition| column. By setting \verb|fixed_file| or \verb|fixed_flux_file| in control file, boundary conditions are copied from the file \verb|[boundary_data_name]|.

An example of the boundary condition file is shown in Figure \ref{fig:boundary_file}. As for the control file, a line starting from '\verb|#|' or '\verb|!|' is recognized as a comment line. In \verb|[boundary_data_name]|, boundary condition data is defined as following:
%
\begin{enumerate}
\item  Number of total boundary conditions to be defined in this file.
\item  Field name to define the first boundary condition
\item  Place to define the first boundary condition (\verb|ICB| or \verb|CMB|)
\item  Number of spherical harmonics modes for each boundary condition
\item  Spectrum data for the boundary conditions (degree $l$, order $m$, and harmonics coefficients)
\item  After finishing the list of spectrum data return to Step 2 for the next boundary condition
\end{enumerate}
%
If harmonics coefficients of the boundary conditions are not listed in item 5, 0.0 is automatically applied for the harmonics coefficients of the boundary conditions. So, only non-zero components need to be listed in the boundary condition file.

%
\begin{figure}[htbp]
\begin{center}
\begin{verbatim}
#
#  number of boundary conditions
      4
#
#   boundary condition data list
#
#    Fixed temperature at ICB
temperature
ICB
    3
  0  0   1.0E+00
  1  1   2.0E-01
  2  2   3.0E-01
#
#    Fixed heat flux at CMB
heat_flux
CMB
   2
  0  0    -0.9E+0
  1  -1    5.0E-1
#
#    Fixed composition flux at ICB
composite_flux
ICB
   2
  0  0    0.0E+00
  2  0   -2.5E-01
#
#    Fixed composition at CMB
composition
CMB
   2
  0   0   1.0E+00
  2  -2   5.0E-01
\end{verbatim}
\end{center}
\caption{An example of boundary condition file.}
\label{fig:boundary_file}
\end{figure}
%

\subsection{Field data for visualization}
\label{sec:VTK}
Field data is used for the visualization processes. Field data are written with XDMF format (\url{http://www.xdmf.org/index.php/Main_Page}), merged VTK, or distributed VTK format (\url{http://www.vtk.org/VTK/img/file-formats.pdf}). The output data format is defined by \verb|fld_format|. Visualization applications which we checked are listed in Table \ref{table:Viz_app}. Because the field data is written by using Cartesian coordinate $(x,y,z)$ system, coordinate conversion is required to plot vector field in spherical coordinate $(r, \theta, \phi)$ or cylindrical coordinate $(s,\phi, z)$. We will introduce a example of visualization process using ParaView in Section \ref{sec:paraview}. Field data also output merged ASCII or binary format including compression using zlib. These original formats have smaller file size than VTK format because of excluding grid information. Program \hyperref[sec:field_to_VTK]{\tt field\_to\_VTK} generates VTK file from FEM mesh data and field data.
%
\begin{table}[htp]
\caption{Checked visualization application}
\label{table:Viz_app}
\begin{center} 
\begin{tabular}{|c|c|c|}
\hline
Control flag & \verb|fld_format| & Application \\ \hline \hline
\verb|VTK| & Distributed VTK & ParaView \\ \hline
\verb|single_VTK| & Merged VTK & ParaView, VisIt, or Mayavi \\ \hline
\verb|VTK_gzip| & Compressed Distributed VTK & ParaView \\
 & & after expanding by {\tt gzip} \\ \hline
\verb|single_VTK_gz| & Compressed Merged VTK & ParaView, VisIt or Mayavi \\
 & &  after expanding by {\tt gzip} \\ \hline
\verb|single_HDF5| & XDMF   & ParaView, VisIt   \\ \hline
\verb|ascii| & Distributed ASCII & - \\
\verb|binary| & Distributed binary & - \\
\verb|gzip| & Distributed compressed ASCII & - \\
\verb|bin_gz| & Distributed compressed binary & - \\
\verb|merged| & Merged ASCII & - \\
\verb|merged_bin| & Merged binary & - \\
\verb|merged_gzip| & Merged compressed ASCII & - \\
\verb|merged_bin_gz| & Merged compressed binary & - \\ \hline

\end{tabular}
\end{center}
More informations about ParaView is in \url{https://www.paraview.org}. \\
More informations about VisIt is in \url{https://wci.llnl.gov/codes/visit/}. \\
More informations about Mayavi is in \url{http://mayavi.sourceforge.net/}. \\
\end{table}
%

\subsubsection{Distributed VTK data}
Distributed VTK data have the following advantage and disadvantages to use:
%
\begin{itemize}
\item Advantage
\begin{itemize} 
\item Faster output
\item No external library is required
\end{itemize}
\item Disadvantage
\begin{itemize} 
\item Many data files are generated
\item Total data file size is large
\item Only ParaView supports this format
\end{itemize}
\end{itemize}
%
Distributed VTK data consist files listed in Table \ref{table:parallel_vtk}. For ParaView, all subdomain data is read by choosing \verb|[fld_prefix].[step#].pvtk| in file menu.
%
\begin{table}[htp]
\caption{List of written files for distributed VTK format}
\begin{center} 
\begin{tabular}{|c|c|}
\hline
 name &  \\ \hline \hline
\verb|[fld_prefix].[step#].[domain#].vtk|  & VTK data for each subdomain  \\ \hline
\verb|[fld_prefix].[step#].pvtk| & Subdomain file list for Paraview  \\ \hline
\end{tabular}
\end{center}
\label{table:parallel_vtk}
\end{table}
%

\subsubsection{Merged VTK data}
Merged VTK data have the following advantage and disadvantages to use:
%
\begin{itemize}
\item Advantage
\begin{itemize} 
\item Merged field data is generated
\item No external library is required
\item Many applications support VTK format
\end{itemize}
\item Disadvantage
\begin{itemize} 
\item Very slow to output
\item Total data file size is large
\end{itemize}
\end{itemize}
%
Merged VTK data generate files listed in Table \ref{table:Merged_vtk}. 
%
\begin{table}[htp]
\caption{List of written files for merged VTK format}
\begin{center} 
\begin{tabular}{|c|c|}
\hline
 name &  \\ \hline \hline
\verb|[fld_prefix].[step#].vtk|  & Merged VTK data  \\ \hline
\end{tabular}
\end{center}
\label{table:Merged_vtk}
\end{table}
%

\subsubsection{Merged XDMF data}
Merged XDMF data have the following advantage and disadvantages to use:
%
\begin{itemize}
\item Advantage
\begin{itemize} 
\item Fastest output
\item Merged field data is generated
\item File size is smaller than the VTK formats
\end{itemize}
\item Disadvantage
\begin{itemize} 
\item Parallel HDF5 library should be required to use
\end{itemize}
\end{itemize}
%
Merged XDMF data generate files listed in Table \ref{table:XDMF}. For ParaView, all subdomain data is read by choosing \verb|[fld_prefix].solution.xdmf| in file menu.
%
\begin{table}[htp]
\caption{List of written files for XDMF format}
\begin{center} 
\begin{tabular}{|c|c|}
\hline
 name &  \\ \hline \hline
\verb|[fld_prefix].mesh.h5|  & HDF5 file for geometry data \\ \hline
\verb|[fld_prefix].[step#].h5|  &HDF5 file for field data   \\ \hline
\verb|[fld_prefix].solution.xdmf|  & HDF5 file lists to be read  \\ \hline
\end{tabular}
\end{center}
\label{table:XDMF}
\end{table}
%
\subsubsection{Calypso field data}
Calypso field data is based on the spectr data for restarting. The data is simply replaced from spherical harmonics coefficients to each component of field data in the cartesian coordinate. The file format flag \verb|[field_file_fmt_ctl]| and corresponding extensiton are showw in Table \ref{table:field_format}.
%
\begin{table}[htp]
\caption{Data format flag {[\tt field\_file\_fmt\_ctl]} and extensions for the field file.}
\begin{center} 
\begin{tabular}{|c||c|c|c|c|}
\hline
\multicolumn{5}{|c|}{Distributed files} \\ \hline
\verb|[field_file_fmt_ctl]| &  \verb|ascii| & \verb|binary| & \verb|gzip| & \verb|gzip_bin| \\ \hline
\verb|[extension]| & \verb|[#].fld| & \verb|[#].flb| & \verb|[#].fld| & \verb|[#].flb.gz| \\ \hline \hline
\multicolumn{5}{|c|}{Single file}  \\ \hline
\verb|[field_file_fmt_ctl]| & \verb|merged| & \verb|merged_bin| & \verb|merged_gz| & \verb|merged_bin_gz| \\ \hline
\verb|[extension]| & \verb|.fld| & \verb|.flb| & \verb|.fld| & \verb|.flb.gz| \\ \hline \hline
\multicolumn{5}{c}{{\tt [\#]} is the domain or process number} \\
\end{tabular}
\end{center}
\label{table:field_format}
\end{table}
%

%
\subsection{Cross section data (Parallel Surfacing module)}
\label{section:PSF}
Calypso can output cross section data for visualization with finer time increment than the whole domain data. The cross section data consist of triangle patches with VTK format, then data can be visualized by Paraview like as the whole field data. This cross sectioning module can output arbitrary quadrature surface, but plane, sphere, and cylindrical section would be useful for the geodynamo simulations.

To output cross sectioning, increment of the surface output data should be defined by \verb|i_step_sectioning_ctl| in \verb|time_step_ctl| block. And, array block \\ \verb|cross_section_ctl| in \verb|visual_control| section is required to define cross sections. Each \verb|cross_section_ctl| block defines one cross section. Each cross section can also define by an external file by specifying external file name with \verb|file| label.
%
The sections shown in Table \ref{table:section_list} are supported in the sectioning module. These surfaces are defined in the Cartesian coordinate.
\begin{table}[htp]
\caption{Supported cross sections}
\begin{center}
\begin{tabular}{|c|c|}
\hline
Surface type & equation \\ \hline
Quadrature surface 
 & $a x^2 + b y^2 + c z^2 + d y z + e z x + f x y + g x + h y + j z + k = 0$ \\
Plane surface 
& $a \left(x-x_{0} \right) + b \left(y-y_{0} \right) + c \left(z-z_{0} \right) = 0$ \\
 Sphere 
& $\left(x-x_{0} \right)^2 + \left(y-y_{0} \right)^2 + \left(z-z_{0} \right)^2 = r^2$  \\
 Ellipsoid 
& $\displaystyle{ \left(\frac{x-x_{0}}{a} \right)^2 + \left( \frac{y-y_{0}}{b} \right)^2 + \left( \frac{z-z_{0}}{c} \right)^2} = 1$ \\
\hline
\end{tabular}
\end{center}
\label{table:section_list}
\end{table}
%
The easiest approarch is using sections defined by quadrature function with ten coefficients from $a$ to $k$ in the control array \verb|coefs_ctl|.

A plane surface is defined by a normal vector $(a, b, c)$ and one point including the surface $(x_{0}, y_{0}, z_{0})$ in arrays \verb|normal_vector| and \verb|center_position|, respectively.

A sphere surface is defined by the position of the center $(x_{0}, y_{0}, z_{0})$ and radius $r$ in array \verb|center_position| and \verb|radius|, respectively.

An Ellipsoid surface is defined the position of the center $(x_{0}, y_{0}, z_{0})$ and length of the each axis $(a, b, c)$ in arrays  \verb|center_position| and \verb|axial_length|, respectively. If one component of the \verb|axial_length| is set to 0, surfacing module generate a Ellipsoidal tube along with the axis where \verb|axial_length| is set to 0.

Area for visualization can be defined by array \verb|chosen_ele_grp_ctl| by choosing \verb|outer_core|, \verb|inner_core|, and \verb|all|. Fields to display is defined in array \verb|output_field|. In array \verb|output_field|, field type in Table \ref{table:field_type} needs to defined. The same field can be defined more than once in array \verb|output_field| to output vector field in Cartesian coordinate and radial component, for example.
%
\begin{table}[htp]
\caption{List of field type for cross sectioning and isosurface module}
\label{table:field_type}
\begin{center} 
\begin{tabular}{|c|c|}
\hline
 Definition & Field type  \\ \hline \hline
 \verb|scalar| & scalar field  \\ \hline
 \verb|vector| & Cartesian vector field \\ \hline
 \verb|x| & $x$-component  \\ \hline
 \verb|y| & $y$-component  \\ \hline
 \verb|z| & $z$-component  \\ \hline
 \verb|radial| & radial ($r$-) component  \\ \hline
 \verb|theta| & $\theta$-component  \\ \hline
 \verb|phi| & $\phi$-component  \\ \hline
 \verb|cylinder_r| & cylindrical radial ($s$-) component  \\ \hline
 \verb|magnitude| & magnitude of vector  \\ \hline
\end{tabular}
\end{center}
\end{table}
%
\subsubsection{Control data} \label{section:section_control}
The format of the control file or block for cross sections is described below. The detail of each block is described in section \ref{section:def_control}.  \verb|cross_section_ctl| block can be read from an external file. To define the external file name, as \verb|file cross_section_ctl [file name]| in \verb|control_MHD| or \verb|control_snapshot|. \\
\\
%
Block \verb|cross_section_ctl| (Top level for sectioning)
\label{href_i:cross_section_ctl}
\begin{itemize}
	\item \hyperref[href_t:section_file_prefix]
			{\tt section\_file\_prefix    [section\_prefix]}
	\item \hyperref[href_t:psf_output_type]
			{\tt psf\_output\_type        [file\_format]}
	\item Block \hyperref[href_t:surface_define]{\tt surface\_define}
		\begin{itemize}
			\item \hyperref[href_t:section_method]
				{\tt section\_method    [METHOD]}
			\item Array \hyperref[href_t:psf_coefs_ctl]
				{\tt coefs\_ctl        [TERM]         [COEFFICIENT]}
			\item \hyperref[href_t:psf_radius]{\tt radius    [SIZE]}
			\item Array \hyperref[href_t:psf_normal_vector]
				{\tt normal\_vector    [DIRECTION]    [COMPONENT]}
			\item Array \hyperref[href_t:psf_axial_length]
                {\tt axial\_length     [DIRECTION]    [COMPONENT]}
			\item Array \hyperref[href_t:psf_center_position]
                {\tt center\_position  [DIRECTION]    [COMPONENT]}
%
			\item Array \hyperref[href_t:section_area_ctl]
				{\tt section\_area\_ctl        [AREA\_NAME]}
	\end{itemize}
%
	\item \hyperref[href_t:output_field_define]{\tt output\_field\_define}
		\begin{itemize}
			\item Array \hyperref[href_t:psf_output_field]
                {\tt output\_field     [FIELD]    [COMPONENT]}
		\end{itemize}
\end{itemize}

\subsubsection{Output data format of sectioning module}
\label{sec:PSF_data}
Sectioning data are written with VTK format and VTK data compressed by zlib. Field data also output by binary format and binary compressed by zlib. The list of data format and control flag for \verb|psf_output_type| are listed in Table \ref{table:PSF_data}. In the binary data format, position data and field data are saved independently not to write the grid data for each output step. Program \hyperref[section:psf_to_VTK]{\tt psf\_to\_VTK} generates VTK file from the binary section data. The output data format is defined by \verb|psf_output_type|. Because the field data is written by using Cartesian coordinate $(x,y,z)$ system, $(x,y,z)$ components in ParaView corresponds to the spherical components $(r, \theta, \phi)$ or cylindrical componennts $(s,\phi, z)$ if sectioning data is writtein the spherical or cylindrical componnents. Consequently, ParaView can not draw griph or field lines for these spherical or cylindrical vectors.
%
\begin{table}[htp]
\caption{Data format for sectioning data}
\label{table:PSF_data}
\begin{center} 
\begin{tabular}{|c|c|c|c|}
\hline
\verb|fld_format| & File format & \verb|extension| & Application \\ \hline \hline
\verb|VTK| & VTK & .vtk & ParaView \\ \hline
\verb|VTK_gzip| & Compressed VTK & .vtk.gz & ParaView \\
& & & after expanding by {\tt gzip}  \\ \hline
\verb|PSF| & Binary & 0.sgd (grid data) & - \\
           &        &  .sdt (field data) & - \\ \hline
\verb|PSF_gzip| & Compressed binary & .sgd.gz (grid data) & - \\
           &        &  .sdt.gz (field data) & - \\ \hline
\end{tabular}
\end{center} 
\end{table} 

%
\subsection{Isosurface data}
\label{section:ISO}
%
Calypso can also output isosurface data for visualization. Generally, data size of the isosurface is much larger than the sectioning data. The isosurface data is also written as a unstructured grid data with VTK format. The isosurface also consists of triangle patches.

To output cross sectioning, increment of the surface output data should be defined by \verb|i_step_isosurface_ctl| in \verb|time_step_ctl| block. And, array block \verb|isosurface_ctl| in \verb|visual_control| section is required to define cross sections. Each \verb|isosurface_ctl| block defines one cross section. Each cross section can also define by an external file by specifying external file name with \verb|file| label.
%
\subsubsection{Control data}  \label{section:isosurface_control}
The format of the control file or block for isosurfaces is described below. The detail of each block is described in section \ref{section:def_control}.  \verb|isosurface_ctl| block can be read from an external file. To define the external file name, as \verb|file isosurface_ctl [file name]| in \verb|control_MHD| or \verb|control_snapshot|. \\
\\
%
Block \verb|isosurface_ctl|  (Top lebel of the control data)
\label{href_i:isosurface_ctl}
\begin{itemize}
	\item \hyperref[href_t:isosurface_file_prefix]
			{\tt isosurface\_file\_prefix    [file\_prefix]}
	\item \hyperref[href_t:iso_output_type]
			{\tt iso\_output\_type           [file\_format]}
%
	\item Block \hyperref[href_t:isosurf_define]{\tt isosurf\_define}
		\begin{itemize}
			\item \hyperref[href_t:isosurf_field]{\tt isosurf\_field    [FIELD]}
			\item \hyperref[href_t:isosurf_component]
						{\tt isosurf\_component    [COMPONENT]}
			\item \hyperref[href_t:isosurf_value]{\tt isosurf\_value    [VALUE]}
%
			\item Array \hyperref[href_t:isosurf_area_ctl]
						{\tt isosurf\_area\_ctl  [AREA\_NAME]}
		\end{itemize}
%
	\item Block \hyperref[href_t:field_on_isosurf]{\tt field\_on\_isosurf}
		\begin{itemize}
			\item \hyperref[href_t:result_type]{\tt result\_type    [TYPE]}
			\item \hyperref[href_t:result_value]{\tt result\_value    [VALUE]}
			\item Array \hyperref[href_t:iso_output_field]
						{\tt output\_field        [FIELD]    [COMPONENT]}
		\end{itemize}
\end{itemize}

\subsubsection{Output data format of isosurface module}
\label{sec:PSF_data}
Isosurface data are written with VTK format and VTK data compressed by zlib. Field data also output by binary format and binary compressed by zlib. The list of data format and control flag for \verb|iso_output_type| are listed in Table \ref{table:ISO_data}. Like as sectioning data, program \hyperref[section:psf_to_VTK]{\tt psf\_to\_VTK} generates VTK file from the binary section data. The output data format is defined by \verb|iso_output_type|. Because the field data is written by using Cartesian coordinate $(x,y,z)$ system, $(x,y,z)$ components in ParaView corresponds to the spherical components $(r, \theta, \phi)$ or cylindrical componennts $(s,\phi, z)$ if sectioning data is writtein the spherical or cylindrical componnents. Consequently, ParaView can not draw griph or field lines for these spherical or cylindrical vectors.
%
\begin{table}[htp]
\caption{Data format for isosurface data}
\label{table:ISO_data}
\begin{center} 
\begin{tabular}{|c|c|c|c|}
\hline
\verb|fld_format| & File format & \verb|extension| & Application \\ \hline \hline
\verb|VTK| & VTK & .vtk & ParaView \\ \hline
\verb|VTK_gzip| & Compressed VTK & .vtk.gz & ParaView \\
& & & after expanding by {\tt gzip}  \\ \hline
\verb|ISO| & Binary & .sfm & - \\ \hline
\verb|ISO_gzip| & Compressed binary &  .sfm.gz & - \\ \hline
\end{tabular}
\end{center} 
\end{table} 



\subsection{Mean square amplitude data}
This program output mean square amplitude of the fields which is marked as \verb|Monitor_ON| over the fluid shell at every \verb|[increment_monitor]| steps. The data is written in the file \verb|[vol_pwr_prefix]_s.dat| or  \verb|sph_pwr_volume_s.dat| if  \verb|[vol_pwr_prefix]| is not defined in the control file. For vector fields, For the velocity $\bvec{u}$ and magnetic field $\bvec{B}$, the kinetic energy $1/2 u^{2}$ and magnetic energy $1/2 B^{2}$ are calculated instead of mean square amplitude. Labels on the first lines indicate following data. The data file have the following headers in the first 7 lines, and headers of the data and data are stored in the following lines. The header in the first 7 lines is the following. If these mean square amplitude data files exist before starting the simulation, programs append results at the end of files without checking constancy of the number of data and order of the field. If you change the configuration of data output structure, please move the existed data files to another directory before starting the programs.
%
\begin{description}
\item{\tt  line 2: } Number of radial grid and truncation level
\item{\tt  line 4: } radial layer ID for ICB and CMB
\item{\tt  line 6: } Number of field of data, total number of components
\item{\tt  line 7: } Number of components for each field
\end{description}
%
Labels for data indicates as
%
\begin{description}
\item{\tt  t\_step}  Time setp number
\item{\tt  time}   Time
\item{\tt  K\_ene\_pol}  Amplitude of poloidal kinetic energy
\item{\tt  K\_ene\_tor}  Amplitude of toroidal kinetic energy
\item{\tt  K\_ene}       Amplitude of total kinetic energy
\item{\tt  M\_ene\_pol}  Amplitude of poloidal magnetic energy
\item{\tt  M\_ene\_tor}  Amplitude of toroidal magnetic energy
\item{\tt  M\_ene}       Amplitude of total magnetic energy
\item{\tt  [Field]\_pol} Mean square amplitude of poloidal component of {\tt [Field]}
\item{\tt  [Field]\_tor} Mean square amplitude of toroidal component of {\tt [Field]}
\item{\tt  [Field]}      Mean square amplitude of {\tt [Field]}
\end{description}
%
\subsubsection{Volume average data}
Volume average data are written by defining {\tt volume\_average\_prefix} in control file. Volume average data are written in \verb|[vol_ave_prefix].dat| with same format as RMS amplitude data. If you need the sphere average data for specific radial point, you can use picked spectrum data for $l = m = 0$ at specific radius.

\subsubsection{Volume spectrum data}
Volume spectrum data are written by defining {\tt volume\_pwr\_spectr\_prefix} in control file. By defining {\tt volume\_pwr\_spectr\_prefix}, following spectrum data averaged over the fluid shell is written. Data format is the same as the volume mean square data, but degree $l$, order $m$, or meridional wave number $l-m$ is added in the list of data. \\
%
\begin{description}
\item{\tt [vol\_pwr\_prefix\_l.dat}  Volume average of mean square amplitude of the fields as a function of spherical harmonic degree $l$. For scalar field, the spectrum is
%
\begin{eqnarray}
f_{sq}(l) &=& \frac{1}{V} \sum_{m=-l}^{m=l} \int \left({f_{l}^{m}} \right)^{2} dV.
\nonumber
\end{eqnarray}
%
For vector field, spectrum for the poloidal and toroidal components are written by 
%
\begin{eqnarray}
B_{Ssq}(l) &=& \frac{1}{V} \sum_{m=-l}^{m=l} \int \left(\bvec{B}_{Sl}^{\ m} \right)^{2} dV,
\nonumber \\
B_{Tsq}(l) &=& \frac{1}{V} \sum_{m=-l}^{m=l} \int \left(\bvec{B}_{Tl}^{\ m} \right)^{2} dV.
\nonumber
\end{eqnarray}

If the vector field $\bvec{F}$ is not solenoidal (i.e. $\nabla \cdot \bvec{F} \neq 0$), The poloidal component of mean square data are included mean square field of the potential components as
%
\begin{eqnarray}
F_{Ssq}(l) &=& \frac{1}{V} \sum_{m=-l}^{m=l} \int \left[\left(\bvec{B}_{Sl}^{\ m} \right)^{2}
 + \left(-\nabla \phi_{Fl}^{\ m} \right)^{2} \right] dV.
\nonumber
\end{eqnarray}
%

%
\item{\tt [vol\_pwr\_prefix]\_m.dat} Volume average of mean square amplitude of the fields as a function of spherical harmonic order $m$. The zonal wave number is referred in this spectrum data. For scalar field, the spectrum is
\begin{eqnarray}
f_{sq}(m) &=& \frac{1}{V} \sum_{l=0}^{l=m} \int \left[ \left(f_{l}^{m} \right)^{2}
 + \left( f_{l}^{-m} \right)^{2} \right] dV.
\nonumber
\end{eqnarray}
For vector field, spectrum for the poloidal and toroidal components are written by 
\begin{eqnarray}
B_{Ssq}(m) &=& \frac{1}{V} \sum_{l=0}^{l=m} \int \left[\left(\bvec{B}_{Sl}^{\ m} \right)^{2} 
 + \left(\bvec{B}_{Sl}^{\ -m} \right)^{2}  \right] dV,
\nonumber \\
B_{Tsq}(m) &=& \frac{1}{V} \sum_{l=0}^{l=m} \int \left[\left(\bvec{B}_{Tl}^{\ m} \right)^{2} 
 + \left(\bvec{B}_{Tl}^{\ -m} \right)^{2}  \right] dV.
\nonumber
\end{eqnarray}

\item{\tt [vol\_pwr\_prefix]\_lm.dat} Volume average of mean square amplitude of the fields as a function of spherical harmonic order $n = l-m$. The wave number in the latitude direction is referred in this spectrum data. For scalar field, the spectrum is
\begin{eqnarray}
f_{sq}(n) &=& \frac{1}{V} \sum_{l=n}^{l=l-n} \int \left[ \left(f_{l}^{l-n}\right)^{2} + \left(f_{l}^{-l+n}\right)^2 \right] dV.
\nonumber
\end{eqnarray}
For vector field, spectrum for the poloidal and toroidal components are written by 
\begin{eqnarray}
B_{Ssq}(n) &=& \frac{1}{V} \sum_{l=n}^{l=l-n} \int  \left[\left(\bvec{B}_{Sl}^{\ l-n} \right)^{2} 
 + \left(\bvec{B}_{Sl}^{\ -l+n} \right)^{2}  \right]  dV,
\nonumber \\
B_{Tsq}(n) &=& \frac{1}{V} \sum_{l=n}^{l=l-n} \int \left[\left(\bvec{B}_{Tl}^{\ l-n} \right)^{2} 
 + \left(\bvec{B}_{Tl}^{\ -l+n} \right)^{2}  \right] dV.
\nonumber
\end{eqnarray}

\end{description}

%
\subsubsection{Layered spectrum data}
\label{section:layerd_spectr}
Spectrum data for the each radial position are written by defining {\tt layered\_pwr\_spectr\_prefix} in control file. By defining {\tt layered\_pwr\_spectr\_prefix}, following spectrum data averaged over the fluid shell is written. Data format is the same as the volume spectrum data, but radial grid point and radius of the layer is added in the list. The following files are generated. The radial points for output is listed in the array \verb|spectr_layer_ctl|. If \verb|spectr_layer_ctl| is not defined, mean square data at {\bf all} radial levels will be written. See example of \hyperref[section:dynamobench]{dynamo benchmark case 2}.
%
\begin{description}
\item{\tt [layer\_pwr\_prefix]\_s.dat} Surface average of mean square amplitude of the fields.
\item{\tt [layer\_pwr\_prefix]\_l.dat} Surface average of mean square amplitude of the fields as a function of spherical harmonic degree $l$ and radial grid id $k$. For scalar field, the spectrum is
\begin{eqnarray}
f_{sq}(k,l) &=& \frac{1}{S} \sum_{m=-l}^{m=l} \int \left({f_{l}^{m}} \right)^{2} dS.
\nonumber
\end{eqnarray}
For vector field, spectrum for the poloidal and toroidal components are written by 
\begin{eqnarray}
B_{Ssq}(k,l) &=& \frac{1}{S} \sum_{m=-l}^{m=l} \int \left(\bvec{B}_{Sl}^{\ m} \right)^{2} dS,
\nonumber \\
B_{Tsq}(k,l) &=& \frac{1}{S} \sum_{m=-l}^{m=l} \int \left(\bvec{B}_{Tl}^{\ m} \right)^{2} dS.
\nonumber
\end{eqnarray}

\item{\tt [layer\_pwr\_prefix]\_m.dat} Surace average of mean square amplitude of the fields as a function of spherical harmonic order $m$ and radial grid id $k$. The zonal wave number is referred in this spectrum data. For scalar field, the spectrum is
\begin{eqnarray}
f_{sq}(k,m) &=& \frac{1}{S} \sum_{l=m}^{l=L} \int \left[ \left(f_{l}^{m} \right)^{2}
 + \left( f_{l}^{-m} \right)^{2} \right] dS.
\nonumber
\end{eqnarray}
For vector field, spectrum for the poloidal and toroidal components are written by 
\begin{eqnarray}
B_{Ssq}(k,m) &=& \frac{1}{S} \sum_{l=m}^{l=L} \int \left[\left(\bvec{B}_{Sl}^{\ m} \right)^{2} 
 + \left(\bvec{B}_{Sl}^{\ -m} \right)^{2}  \right] dS,
\nonumber \\
B_{Tsq}(k,m) &=& \frac{1}{S} \sum_{l=m}^{l=L} \int \left[\left(\bvec{B}_{Tl}^{\ m} \right)^{2} 
 + \left(\bvec{B}_{Tl}^{\ -m} \right)^{2}  \right] dS.
\nonumber
\end{eqnarray}

\item{\tt [layer\_pwr\_prefix]\_lm.dat} Surface average of mean square amplitude of the fields as a function of spherical harmonic order $n = l-m$ and radial grid id $k$. The wave number in the latitude direction is referred in this spectrum data. For scalar field, the spectrum is
\begin{eqnarray}
f_{sq}(k,n) &=& \frac{1}{S} \sum_{l=n}^{l=L} \int \left[ \left(f_{l}^{l-n}\right)^{2} + \left(f_{l}^{-l+n}\right)^2 \right] dS.
\nonumber
\end{eqnarray}
For vector field, spectrum for the poloidal and toroidal components are written by 
\begin{eqnarray}
B_{Ssq}(k,n) &=& \frac{1}{S} \sum_{l=n}^{l=L} \int  \left[\left(\bvec{B}_{Sl}^{\ l-n} \right)^{2} 
 + \left(\bvec{B}_{Sl}^{\ -l+n} \right)^{2}  \right] dS,
\nonumber \\
B_{Tsq}(k,n) &=& \frac{1}{S} \sum_{l=n}^{l=L} \int \left[\left(\bvec{B}_{Tl}^{\ l-n} \right)^{2} 
 + \left(\bvec{B}_{Tl}^{\ -l+n} \right)^{2}  \right] dS.
\nonumber
\end{eqnarray}

\end{description}
%

\subsection{Volume average data {\tt [volume\_average\_prefix].dat}}
The volume average information is written in the file {\tt [volume\_average\_prefix].dat}.
The volume average is evaluated by the radial integration of $l = m = 0$ component of the spherical harmonics coefficients as
\begin{eqnarray}
f_{ave} &=& \frac{1}{V} \int f_{0}^{0}(r) 4 \pi r^{2} dr.
\end{eqnarray}
Consequently, volume average of the solenoidal vector field to be 0, but but average data for the solenoidal vector is also written in the data to share the data IO routine with other monitor data output. To ouytput the average value over the specific radial level, use spectrum monitor data output \hyperref[sec:pickup_spectr_ctl]{\tt pickup\_spectr\_ctl} with $l = m = 0$.

\subsection{Gauss coefficient data {\tt [gauss\_coef\_prefix].dat}}
This program output selected Gauss coefficients of the magnetic field. Gauss coefficients is evaluated for radius defined by \verb|[gauss_coef_radius]| every \verb|[increment_monitor]| steps. Gauss coefficients are evaluated by using poloidal magnetic field at CMB $B_{Sl}^{\ m}(r_{o})$ and radius defined by \verb|[gauss_coef_radius]| $r_{e}$ as
%
\begin{eqnarray}
g_{l}^{m} &=& \frac{l}{r_{e}^2} \left(\frac{r_{o}}{r_{e}}\right)^{l} B_{Sl}^{\ m}(r_{o}),
\nonumber \\
h_{l}^{m} &=& \frac{l}{r_{e}^2} \left(\frac{r_{o}}{r_{e}}\right)^{l} B_{Sl}^{\ -m}(r_{o}).
\nonumber
\end{eqnarray}
%
The data file has the following headers in the first three lines,
%
\begin{description}
\item{\tt  line 2: } Number of saved Gauss coefficients and reference radius.
\item{\tt  line 3: } Labels of Gauss coefficients data.
\end{description}
%
The data consists of time step, time, and Gauss coefficients for each step in one line. If the Gauss coefficients data file exist before starting the simulation, programs append Gauss coefficients at the end of files without checking constancy of the number of data and order of the field. If you change the configuration of data output structure, please move the old Gauss coefficients file to another directory before starting the programs.

\subsection{Spectrum monitor data {\tt [picked\_sph\_prefix].dat}}
\label{sec:pickup_spectr_ctl}
This program outputs spherical harmonics coefficients at specified spherical harmonics modes and radial points in single text file. Spectrum data marked \verb|[Monitor_On]| are written in our line for each spherical harmonics mode and radial point every \\
\verb|[increment_monitor]| steps. If the spectrum monitor data file exist before starting the simulation, programs append spectrum data at the end of files without checking constancy of the number of data and order of the field. If you change the configuration of data output structure, please move the old spectrum monitor file to another directory before starting the programs.

If a vector field $\bvec{F}$ is not a solenoidal field, $\bvec{F}$ is described by the spherical harmonics coefficients of the poloidal $F_{Sl}^{\ m}$, toroidal $F_{Tl}^{\ m}$, and potential $\varphi_{l}^{m}$ components as
\begin{eqnarray}
\bvec{F}(r, \theta, \phi) & = &  - \frac{1}{r^{2}}\frac{\partial \varphi_{0}^{0}}{\partial r} \hat{r}
 + \sum_{l=1}^{L} \sum_{m=-l}^{l} 
\left[\nabla \times \nabla \times \left( F_{Sl}^{\ m} \hat{r} \right) +  \nabla \times \left(F_{Tl}^{\ m}\right)
 - \nabla \left(\varphi_{l}^{m} Y_{l}^{m} \right)\right].
\nonumber
\end{eqnarray}
In Calypso, the following coefficients are written for the non-solenoidal vector.
\begin{description}
\item{\tt  $\verb|[field_name]_pol|$ : }
 $\left\{\begin{array}{ccr}
\displaystyle{
F_{Sl}^{\ m} - \frac{r^{2}}{l \left(l+1\right)} \frac{\partial \varphi_{l}^{m}}{\partial r} }
& \mbox{for} & \left (l \ne 0 \right)\\
\displaystyle{
 -r^{2} \frac{\partial \varphi_{0}^{0}}{\partial r}
} & \mbox{for} & \left (l = 0 \right)
\end{array}
\right.$
\item{\tt  $\verb|[field_name]_dpdr|$ : } 
$
\left\{
\begin{array}{ccr}
\displaystyle{
\frac{\partial F_{Sl}^{\ m}}{\partial r} - \varphi_{l}^{m}}
 & \mbox{for} & \left (l \ne 0 \right)\\
 0 & \mbox{for} & \left (l = 0 \right)
\end{array}
\right. $
\item{\tt  $\verb|[field_name]_tor|$ : }  $F_{Tl}^{\ m}$
\end{description}


\subsection{Nusselt number data {\tt [nusselt\_number\_prefix].dat}}
{\bf CAUTION: Nusselt number is not evaluated if heat source is exsist.}
The Nusselt number Nu at CMB and ICB is written for each step in one line. The Nusselt number is evaluated by
%
\begin{eqnarray*}
Nu = \frac{<\partial T / \partial r>}{\partial T_{diff}/ \partial r},
\end{eqnarray*}
where, $<\partial T / \partial r>$ and $T_{diff}$ are the horizontal average of the temperature gradient at ICB and CMB and diffusive temperature profile, respectively. $T_{diff}$ is evaluated without heat source, as
\begin{eqnarray*}
T_{diff} = \frac{r_{o}T_{o} - r_{i}T_{i}}{r_{o} - r_{i}}
    +  \frac{r_{o}r_{i}\left(T_{i} - T_{o}\right)}{r_{o} - r_{i}} \frac{1}{r}.
\end{eqnarray*}
%
This diffusive temperature profile is for the case without heat source in the fluid. If simulation is performed including the heat source, this data file does not written.
If the Nusselt number data file exist before starting the simulation, programs append spectrum data at the end of files without checking constancy. If you change the configuration of data output structure, please move the old spectrum monitor file to another directory before starting the programs.


\newpage
\section{Data transform program ({\tt sph\_snapshot})}
\label{section:sph_snapshot}
%
\begin{figure}[htbp]
\begin{center}
\includegraphics*[width=130mm]{images/flow_3}
\end{center}
\caption{Data flow for data transform program.}
\label{fig:flow_3}
\end{figure}
%
Simulation program outputs spectrum data as a whole field data. This program is made from simulation program by replacing from time integration routines to restart data input routine. Consequently, Input/Output files in Table \ref{table:sph_mhd} are the same for {\tt sph\_snapshot}, except for the required input restart data \verb|[rst_prefix].[step #].[rst_extension]|.
%
This program requires control file \verb|control_snapshot| insteacd of \verb|control_mhd|. File format of the control file is same as the control field for simulation \hyperref[href_i:MHD_control]{\tt control\_MHD}.

The same files as the simulation program are read in this program, and field data are generated from the snapshots of spectrum data. The monitoring data for snapshots can also be generated. \verb|[step #]| is added in the file name, and the \verb|[step #]| is calculated by \verb|time step|/\verb|[ISTEP_FIELD]|.

We recommend to output cross section data at $y = 0$ by using sectioning module (see \ref{section:PSF}) for zonal mean snapshot program \verb|sph_zm_snapshot| to reduce data size.

\newpage
\section{Initial field generation program \\
({\tt sph\_initial\_field})}
\label{sec:sph_initial_field}
%
\begin{figure}[htbp]
\begin{center}
\includegraphics*[width=130mm]{images/flow_ini}
\end{center}
\caption{Data flow for initial field generation program.}
\label{fig:flow_ini}
\end{figure}
%
 The initial fields for dynamo benchmark can set in the simulation program by setting \verb|[INITIAL_TYPE]| flag. This program is used to generate initial field by user.  The heat source $q_{T}$ and light element source $q_{C}$ are also defined by this program because $q_{T}$ and $q_{C}$ are defined as scalar fields. Spherical harmonics indexing data files are also generated by using information in \verb|spherical_shell_ctl| block if these indexing data files do not exist. The Fortran source file to define initial field \\
 \verb|const_sph_initial_spectr.f90| is saved in \verb|src/programs/data_utilities| \\
 \verb|/INITIAL_FIELD/| directory,  and please compile again after modifying this module. This program also needs the files listed in Table \ref{table:inital_fld}.
%
\begin{table}[htp]
\caption{List of files for simulation {\tt sph\_initial\_field} }
\begin{center} 
\begin{tabular}{|c|c|c|}
\hline
 name & Parallelization & I/O \\ \hline \hline
\verb|control_MHD| & Serial & Input \\ \hline
\verb|[sph_prefix].[rj_extension]|  & - & Input/(Output) \\
\verb|[sph_prefix].[rlm_extension]| & - & Input/(Output) \\
\verb|[sph_prefix].[rtm_extension]| & - & Input/(Output) \\
\verb|[sph_prefix].[rtp_extension]| & - & Input/(Output) \\ \hline
\verb|[rst_prefix].[step #].[rst_extension]| & - & Input/Output  \\ \hline
\end{tabular}
\end{center}
(Output): Marked files are generated if files do not exist.
\label{table:inital_fld}
\end{table}
%
This program generates the spectrum data files \verb|[rst_prefix].[step #].[rst_extension]|. To use generated initial data file, please set 
 \hyperref[href_t:i_step_init_ctl]{{\tt [ISTEP\_START]}} to be 0 and \hyperref[href_t:restart_file_ctl]{{\tt [INITIAL\_TYPE]}} to be \\\hyperref[href_t:restart_file_ctl]{{\tt start\_from\_rst\_file}}.

\subsection{Definition of the initial field}
\label{sec:def_initial}
To construct Initial field data, you need to edit the source code \verb|const_sph_initial_spectr.f90| in \verb|src/programs/data_utilities/INITIAL_FIELD/| directory. The module \verb|const_sph_initial_spectr| consists of the following subroutines:
%
\begin{description}
\item{\verb|sph_initial_spectrum|:}    Top subroutine to construct initial field.
\item{\verb|set_initial_velocity|:}        Routine to construct initial velocity.
\item{\verb|set_initial_temperature|:} Routine to construct initial temperature.
\item{\verb|set_initial_composition|:} Routine to construct initial composition.
\item{\verb|set_initial_magne_sph|:} Routine to construct initial magnetic field.
\item{\verb|set_initial_heat_source_sph|:} Routine to construct heat source.
\item{\verb|set_initial_light_source_sph|:}  Routine to construct composition source.
\end{description}
%
The construction routine for each field are called from the top routine \\
\verb|const_sph_initial_spectr.f90|. If lines to call subroutines are commented out, corresponding initial fields are set to 0. In addition, the initial fields to be constructed need to be defined by \verb|nod_value_ctl| array in the \verb|control_MHD|.
%
\begin{table}[htp]
\caption{Field name and corresponding field id in Calypso}
\begin{center}
\begin{tabular}{|c|c|cc|}
\hline
field name & scalar & poloidal  & toroidal  \\ \hline
Velocity & - & \verb|ipol%i_velo| &   \verb|itor%i_velo| \\ 
Magnetic field & - & \verb|ipol%i_magne| &  \verb|itor%i_magne| \\ 
Current density & - & \verb|ipol%i_current| &  \verb|itor%i_current| \\ 
Temperature & \verb|ipol%i_temp| & - & - \\ 
Composition & \verb|ipol%i_light| & - & - \\ 
Heat source & \verb|ipol%i_heat_source| & - & - \\ 
Composition source & \verb|ipol%i_light_source| & - & - \\ \hline
\end{tabular}
\end{center}
\label{table:field_point}
\end{table}%
% 

Initial fields need to be defined by the spherical harmonics coefficients at each radial points as array \verb|d_rj(i,i_field)|, where \verb|i| and \verb|i_field| are the local address of the spectrum data and field id, respectively. The address of the fields are listed in Table \ref{table:field_point}.

In Calypso, local data address for each MPI process is used for the spectrum data address \verb|i|. To find the local address \verb|i|, two functions are required. \\
First, \verb|j = find_local_sph_mode_address(l,m)| returns the local spherical  harmonics address \verb|j| from aa spherical harmonics mode $Y_{l}^{m}$. If process does not have the data for $Y_{l}^{m}$, \verb|j| is set to 0. Second, \verb|i = local_sph_data_address(k,j)| returns the local data address \verb|i| from radial grid number \verb|k| and local spherical harmonics id \verb|j|. For do loops in the radial direction, the total number of radial grid points, radial address for ICB, and radial address for CMB are defined as \verb|nidx_rj(1)|, \verb|nlayer_ICB|, and \verb|nlayer_CMB|, respectively. The radius for the \verb|k|-th grid points can be obtained by \verb|r = radius_1d_rj_r(k)|. The subroutines to define initial temperature for the dynamo benchmark Case 1 is shown below as an example.

After updating the source code, the program \verb|sph_initial_field| needs to be updated. To update the program, move to the work directory \verb|[CALYPSO_HOME]/work| and run make command as
% 
\begin{verbatim}
% cd \verb|[CALYPSO_HOME]/work|
% make
\end{verbatim}
%
Then, the program  \verb|sph_initial_field| and  \verb|sph_add_initial_field| are updated.

%
\begin{verbatim}
!
      subroutine set_initial_temperature
!
      use m_sph_spectr_data
!
      integer ( kind = kint) :: inod, k, jj
      real (kind = kreal) :: pi, rr, xr, shell
      real(kind = kreal), parameter :: A_temp = 0.1d0
!
!
!$omp parallel do
      do inod = 1, nnod_rj
        d_rj(inod,ipol%i_temp) = zero
      end do
!$omp end parallel do
!
      pi = four * atan(one)
      shell = r_CMB - r_ICB
!
!   search address for (l = m = 0)
      jj = find_local_sph_mode_address(0, 0)
!
!   set reference temperature if (l = m = 0) mode is there
      if (jj .gt. 0) then
        do k = 1, nlayer_ICB-1
          inod = local_sph_data_address(k,jj)
          d_rj(inod,ipol%i_temp) = 1.0d0
        end do
        do k = nlayer_ICB, nlayer_CMB
          inod = local_sph_data_address(k,jj)
          d_rj(inod,ipol%i_temp) = (ar_1d_rj(k,1) * 20.d0/13.0d0        &
     &                              - 1.0d0 ) * 7.0d0 / 13.0d0
        end do
      end if
!
!
!    Find local addrtess for (l,m) = (4,4)
      jj =  find_local_sph_mode_address(4, 4)
!      jj =  find_local_sph_mode_address(5, 5)
!
!    If data for (l,m) = (4,4) is there, set initial temperature
      if (jj .gt. 0) then
!    Set initial field from ICB to CMB
        do k = nlayer_ICB, nlayer_CMB
!
!    Set radius data
          rr = radius_1d_rj_r(k)
!    Set 1d address to substitute at (Nr, j)
          inod = local_sph_data_address(k,jj)
!
!    set initial temperature
          xr = two * rr - one * (r_CMB+r_ICB) / shell
          d_rj(inod,ipol%i_temp) = (one-three*xr**2+three*xr**4-xr**6)  &
     &                            * A_temp * three / (sqrt(two*pi))
        end do
      end if
!
!    Center
      if(inod_rj_center .gt. 0) then
        jj = find_local_sph_mode_address(0, 0)
        inod = local_sph_data_address(1,jj)
        d_rj(inod_rj_center,ipol%i_temp) = d_rj(inod,ipol%i_temp)
      end if
!
      end subroutine set_initial_temperature
!
\end{verbatim}
%

\section{Initial field modification program \\
({\tt sph\_add\_initial\_field})}
\label{sec:add_initial_field}
%
\begin{figure}[htbp]
\begin{center}
\includegraphics*[width=130mm]{images/flow_ini}
\end{center}
\caption{Data flow for initial field modification program.}
\label{fig:flow_add_ini}
\end{figure}
%
{\bf Caution: This program overwrites existing initial field data. Please run it after taking a backup.} \\

 This program modifies or adds new data to an initial field file. It could be used to start a new geodynamo simulation by adding seed magnetic field or source terms to a non-magnetic convection simulation. The initial fields to be added are also defined in \verb|const_sph_initial_spectr.f90|. \verb|data_utilities/INITIAL_FIELD/| directory. This program also needs the files listed in Table \ref{table:add_inital_fld}.
%
\begin{table}[htp]
\caption{List of files for simulation {\tt sph\_add\_initial\_field} }
\begin{center} 
\begin{tabular}{|c|c|c|}
\hline
 name & Parallelization & I/O \\ \hline \hline
\verb|control_MHD| & Serial & Input \\ \hline
\verb|[sph_prefix].[rj_extension]|  & - & Input / (Output) \\
\verb|[sph_prefix].[rlm_extension]| & - & Input / (Output)  \\
\verb|[sph_prefix].[rtm_extension]| & - & Input / (Output)  \\
\verb|[sph_prefix].[rtp_extension]| & - & Input / (Output)  \\ \hline
\verb|[rst_prefix].[step #].[rst_extension]| & - & Input/Output  \\ \hline
\end{tabular}
\end{center}
(Output): Marked files are generated if files do not exist.
\label{table:add_inital_fld}
\end{table}
%
This program generates the spectrum data files \verb|[rst_prefix].[step#].[rst_extension]|. To use generated initial data file, set 
 \hyperref[href_t:i_step_init_ctl]{{\tt [ISTEP\_START]}} and \verb|[ISTEP_RESTART]| to be appropriate time step and increment, respectively.
To read the original initial field data, \hyperref[href_t:restart_file_ctl]{{\tt [INITIAL\_TYPE]}} is set to be \hyperref[href_t:restart_file_ctl]{{\tt start\_from\_rst\_file}} in \verb|control_MHD|. In other words, the \verb|[step #]| in the file name, \hyperref[href_t:i_step_init_ctl]{{\tt [ISTEP\_START]}}, and \verb|[ISTEP_RESTART]| in the control file should be the consistent.

This program also uses the module file \verb|const_sph_initial_spectr.f90| to define the initial field. The initial fields are defined as following the previous section \ref{sec:def_initial}. After updating the source code, the program \verb|sph_initial_field| needs to be updated. After modifying  \verb|const_sph_initial_spectr.f90|, the program is build by make command in  the work directory \verb|[CALYPSO_HOME]/work|.

\section{Check program for dynamo benchmark \\
({\tt sph\_dynamobench})}
This program is only used to check solution for dynamo benchmark by Christensen {\it et. al}. The following files are used for this program.

\begin{table}[htp]
\caption{List of files for dynamo benchmark check {\tt sph\_dynamobench} }
\begin{center} 
\begin{tabular}{|c|c|c|}
\hline
 name & Parallelization & I/O \\ \hline \hline
\verb|control_snapshot| & Serial & Input \\ \hline
\verb|[sph_prefix].[rj_extension]|  & - & Input \\
\verb|[sph_prefix].[rlm_extension]| & - & Input \\
\verb|[sph_prefix].[rtm_extension]| & - & Input \\
\verb|[sph_prefix].[rtp_extension]| & - & Input \\ \hline
\verb|[rst_prefix].[step#].[rst_extension]| &  - & Input  \\ \hline
\verb|dynamobench.dat| & Single & Output \\ \hline
\end{tabular}
\end{center}
\label{table:sph_dynamobench}
\end{table}

\subsection{Dynamo benchmark data {\tt dynamobench.dat}}
 In benchmark test by Christensen {\it et. al.}, both global values and local values are checked. As global results, Kinetic energy 
 $\displaystyle{ \frac{1}{V} \int \frac{1}{2} u^{2} dV}$ in the fluid shell, magnetic energy in the fluid shell 
 $\displaystyle{ \frac{1}{V} \frac{1}{E Pm} \int \frac{1}{2} B^{2} dV}$ (for case 1 and 2), and magnetic energy in the solid inner sphere 
 $\displaystyle{ \frac{1}{V_{i}} \frac{1}{E Pm} \int \frac{1}{2} B^{2} dV_{i}}$ (for case 2 only). Benchmark also requests 
 By increasing number of grid point at mid-dpeth of the fluid shell in the equatorial plane by \hyperref[href_t:nphi_mid_eq_ctl]{{\tt nphi\_mid\_eq\_ctl}}, program can find accurate solution for the point where $u_{r} = 0$ and $\partial u_{r} / \partial \phi > 0$. Angular frequency of the field pattern with respect to the $\phi$ direction is also required. The benchmark test also requires temperature and $\theta$ component of velocity. In the text file {\tt dynamobench.dat}, the following data are written in one line for every \verb|[i_step_rst_ctl]| step.
%
\begin{description}
\item{\tt t\_step:  }  Time step number
\item{\tt time:     }  Time
\item{\tt KE\_pol:   }  Poloidal kinetic energy
\item{\tt KE\_tor:   }  Toroidal kinetic energy
\item{\tt KE\_total: }  Total kinetic energy
\item{\tt ME\_pol:   }  Poloidal magnetic energy  (Case 1 and 2)
\item{\tt ME\_tor:   }  Toroidal magnetic energy  (Case 1 and 2)
\item{\tt ME\_total: }  Total magnetic energy  (Case 1 and 2)
\item{\tt ME\_pol\_ic:      }  Poloidal magnetic energy in inner core  (Case 2)
\item{\tt ME\_tor\_icore:   }  Toroidal magnetic energy in inner core (Case 2)
\item{\tt ME\_total\_icore: }  Total magnetic energy in inner core (Case 2)
\item{\tt omega\_ic\_z: } Angular velocity of inner core rotation (Case 2)
\item{\tt MAG\_torque\_ic\_z: }  Magnetic torque integrated over the inner core (Case 2)
\item{\tt phi\_1...4: } Longitude where $u_{r} = 0$ and $\partial u_{r} / \partial \phi > 0$ at mid-depth in equatorial plane.
\item{\tt omega\_vp44:} Drift frequency evaluated by $V_{S4}^{\ 4}$ component
\item{\tt omega\_vt54:} Drift frequency evaluated by $V_{T5}^{\ 4}$ component
\item{\tt B\_theta: } $\Theta$ component of magnetic field at requested point.
\item{\tt v\_phi: } $\phi$ component of velocity at requested point.
\item{\tt temp: } Temperature at requested point.


\end{description}

{\small 
\begin{verbatim}
t_step    time    KE_pol    KE_tor    KE_total    ME_pol    ME_t
or    ME_total    ME_pol_icore    ME_tor_icore    ME_total_icore
    omega_ic_z    MAG_torque_ic_z    phi_1    phi_2    phi_3    
phi_4    omega_vp44    omega_vt54    B_theta    v_phi    temp
     20000   9.999999999998981E-001   1.534059732073072E+001   2
.431439471284618E+001   3.965499203357688E+001   2.4056940119550
09E+000   1.648662987055900E+000   4.054356999010911E+000   3.90
8687924452961E+001   4.812865754441352E-001   3.956816581997376E
+001   5.220517005592486E+000  -2.321885847438682E+002   3.59417
5626663308E-001   1.930213889461227E+000   3.501010216256124E+00
0   5.071806543051021E+000   7.808553595635292E-001  -1.64958344
1437563E-001  -5.136522824340612E+000  -8.047915942925034E+000  
 3.752181234262930E-001
...
\end{verbatim}
}
%

\section{Sectioning program ({\tt sectioning})} \label{sec:sectioning}
This program generates cross sections and isosurfaces from FEM mesh data and field data using the sectioning and isosurface module in the simulation program {\tt sph\_mhd}. The data for this program is listed in Table \ref{table:sectioning}. This program run on the parallel environment, and needs to use the same number of MPI processes as the number of processes which is used for the simulation program. VTK and compressed VTK data is not supported for the input field data.
%
\begin{table}[htp]
\caption{List of files for sectioning {\tt sectioning} }
\begin{center} 
\begin{tabular}{|c|c|c|}
\hline
name & Parallelization & I/O \\ \hline \hline
\verb|control_viz| & Serial & Input \\ \hline
\verb|[mesh_prefix].[fem_extension]| & - & Input \\ \hline
\verb|[fld_prefix].[step#].[domain#].[extension]| & - & Input  \\ \hline
\verb|[section_prefix].[step#].[extension]| &  Single & Output  \\
\verb|[isosurface_prefix].[step#].[extension]| &  Single & Output  \\ \hline
\end{tabular}
\end{center}
\label{table:sectioning}
\end{table}
%

\subsection{Control file}
The format of the control file \verb|control_viz| is described below. The detail of each block is described in section \ref{section:def_control}. You can jump to detailed description by clicking each item". \\
\\
%
Block \verb|visualizer|  (Top block of the control file)
\label{href_i:visualizer}
%
\begin{itemize}
\item Block \hyperref[href_t:data_files_def]{\tt data\_files\_def}
	\label{href_i:data_files_def_v}
%
	\begin{itemize}
	\item \hyperref[href_t:num_subdomain_ctl]
			{\tt num\_subdomain\_ctl    [Num\_PE]}
	\item \hyperref[href_t:num_smp_ctl]
			{\tt num\_smp\_ctl    [Num\_Threads]}
	\item \hyperref[href_t:mesh_file_prefix]
			{\tt mesh\_file\_prefix    [mesh\_prefix]}
	\item \hyperref[href_t:field_file_prefix]
			{\tt field\_file\_prefix    [fld\_prefix]}
%
	\item \hyperref[href_t:mesh_file_fmt_ctl]
			{\tt mesh\_file\_fmt\_ctl    [mesh\_format]}
	\item \hyperref[href_t:field_file_fmt_ctl]
			{\tt field\_file\_fmt\_ctl    [fld\_format]}
	\end{itemize}
%
\item Block \hyperref[href_t:time_step_ctl]{\tt time\_step\_ctl}
	\begin{itemize} \label{href_i:time_step_ctl_v}
	\item \hyperref[href_t:i_step_init_ctl]
		{\tt i\_step\_init\_ctl        [ISTEP\_START]}
	\item \hyperref[href_t:i_step_finish_ctl]
		{\tt i\_step\_finish\_ctl      [ISTEP\_FINISH]}
	\item \hyperref[href_t:i_step_field_ctl]
		{\tt i\_step\_field\_ctl       [ISTEP\_FIELD]}
	\item \hyperref[href_t:i_step_sectioning_ctl]
		{\tt i\_step\_sectioning\_ctl  [ISTEP\_SECTION]}
	\item \hyperref[href_t:i_step_isosurface_ctl]
		{\tt i\_step\_isosurface\_ctl  [ISTEP\_ISOSURFACE]}
	\end{itemize}
%
\item Block \hyperref[href_t:visual_control]{\tt visual\_control}
    \begin{itemize} \label{href_i:visual_control_v}
    \item \hyperref[href_t:i_step_sectioning_ctl]
        {\tt i\_step\_sectioning\_ctl  [ISTEP\_SECTION]}
    \item Array \hyperref[href_t:cross_section_ctl]{\tt cross\_section\_ctl}
		\begin{itemize}
        \item File or Block {\tt cross\_section\_ctl} \\
                            {\tt [section\_control\_file]} \\
								(See section \ref{section:section_control})
		\end{itemize}
%
    \item \hyperref[href_t:i_step_isosurface_ctl]
		{\tt i\_step\_isosurface\_ctl  [ISTEP\_ISOSURFACE]}
    \item Array \hyperref[href_t:isosurface_ctl]{\tt isosurface\_ctl}
		\begin{itemize}
		\item File or Block {\tt isosurface\_ctl} \\
                            {\tt [isosurface\_control\_file]} \\
								(See section \ref{section:isosurface_control})
		\end{itemize}
    \end{itemize}
\end{itemize}
%

\section{Field data converter program ({\tt field\_to\_VTK})} \label{sec:field_to_VTK}
This program generates VTK data from FEM mesh data and field data. The data for this program is listed in Table \ref{table:fld_to_vtk}. This program run on the parallel environment, and needs to use the same number of MPI processes as the number of processes which is used for the simulation program.
%
\begin{table}[htp]
\caption{List of files for sectioning {\tt sectioning} }
\begin{center} 
\begin{tabular}{|c|c|c|}
\hline
name & Parallelization & I/O \\ \hline \hline
\verb|control_viz| & Serial & Input \\ \hline
\verb|[mesh_prefix].[fem_extension]| & - & Input \\ \hline
\verb|[fld_prefix].[step#].[domain#].[extension]| & - & Input  \\ \hline
\verb|[fld_prefix].[step#].[domain#].[vtk]| or [vtk.gz] & - & Output  \\ \hline
\end{tabular}
\end{center}
\label{table:sectioning}
\end{table}
%

\newpage
\subsection{Control file}
The format of the control file \verb|control_viz| is described below. The detail of each block is described in section \ref{section:def_control}. You can jump to detailed description by clicking each item". \\
\\
%
Block \verb|visualizer|  (Top block of the control file)
\label{href_i:visualizer}
%
\begin{itemize}
\item Block \hyperref[href_t:data_files_def]{\tt data\_files\_def}
	\label{href_i:data_files_def_f}
%
	\begin{itemize}
	\item \hyperref[href_t:num_subdomain_ctl]
			{\tt num\_subdomain\_ctl    [Num\_PE]}
	\item \hyperref[href_t:num_smp_ctl]
			{\tt num\_smp\_ctl    [Num\_Threads]}
	\item \hyperref[href_t:mesh_file_prefix]
			{\tt mesh\_file\_prefix    [mesh\_prefix]}
	\item \hyperref[href_t:field_file_prefix]
			{\tt field\_file\_prefix    [fld\_prefix]}
%
	\item \hyperref[href_t:mesh_file_fmt_ctl]
			{\tt mesh\_file\_fmt\_ctl    [mesh\_format]}
	\item \hyperref[href_t:field_file_fmt_ctl]
			{\tt field\_file\_fmt\_ctl    [fld\_format]}
	\end{itemize}
%
\item Block \hyperref[href_t:time_step_ctl]{\tt time\_step\_ctl}
	\begin{itemize} \label{href_i:time_step_ctl_f}
	\item \hyperref[href_t:i_step_init_ctl]
		{\tt i\_step\_init\_ctl        [ISTEP\_START]}
	\item \hyperref[href_t:i_step_finish_ctl]
		{\tt i\_step\_finish\_ctl      [ISTEP\_FINISH]}
	\item \hyperref[href_t:i_step_field_ctl]
		{\tt i\_step\_field\_ctl       [ISTEP\_FIELD]}
	\end{itemize}
%
\item Block \hyperref[href_t:visual_control]{\tt visual\_control}
    \begin{itemize} \label{href_i:visual_control_v}
    \item \hyperref[href_t:output_field_file_fmt_ctl]
		{\tt output\_field\_file\_fmt\_ctl  [VTK\_format]}
    \end{itemize}
\end{itemize}
%

\section{Section and isosurface data converter program ({\tt psf\_to\_VTK})} \label{section:psf_to_VTK}
This program generates VTK data from bindary sectioning and isosurface data. This program run on a single processor, and needs interactive input. The following is the console output of the program.

{\small
\begin{verbatim}
% /usr/local/Calypso/bin/psf_to_vtk 
Input file prefix
zm_y0									<- Input file prefix
Input file extension from following:
vtk, vtk.gz, vtd, vtd.gz, inp, inp.gz, udt, udt.gz, psf, psf.gz, sdt, sdt.gz  
sdt.gz									<- Input extension
ifmt_input          23
Input start, end, and increment of file step
{\color{red} 2004 2000 1								<- Input start, end, and increment of file step} 
Write ascii VTK file: zm_y0.2000.vtk
\end{verbatim}
}

\section{Data assemble program ({\tt assemble\_sph})}
\label{section:assemble_sph}
%
\begin{figure}[htbp]
\begin{center}
\includegraphics*[width=130mm]{images/flow_4}
\end{center}
\caption{Data flow for spectrum data assemble program}
\label{fig:flow_4}
\end{figure}
%
Calypso uses distributed data files for simulations. This program is to generate new spectrum data for restarting with different spatial resolution or parallel configuration. This program organizes new spectral data by using specter indexing data using different domain decomposition. The following files used for data IO. If radial resolution is changed from the original data, the program makes new spectrum data by linear interpolation. If new data have smaller or larger truncation degree, the program fills zero to the new spectrum data or truncates the data to fit the new spatial resolution, respectively. This program can perform with any number of MPI processes, but we recommend to run the program with {\bf one} process or the same number of processes as the number of subdomains for the target configuration which is defined by \verb|num_new_domain_ctl|. Data files for the program are shown In Table \ref{table:assemble_newsph}. The time and number of time step can also be changed by this program. The new time and time step are defined by the parameters in \verb|new_time_step_ctl| block. The step number of the restart data will be \verb|i_step_init_ctl| / \verb|i_step_rst_ctl| in  \verb|new_time_step_ctl|. If \verb|new_time_step_ctl| block is not defined, time and time step informations are carried from the original restart data.

%
\begin{table}[htp]
\caption{List of files for {\tt assemble\_sph} }
\begin{center} 
\begin{tabular}{|c|c|c|}
\hline
 extension & Distributed? & I/O \\ \hline
\verb|control_assemble_sph| & Serial & Input \\ \hline
\verb|[sph_prefix].[rj_extension]|  & - & Input \\  \hline
\verb|[new_sph_prefix].[domain#].rj| &  Distributed & Input \\ \hline
\verb|[rst_prefix].[step#].[rst_extension]| & - & Input  \\
\verb|[new_rst_prefix].[step#].[domain#].fst| &  Distributed & Output \\ \hline
\end{tabular}
\end{center}
\label{table:assemble_newsph}
\end{table}
%

\subsection{Format of control file}
Control file consists the following groups. \\
%
Block \verb|assemble_control| \label{href_i:assemble_control} (Top lebel of the block)
\begin{itemize}
\item Block \verb|data_files_def|
	\hyperref[href_t:data_files_def]{(Detail)}
	\begin{itemize}
	\item \hyperref[href_t:num_subdomain_ctl]
			{\tt num\_subdomain\_ctl    [Num\_PE]}
	\item \hyperref[href_t:sph_file_prefix]
			{\tt sph\_file\_prefix      [sph\_prefix]}
	\item \hyperref[href_t:restart_file_prefix]
            {\tt restart\_file\_prefix  [rst\_prefix])}
%
	\item \hyperref[href_t:sph_file_fmt_ctl]
			{\tt sph\_file\_fmt\_ctl    [sph\_format]}
	\item \hyperref[href_t:restart_file_fmt_ctl]
			{\tt restart\_file\_fmt\_ctl    [rst\_format]}
	\end{itemize}
%
\item Block \verb|new_data_files_def|
	\label{href_i:new_data_files_def}
	\hyperref[href_t:new_data_files_def]{(Detail)}
	\begin{itemize}
	\item \hyperref[href_t:num_subdomain_ctl]
			{\tt num\_subdomain\_ctl    [Num\_PE]}
	\item \hyperref[href_t:sph_file_prefix]
			{\tt sph\_file\_prefix      [sph\_prefix]}
	\item \hyperref[href_t:restart_file_prefix]
            {\tt restart\_file\_prefix  [rst\_prefix])}
%
	\item \hyperref[href_t:sph_file_fmt_ctl]
			{\tt sph\_file\_fmt\_ctl    [sph\_format]}
	\item \hyperref[href_t:restart_file_fmt_ctl]
			{\tt restart\_file\_fmt\_ctl    [rst\_format]}
%
	\item \hyperref[href_t:delete_original_data_flag]
			{\tt delete\_original\_data\_flag    [YES or NO]}
	\end{itemize}
%
\item Block \verb|control|
	\begin{itemize}
	\item Block \hyperref[href_t:time_step_ctl]{\tt time\_step\_ctl}
		\begin{itemize} \label{href_i:time_step_ctl2}
		\item \hyperref[href_t:i_step_init_ctl]
			{\tt i\_step\_init\_ctl        [ISTEP\_START]}
		\item \hyperref[href_t:i_step_finish_ctl]
			{\tt i\_step\_finish\_ctl      [ISTEP\_FINISH]}
		\item \hyperref[href_t:i_step_rst_ctl]
			{\tt i\_step\_rst\_ctl         [ISTEP\_RESTART]}
		\end{itemize}
%
	\item  Block \hyperref[href_t:i_step_init_ctl]{\tt new\_time\_step\_ctl}
		\begin{itemize} \label{href_i:new_time_step_ctl}
		\item \hyperref[href_t:i_step_init_ctl_a]
			{\tt i\_step\_init\_ctl        [ISTEP\_START]}
		\item \hyperref[href_t:i_step_rst_ctl_a]
			{\tt i\_step\_rst\_ctl         [ISTEP\_RESTART]}
		\item \hyperref[href_t:time_init_ctl_a]
			{\tt time\_init\_ctl           [INITIAL\_TIME]}
		\end{itemize}
	\end{itemize}
%
\item Block \hyperref[href_t:newrst_magne_ctl]{\tt newrst\_magne\_ctl}
	\begin{itemize} \label{href_i:newrst_magne_ctl}
	\item \hyperref[href_t:magnetic_field_ratio_ctl]
		{\tt magnetic\_field\_ratio\_ctl    [ratio]}
	\end{itemize}
\end{itemize}

\section{Time averaging programs}
These small programs are used to evaluate time average and standard deviation of the time evolution data.
\subsection{Averaging for mean square and power spectrum \\
 ({\tt t\_ave\_sph\_mean\_square})}
This program generate time average and standard deviation of power spectrum data. The program processes one of data files listed in Table \ref{table:time_averages_sprectr}. The number for the first and second interactive input is also listed in Table \ref{table:time_averages_sprectr}. For the third input, the file name excluding \verb|.dat| is required. Start and end time is also required in the last input. If data is end before the end time, the program will finish at the end of file. \verb|t_ave| and \verb|t_sigma| are added at the beginning of the input file name for the time average and standard deviation data file, respectively.
%
\begin{table}[htp]
\caption{List of programs to take time average}
\begin{center} 
\begin{tabular}{|c|c|c|c|}
\hline
 name & First input & Second input \\ \hline \hline
\verb|[vol_pwr_prefix]_s.dat| & 1 & 1 \\ \hline
\verb|[vol_pwr_prefix]_l.dat| & 2 & 1 \\
\verb|[vol_pwr_prefix]_m.dat| & 2 & 1 \\
\verb|[vol_pwr_prefix]_lm.dat| & 2 & 1 \\ \hline
\verb|[layer_pwr_prefix]_s.dat| & 1 & 0  \\ \hline
\verb|[layer_pwr_prefix]_l.dat| & 2 & 0 \\
\verb|[layer_pwr_prefix]_m.dat| & 2 & 0 \\
\verb|[layer_pwr_prefix]_lm.dat| & 2 & 0 \\ \hline
\end{tabular}
\end{center}
\label{table:time_averages_sprectr}
\end{table}
%
\subsection{Averaging for picked harmonics mode data \\
 ({\tt t\_ave\_picked\_sph\_coefs})}
This program generate time average and standard deviation of spherical harmonic coefficients which selected in the file \verb|[picked_sph_prefix].dat|.

In this program, file prefix  \verb|[picked_sph_prefix]| and start and end time are required in the interactive input. If data is end before the end time, the program will finish at the end of file. \verb|t_ave| and \verb|t_sigma| are added at the beginning of the input file name for the time average and standard deviation data file, respectively.
%
\subsection{Averaging for Nusselt number data \\
 ({\tt t\_ave\_nusselt})}
This program generate time average and standard deviation of the Nusselt number in the file \verb|[nusselt_number_prefix].dat|. 

In this program, file prefix  \verb|[nusselt_number_prefix]| and start and end time are required in the interactive input. If data is end before the end time, the program will finish at the end of file. \verb|t_ave| and \verb|t_sigma| are added at the beginning of the input file name for the time average and standard deviation data file, respectively.
%

\section{Module dependency program ({\tt module\_dependency})}
This program is only used to generate Makefile in {\tt work} directory. Most of case, Fortran 90 modules have to compiled prior to be referred by another fortran90 routines. This program is generates dependency lists in Makefile. To use this program, the following limitation is required.
\begin{itemize}
\item One source code has to consist of one module.
\item The module name should be the same as the file name.
\end{itemize}
%
\section{Visualization using field data}
\label{sec:paraview}
The field data is written by XDMF or VTK data format using Cartesian coordinate. In this section we briefly introduce how to display the radial magnetic field using ParaView as an example.
%
\begin{figure}[htbp]
\begin{center}
\includegraphics*[width=130mm]{images/paraview_open}
\caption{File open window for ParaView}
\label{fig:paraview_load}
\end{center}
\end{figure}
%

After the starting Paraview, the file to be read is chosen in the file menu, and press "apply", button. Then, Paraview load the data from files (see Figure \ref{fig:paraview_load}). 
Because the magnetic field is saved by the Cartesian coordinate, the radial magnetic field is obtained by the calculator tool. The procedure is as following (see Figure \ref{fig:paraview_gen_Br})
%
\begin{enumerate}
\item Push calculator button.
\item Choose "Point Data" in Attribute menu
\item Input data name for radial magnetic field ("B\_r" in  Figure \ref{fig:paraview_gen_Br})
\item Enter the equation to evaluate radial mantic field $B_{r} = \bvec{B} \cdot \bvec{r} / |r|$.
\item Finally, push "Apply" button.
\end{enumerate}
%
%
\begin{figure}[htbp]
\begin{center}
\includegraphics*[width=100mm]{images/paraview_calc}
\caption{File open window for ParaView}
\label{fig:paraview_gen_Br}
\end{center}
\end{figure}
%
After obtaining the radial mantric field, the image in figure \ref{fig:paraview_br} is obtained by using "slice" and  "Contour" tools with appropriate color mapping.
%
\begin{figure}[htbp]
\begin{center}
\includegraphics*[width=100mm]{images/Paraview_Br}
\end{center}
\caption{Visualization of radial magnetic field by Paraview.}
\label{fig:paraview_br}
\end{figure}
%



\newpage
\begin{thebibliography}{10}

\bibitem{Bullard:54} Bullard, E. C. and Gellman, H., Homogeneous dynamos and terrestrial magnetism, {\it Proc. of the Roy. Soc. of London}, {\bf A247}, 213--278, 1954.
\bibitem{Uli:2001} Christensen, U.R., Aubert, J., Cardin, P., Dormy, E., Gibbons, S., Glatzmaier, G. A., Grote, E., Honkura, H., Jones, C., Kono, M., Matsushima, M., Sakuraba, A., Takahashi, F., Tilgner, A., Wicht, J. and Zhang, K., A numerical dynamo benchmark, {\it Physics of the Earth and Planetary Interiors}, {\bf 128}, 25--34, 2001.

\end{thebibliography}


\newpage
\begin{appendices}
\section{Definition of parameters for control files}
\label{section:def_control}

\subsection{Block {\tt data\_files\_def}}
\label{href_t:data_files_def}
File names and number of processes and threads are defined in this block. \\
\hyperref[href_i:MHD_control]{(Back to {\tt control\_MHD})} \\
\hyperref[href_i:spherical_shell_ctl]{(Back to {\tt control\_sph\_shell})} \\
\hyperref[href_i:assemble_control]{(Back to {\tt control\_assemble\_sph})}

\paragraph{\tt num\_subdomain\_ctl}
\label{href_t:num_subdomain_ctl}
\verb|[Num_PE]| \\
Number of subdomain for the MPI program \verb|[Num_PE]| is defined by integer. If number of processes in \verb| mpirun -np | is different from number of subdomains, program will be stopped with message.

\paragraph{\tt num\_smp\_ctl}
\label{href_t:num_smp_ctl}
\verb|[Num_Threads]| \\
Number of SMP threads for OpenMP \verb|[Num_Threads]| is defined by integer. You can set larger number than the actual umber of thread to be used. If actual number of thread is less than this number, number of threads is set to the number which is defined in this field.

\paragraph{\tt sph\_file\_prefix}
\label{href_t:sph_file_prefix}
\verb|[sph_prefix]| \\
File prefix of spherical harmonics indexing and FEM mesh file \verb|[sph_prefix]| is defined by text. Process ID and extension are added after this file prefix.

\paragraph{\tt mesh\_file\_prefix}
\label{href_t:mesh_file_prefix}
\verb|[mesh_prefix]| \\
File prefix of FEM mesh file \verb|[mesh_prefix]| is defined by text. Process ID and extension are added after this file prefix. This flag is only used for the sectioning program (\hyperref[sec:sectioning]{\tt sectioning}) and data converter to VTK (\hyperref[sec:field_to_VTK]{\tt field\_to\_VTK}).

\paragraph{\tt boundary\_data\_file\_name}
\label{href_t:boundary_data_file_name}
\verb|[boundary_data_name]| \\
File name of boundary condition data file \verb|[boundary_data_name]| is defined by text. 

\paragraph{\tt restart\_file\_prefix}
\label{href_t:restart_file_prefix}
\verb|[rst_prefix]| \\
File prefix of spectrum data for restarting and snapshots \verb|[rst_prefix]| is defined by text. Step number, process ID, and extension are added after this file prefix.

\paragraph{\tt field\_file\_prefix}
\label{href_t:field_file_prefix}
\verb|[fld_prefix]| \\
File prefix of field data for visualize snapshots \verb|[fld_prefix]| is defined by text. Step number and file extension are  added after this file prefix.

\paragraph{\tt sph\_file\_fmt\_ctl}
\label{href_t:sph_file_fmt_ctl}
\verb|[sph_formayt]| \\
File format of spherical harmonics indexing and FEM mesh file \verb|[sph_format]| is defined by text. Following data formats can be defined. Extensions of each data format is listed in Table \ref{table:mesh_format}.
%
\begin{description}
\item{\tt ascii: }   Distributed ASCII data
\item{\tt binary: }  Distributed binary data
\item{\tt merged: }  Merged ASCII data
\item{\tt merged\_bin: }   Merged binary data
\item{\tt gzip: }            Compressed distributed ASCII data
\item{\tt binary\_gz: }      Compressed distributed binary data
\item{\tt merged\_gz: }      Compressed merged ASCII data
\item{\tt merged\_bin\_gz: } Compressed merged binary data
\end{description}
%

\paragraph{\tt mesh\_file\_fmt\_ctl}
\label{href_t:mesh_file_fmt_ctl}
\verb|[mesh_formayt]| \\
File format of FEM mesh file \verb|[mesh_format]| is defined by text. Data formats can be defined the same as {\tt sph\_file\_fmt\_ctl}. Extensions of each data format is listed in Table \ref{table:mesh_format}. This flag is only used for the sectioning program (\hyperref[sec:sectioning]{\tt sectioning}) and data converter to VTK (\hyperref[sec:field_to_VTK]{\tt field\_to\_VTK}).

\paragraph{\tt restart\_file\_fmt\_ctl}
\label{href_t:restart_file_fmt_ctl}
\verb|[rst_format]| \\
File format of restart files \verb|[rst_format]| is defined by text. Following data formats can be defined. Extensions of each data format is listed in Table \ref{table:restart_format}.
%
\begin{description}
\item{\tt ascii: }   Distributed ASCII data
\item{\tt binary: }  Distributed binary data
\item{\tt merged: }  Merged ASCII data
\item{\tt merged\_bin: }   Merged binary data
\item{\tt gzip: }            Compressed distributed ASCII data
\item{\tt binary\_gz: }      Compressed distributed binary data
\item{\tt merged\_gz: }      Compressed merged ASCII data
\item{\tt merged\_bin\_gz: } Compressed merged binary data
\end{description}
%

\paragraph{\tt field\_file\_fmt\_ctl}
\label{href_t:field_file_fmt_ctl}
\verb|[fld_format]| \\
Field data field format for visualize snapshots \verb|[fld_format]| is defined by text. The following formats are currently supported.
%
\begin{description}
\item{\tt single\_HDF5: }  Merged HDF5 file (Available if HDF5 library is linked)
\item{\tt single\_VTK: }   Merged VTK file (Default)
\item{\tt VTK: }           Distributed VTK file
\item{\tt single\_VTK\_gz: }   Compressed merged VTK file (Available if zlib library is linked)
\item{\tt VTK\_gz: }           Compressed distributed VTK file (Available if zlib library is linked)
\end{description}
%
%
%
%

\subsection{\tt spherical\_shell\_ctl}
\label{href_t:spherical_shell_ctl}
Configuration of the spherical shell and parallelization are defined by in this block. This block can be stored in an external file.
%

\subsubsection{\tt FEM\_mesh\_ctl}
\label{href_t:FEM_mesh_ctl}
Configuration of the FEM mesh is defined in this block. This block is optional.
\hyperref[href_i:FEM_mesh_ctl]{(Back to {\tt control\_sph\_shell})}

\paragraph{\tt FEM\_mesh\_output\_switch}
\label{href_t:FEM_mesh_output_switch}
\verb|[ON or OFF]| \\
Set \verb|ON| if FEM mesh data need to be written.
%

\subsubsection{\tt num\_domain\_ctl}
\label{href_t:num_domain_ctl}
Parallelization is defined in this block. Domain decomposition is defined for spectrum data, field data, and Legendre transform. \\
\hyperref[href_i:num_domain_ctl]{(Back to {\tt control\_sph\_shell})}

\paragraph{\tt num\_radial\_domain\_ctl}
\label{href_t:num_radial_domain_ctl}
\verb|[Ndomain]| \\
Number of subdomains in the radial direction for the spherical grid $(r, \theta, \phi)$ and spherical transforms $(r, \theta, m)$ and $(r, l, m)$.

\paragraph{\tt num\_horizontal\_domain\_ctl}
\label{href_t:num_horizontal_domain_ctl} 
\verb|[Ndomain]| \\
Number of subdomains in the horizontal direction. The number will be the number of subdomains for the meridional directios for the spherical grid $(r, \theta, \phi)$ and Fourier transform $(r, \theta, m)$. For Legendre transform $(r, \theta, m)$ and $(r, l, m)$, the number will be the number of subdomains for the h.armonics ordedr $m$.


\paragraph{\color{magenta} \tt num\_domain\_sph\_grid    [Direction]    [Ndomain]}
\label{href_t:num_domain_sph_grid} 
{\color{magenta} (Depricated)}\\
 Definition of number of subdomains for physical data in spherical coordinate $(r, \theta, \phi)$. Direction {\tt  radial} or {\tt meridional} is set in \verb|[Direction]|, and number of subdomains \verb|[Ndomain]| are defined in the integer field.

\paragraph{\color{magenta} \tt num\_domain\_legendre    [Direction]    [Ndomain]}
\label{href_t:num_domain_legendre}
{\color{magenta} (Depricated)}\\
 Definition of number of subdomains for Legendre transform between $(r, \theta, m)$ and $(r, l, m)$. Direction {\tt  radial} or {\tt zonal} is set in \verb|[Direction]|, and number of subdomains \verb|[Ndomain]| are defined in the integer field.

\paragraph{\color{magenta} \tt num\_domain\_spectr    [Direction]    [Ndomain]}
\label{href_t:num_domain_spectr}
{\color{magenta} (Depricated)}\\
Definition of number of subdomains for spectrum data in $(r, l, m)$. Direction {\tt  modes} is set in the \verb|[Direction]| field, and number of subdomains \verb|[Ndomain]| are defined in the integer field.


\subsubsection{\tt num\_grid\_sph}
\label{href_t:num_grid_sph}
Spatial resolution of the spherical shell is defined in this block. \\
\hyperref[href_i:num_grid_sph]{(Back to {\tt control\_sph\_shell})}

\paragraph{\tt truncation\_level\_ctl}
\label{href_t:truncation_level_ctl}
\verb|[Lmax]| \\
Truncation level $L$ is defined by integer. Spherical harmonics is truncated by triangular $0 \le l \le L$ and $0 <m < l$.

\paragraph{\tt ngrid\_meridonal\_ctl}
\label{href_t:ngrid_meridonal_ctl}
\verb|[Ntheta]| \\
Number of grid in the meridional direction \verb|[Ntheta]| is defined by integer.

\paragraph{\tt ngrid\_zonal\_ctl}
\label{href_t:ngrid_zonal_ctl}
\verb|[Nphi]| \\
Number of grid in the zonal direction \verb|[Nphi]| is defined by integer.

\paragraph{\tt raidal\_grid\_type\_ctl}
\label{href_t:radial_grid_type_ctl}
\verb|[explicit, Chebyshev, or equi_distance]| \\
Type of the radial grid spacing is defined by text. The following types are supported in Calypso.
%
\begin{description}
	\item{\tt explicit}  Equi-distance grid
	\item{\tt Chebyshev} Chebyshev collocation points
	\item{\tt equi\_distance} Set explicitly by \verb|r_layer| array
\end{description}
%

\paragraph{\tt num\_fluid\_grid\_ctl}
\label{href_t:num_fluid_grid_ctl}
\verb|[Nr_shell]| \\
(This option works with \verb|radial_grid_type_ctl| is {\tt explicit} or {\tt Chebyshev}.)
Number of layer in the fluid shell \verb|[Nr_shell]| is defined by integer. Number of grids including CMB and ICB will be (\verb|[Nr_shell]| + 1).

\paragraph{\tt fluid\_core\_size\_ctl}
\label{href_t:fluid_core_size_ctl}
\verb|[Length]| \\
(This option works with \verb|radial_grid_type_ctl| is {\tt explicit} or {\tt Chebyshev}.)
Size of the outer core \verb|[Length]| ($ = r_{o}-r_{i}$) is defined by real.

\paragraph{\tt ICB\_to\_CMB\_ratio\_ctl}
\label{href_t:ICB_to_CMB_ratio_ctl} 
\verb|[R_ratio]| \\
(This option works with \verb|radial_grid_type_ctl| is {\tt explicit} or {\tt Chebyshev}.)
Ratio of the inner core radius to outer core \verb|[R_ratio]| ($ = r_{i} / r_{o}$) is defined by real.

\paragraph{\tt Min\_radius\_ctl}
\label{href_t:Min_radius_ctl}
\verb|[Rmin]| \\
(This option works with \verb|radial_grid_type_ctl| is {\tt explicit} or {\tt Chebyshev}.)
Minimum radius of the domains \verb|[Rmin]| is defined by real. If this value is not defined, ICB becomes inner boundary of the domain.

\paragraph{\tt Max\_radius\_ctl }
\label{href_t:Max_radius_ctl} 
\verb|[Rmax]| \\
(This option works with \verb|radial_grid_type_ctl| is {\tt explicit} or {\tt Chebyshev}.)
Maximum radius of the domains \verb|[Rmax]| is defined by real. If this value is not defined, CMB becomes outer boundary of the domain.

\paragraph{\tt r\_layer}
\label{href_t:r_layer}
\verb|[Layer #]   [Radius]| \\
(This option works with \verb|[radial_grid_type_ctl]| is {\tt explicit}.)
List of the radial grid points in the simulation domain. Index of the radial point \verb|[Layer #]| is defined by integer, and radius \verb|[Radius]| is defined by real.

\paragraph{\tt array boundaries\_ctl}
\verb|[Boundary_name]  [Layer #]| \\
\label{href_t:boundaries_ctl} 
(This option works with \verb|[radial_grid_type_ctl]| is {\tt explicit}.)
Boundaries of the simulation domain is defined by \verb|[Layer #]| in \verb|[r_layer]| array. The following boundary name can be defined for \verb|[Boundary_name]|.
%
\begin{description}
	\item{\tt to\_Center} Inner boundary of the domain to fill the center.
	\item{\tt ICB} ICB
	\item{\tt CMB} CMB
\end{description}
%
%
%

\subsection{\tt phys\_values\_ctl}
\label{href_t:phys_values_ctl}
Fields for the simulation are defined in this block. \\
\hyperref[href_i:phys_values_ctl]{(Back to {\tt control\_MHD})}
%
\paragraph{\tt array nod\_value\_ctl}
\label{href_t:nod_value_ctl}
\verb|[Field] [Viz_flag] [Monitor_flag]| \\
Fields name \verb|[Field]| for the simulation are listed in this array. If required fields for simulation are not in the list, simulation program adds required field in the list, but does not output any field data and monitoring data. \verb|[Viz_flag]| is set to output of the field data for visualization by
%
\begin{description}
\item{\tt VIz\_On}  Write field data to VTK file
\item{\tt VIz\_Off} Do not write field data to VTK file.
\end{description}
%
In the \verb|[Monitor_flag]|, output in the monitoring data is defined by
%
\begin{description}
\item{\tt Monitor\_On}  Write spectrum into monitoring data
\item{\tt Monitor\_Off} Do not write spectrum into monitoring data
\end{description}
%
Supported field in the present version is listed in Table \ref{table:fields}
%
\begin{table}[htp]
\caption{List of field name}
\begin{center}
\begin{tabular}{|c|c|c|}
\hline
\verb|[Name]| & field name & Description \\ \hline \hline
\verb|velocity| &    Velocity &  \bvec{u} \\
\verb|vorticity| &   Vorticity & $\bvec{\omega} = \nabla \times \bvec{u} $ \\
\verb|pressure| &    Pressure & $P$ \\
\hline
\verb|temperature| & Temperature & $T$ \\
\verb|perturbation_temp| & Perturbation of temperature
& $\Theta = T - T_{0}$ \\
\verb|heat_source| & Heat source
& $q_{T}$ \\
\hline
\verb|composition| & Composition variation & $C$ \\
\verb|composition_source| & Composition source & $q_{C}$ \\
\hline
\verb|magnetic_field| &  Magnetic field  & $\bvec{B}$ \\
\verb|current_density| & Current density & $\bvec{J} = \nabla \times \bvec{B} $ \\
\verb|electric_field| & Electric field & $\bvec{E} = \sigma \left(\bvec{J} - \bvec{u} \times \bvec{B}\right) $ \\
 \hline
\verb inertia & Inertia term &  $ -\bvec{\omega} \times \bvec{u} $ \\
\verb|viscous_diffusion| & Viscous diffusion
& $-\nu \nabla \times \nabla \times \bvec{u}$ \\
\verb buoyancy                   & Thermal buoyancy &  $ -\alpha_{T} T \bvec{g}  $ \\
\verb composite_buoyancy & Compositional buoyancy &  $ -\alpha_{C} C\bvec{g}  $\\
\verb Lorentz_force & Lorentz force &  $ \bvec{J} \times \bvec{B} $ \\
\verb Coriolis_force & Coriolis force &  $ -2 \Omega \hat{z} \times \bvec{u} $ \\
\hline
\verb|thermal_diffusion| & Termal diffusion & $ \kappa_{T} \nabla^{2} T $ \\
\verb grad_temp & Temperature gradient & $ \nabla T$ \\
\verb heat_flux & Advective heat flux & $ \bvec{u} T$ \\
\verb heat_advect & Heat advection & $ -\bvec{u} \cdot \nabla T = -\nabla \cdot \left(  \bvec{u} T \right) $ \\
\hline
\verb composition_diffusion & Compositional diffusion & $ \kappa_{C} \nabla^{2} C $ \\
\verb grad_composition & Composition gradient & $ \nabla C$ \\
\verb composite_flux & Advective composition flux & $ \bvec{u} C$ \\
\verb composition_advect & Compositional advection & $ -\bvec{u} \cdot \nabla C = -\nabla \cdot \left(  \bvec{u} C \right) $ \\
\hline
\verb magnetic_diffusion & Magnetic diffusion 
& $-\eta \nabla \times \nabla \times \bvec{B}$ \\
\verb poynting_flux & Poynting flux &  $ \bvec{E} \times \bvec{B} $ \\
\hline
\verb rot_Lorentz_force & Curl of Lorentz force &  $ \nabla \times \left(\bvec{J} \times \bvec{B}\right) $ \\
\verb rot_Coriolis_force & Curl of Coriolis force &  $ -2 \Omega \nabla \times \left(\hat{z} \times \bvec{u} \right) $ \\
\verb rot_buoyancy                   & Curl of thermal buoyancy &  $ - \nabla \times \left(\alpha_{T} T \bvec{g}\right)  $ \\
\verb rot_composite_buoyancy & Curl of compositional buoyancy &  $ - \nabla \times \left(\alpha_{C} C\bvec{g}\right)  $\\
\hline
\verb|buoyancy_flux| & Buoyancy flux & $ -\alpha_{T} T \bvec{g} \cdot \bvec{u} $ \\
\verb|Lorentz_work| & Work of Lorentz force
 & $\bvec{u}\cdot \left( \bvec{J} \times \bvec{B} \right) $ \\ \hline
\end{tabular}
\end{center}
\label{table:fields}
\end{table}
%

\subsection{\tt time\_evolution\_ctl}
\label{href_t:time_evolution_ctl}
Fields for time evolution are defined in this block. \\
\hyperref[href_i:time_evolution_ctl]{(Back to {\tt control\_MHD})}

\paragraph{\tt array time\_evo\_ctl}
\label{href_t:time_evo_ctl}
\verb|[Field]| \\
Fields name for time evolution are listed in this array in \verb|[Field]| by text.
Available fields are listed in Table \ref{table:evolution_field}.
%
\begin{table}[htp]
\caption{List of field name for time evolution}
\begin{center}
\begin{tabular}{|c|c|c|}
\hline
 label & field name & Description \\ \hline
\verb|velocity| &    Velocity &  \bvec{u} \\
\verb|temperature| & Temperature & $T$ \\
\verb|composition| & Composition variation & $C$ \\
\verb|magnetic_field| &  Magnetic field  & $\bvec{B}$ \\ \hline
\end{tabular}
\end{center}
\label{table:evolution_field}
\end{table}

\subsection{\tt boundary\_condition}
\label{href_t:boundary_condition}
Boundary condition are defined in this block. \\
\hyperref[href_i:boundary_condition]{(Back to {\tt control\_MHD})}

\paragraph{\tt array bc\_temperature}
\label{href_t:bc_temperature}
\verb|[Group]  [Type]  [Value]| \\
Boundary conditions for temperature are defined by this array. Position of boundary is defined in \verb|[Group]| column by {\tt ICB} or {\tt CMB}. The following type of boundary conditions are available for temperature in \verb|[Type]| column.
%
\begin{description}
\item{\tt fixed}			Fixed homogeneous temperature on the boundary. The fixed value is defined in \verb|[Value]| by real.
\item{\tt fixed\_file}			Fixed temperature defined by external file. \verb|[Value]| in this line is ignored. See section \ref{sec:boundary_file}.
\item{\tt fixed\_flux}	Fixed homogeneous heat flux on the boundary. The value is defined in \verb|[Value]| by real. Positive value indicates outward flux from fluid shell. ({\it e.g.} Flux to center at ICB and Flux to mantle at CMB are positive.)
\item{\tt fixed\_flux\_file}			Fixed heat flux defined by external file. \verb|[Value]| in this line is ignored.  See section \ref{sec:boundary_file}.
\end{description}
%

\paragraph{\tt array bc\_velocity}
\label{href_t:bc_velocity}
\verb|[Group]  [Type]  [Value]| \\
Boundary conditions for velocity are defined by this array. Position of boundary is defined in \verb|[Group]| by {\tt ICB} or {\tt CMB}. The following boundary conditions are available for velocity in \verb|[Type]| column.
%
\begin{description}
\item{\tt non\_slip\_sph}	Non-slip boundary is applied to the boundary defined in \verb|[Group]|. Real value is required in \verb|[Value]|, but they value is not used in the program.
\item{\tt free\_slip\_sph}	Free-slip boundary is applied to the boundary defined in \verb|[Group]|. Real value is required in \verb|[Value]|, but they value is not used in the program.
\item{\tt rot\_inner\_core} If this condition is set, inner core ($r < r_{i}$) rotation is solved by using viscous torque and Lorentz torque. This boundary condition can be used for {\tt ICB}, and grid is filled to center. Real value is required in \verb|[Value]|, but they value is not used in the program.

\item{\tt rot\_x} Set constant rotation around $x$-axis in \verb|[Value]| by real. Rotation vector can be defined with {\tt rot\_y} and {\tt rot\_z}.
\item{\tt rot\_y} Set constant rotation around $y$-axis in \verb|[Value]| by real. Rotation vector can be defined with {\tt rot\_z} and {\tt rot\_x}.
\item{\tt rot\_z} Set constant rotation around $z$-axis in \verb|[Value]| by real. Rotation vector can be defined with {\tt rot\_x} and {\tt rot\_y}.
\end{description}
%

\paragraph{\tt array bc\_magnetic\_field}
\label{href_t:bc_magnetic_field}
\verb|[Group]  [Type]  [Value]| \\
Boundary conditions for magnetic field are defined by this array. Position of boundary is defined in \verb|[Group]| by {\tt to\_Center}, {\tt ICB}, or {\tt CMB}. The following boundary conditions are available for magnetic field in \verb|[Type]| column.
%
\begin{description}
\item{\tt insulator}	Magnetic field is connected to potential field at boundary defined in [Group]. real value is required at \verb|[Value]|, but they value is not used in the program.
\item{\tt sph\_to\_center}	 If this condition is set, magnetic field in conductive inner core ($r < r_{i}$) is solved. This boundary condition can be used for {\tt ICB}, and grid is filled to center. The value at \verb|[Value]| does not used.
\end{description}
%

\paragraph{\tt array bc\_composition}
\label{href_t:bc_composition}
\verb|[Group]  [Type]  [Value]| \\
Boundary conditions for composition variation are defined by this array. Position of boundary is defined in \verb|[Group]| by {\tt ICB} or {\tt CMB}. The following boundary conditions are available for composition variation in \verb|[Type]| column.
%
\begin{description}
\item{\tt fixed}			Fixed homogeneous composition on the boundary. The fixed value is defined in \verb|[Value]| by real.
\item{\tt fixed\_file}			Fixed composition defined by external file. \verb|[Value]| in this line is ignored. See section \ref{sec:boundary_file}.
\item{\tt fixed\_flux}	Fixed homogeneous compositional flux on the boundary. The value is defined in \verb|[Value]| by real. Positive value indicates outward flux from fluid shell. ({\it e.g.} Flux to center at ICB and Flux to mantle at CMB are positive.)
\item{\tt fixed\_flux\_file}			Fixed compositional flux defined by external file. \verb|[Value]| in this line is ignored. See section \ref{sec:boundary_file}.
\end{description}
%

\subsection{\tt forces\_define}
\label{href_t:forces_define}
Forces for the momentum equation are defined in this block. \\
\hyperref[href_i:forces_define]{(Back to {\tt control\_MHD})}

\paragraph{\tt array force\_ctl}
\label{href_t:force_ctl}
\verb|[Force]| \\
Name of forces for momentum equation are listed in \verb|[Force]| by text.
The following fields are available.
%
\begin{table}[htp]
\caption{List of force}
\begin{center}
\begin{tabular}{|c|c|c|}
\hline
 Label & Field name & Equation \\ \hline
\verb|Coriolis| & Coriolis force & $-2\Omega \hat{z} \times \bvec{u} $ \\
\verb|Lorentz| & Lorentz force &  $\bvec{J} \times \bvec{B} $ \\
\verb|gravity| & Thermal buoyancy & $-\alpha_{T} T \bvec{g}$ \\
\verb|Composite_gravity| & Compositional buoyancy  & $-\alpha_{C} C \bvec{g}$\\ \hline
\end{tabular}
\end{center}
\label{table:forces}
\end{table}
%

\subsection{\tt dimensionless\_ctl}
\label{href_t:dimensionless_ctl}
Dimensionless numbers are defined in this block. \\
\hyperref[href_i:dimensionless_ctl]{(Back to {\tt control\_MHD})}

\paragraph{\tt array dimless\_ctl}
\label{href_t:dimless_ctl}
\verb|[Name] [Value]| \\
Dimensionless are listed in this array. The name is defined in \verb|[Name]| by text, and value is defined in \verb|[Value]| by real. These name of the dimensionless numbers are used to construct coefficients for each terms in governing equations. The following names can not be used because of reserved name in the program.
%
\begin{table}[htp]
\caption{List of reserved name of dimensionless numbers}
\begin{center}
\begin{tabular}{|c|c|c|}
\hline
 label & field name & value \\ \hline
\verb|Zero| & zero & 0.0 \\
\verb|One| &  one &  1.0 \\
\verb|Two| &  two &  2.0 \\
\verb|Radial_35| & Ratio of outer core thickness to whole core & $0.65 = 1 - 0.35$ \\ \hline
\end{tabular}
\end{center}
\label{table:reserved_params}
\end{table}
%

\subsection{\tt coefficients\_ctl}
\label{href_t:coefficients_ctl}
Coefficients of each term in governing equations are defined in this block.
Each coefficients are defined by list of name of dimensionless number \verb|[Name]| and its power \verb|[Power]|. For example, coefficient for Coriolis term for the dynamo benchmark $ 2E^{-1}$ is defined as
%
\begin{verbatim}
        array coef_4_Coriolis_ctl   2
          coef_4_Coriolis_ctl       Two            1.0
          coef_4_Coriolis_ctl       Ekman_number  -1.0
        end array coef_4_Coriolis_ctl
\end{verbatim}
%
\hyperref[href_i:coefficients_ctl]{(Back to {\tt control\_MHD})}

\subsubsection{\tt thermal}
\label{href_t:thermal}
Coefficients of each term in heat equation are defined in this block. \\
\hyperref[href_i:thermal]{(Back to {\tt control\_MHD})}

\paragraph{\tt coef\_4\_termal\_ctl}
\label{href_t:coef_4_termal_ctl}
\verb|[Name] [Power]| \\
Coefficient for evolution of temperature $\displaystyle \frac{\partial T}{\partial t}$ and advection of heat $\left(\bvec{u} \cdot \nabla \right) T$ is defined by this array.

\paragraph{\tt coef\_4\_t\_diffuse\_ctl}
\label{href_t:coef_4_t_diffuse_ctl}
\verb|[Name] [Power]| \\
Coefficient for thermal diffusion $\displaystyle \kappa_{T} \nabla^{2} T$ is defined by this array.

\paragraph{\tt coef\_4\_heat\_source\_ctll}
\label{href_t:coef_4_heat_source_ctl}
\verb|[Name] [Power]| \\
Coefficient for heat source $\displaystyle q_{T}$ is defined by this array.

\subsubsection{\tt momentum}
\label{href_t:momentum}
Coefficients of each term in momentum equation are defined in this block. \\
\hyperref[href_i:momentum]{(Back to {\tt control\_MHD})}

\paragraph{\tt coef\_4\_velocity\_ctl}
\label{href_t:coef_4_velocity_ctl}
\verb|[Name] [Power]| \\
Coefficient for evolution of velocity $\displaystyle \frac{\partial \bvec{u}}{\partial t}$ (or $\displaystyle \frac{\partial \bvec{\omega}}{\partial t}$ for the vorticity equation) and advection $-\bvec{\omega} \times \bvec{u}$ (or $- \nabla \times \left(\bvec{\omega} \times \bvec{u} \right)$ for the vorticity equation) is defined by this array.

\paragraph{\tt coef\_4\_press\_ctl}
\label{href_t:coef_4_press_ctl}
\verb|[Name] [Power]| \\
Coefficient for pressure gradient $-\nabla P$ is defined by this array. Pressure does not appear the vorticity equation which is used for the time integration. But this coefficient is used to evaluate pressure field.

\paragraph{\tt coef\_4\_v\_diffuse\_ctl}
\label{href_t:coef_4_v_diffuse_ctl}
\verb|[Name] [Power]| \\
Coefficient for viscous diffusion $- \nu \nabla \times \nabla \times \bvec{u}$ is defined by this array.

\paragraph{\tt coef\_4\_buoyancy\_ctl}
\label{href_t:coef_4_buoyancy_ctl}
\verb|[Name] [Power]| \\
Coefficient for buoyancy $- \alpha_{T} T \bvec{g}$ is defined by this array.

\paragraph{\tt coef\_4\_Coriolis\_ctl}
\label{href_t:coef_4_Coriolis_ctl}
\verb|[Name] [Power]| \\
Coefficient for Coriolis force $-2 \Omega \hat{z} \times \bvec{u}$ is defined by this array.

\paragraph{\tt coef\_4\_Lorentz\_ctl}
\label{href_t:coef_4_Lorentz_ctl}
\verb|[Name] [Power]| \\
Coefficient for Lorentz force $ \rho_{0}^{-1} \bvec{J} \times \bvec{B}$ is defined by this array.

\paragraph{\tt coef\_4\_composit\_buoyancy\_ctl}
\label{href_t:coef_4_composit_buoyancy_ctl}
\verb|[Name] [Power]| \\
Coefficient for compositional buoyancy $ -\alpha_{C} C \bvec{g}$ is defined by this array.

\subsubsection{\tt induction}
\label{href_t:induction}
Coefficients of each term in magnetic induction equation are defined in this block. \\
\hyperref[href_i:induction]{(Back to {\tt control\_MHD})}

\paragraph{\tt coef\_4\_magnetic\_ctl}
\label{href_t:coef_4_magnetic_ctl}
\verb|[Name] [Power]| \\
Coefficient for evolution of temperature $\displaystyle \frac{\partial \bvec{B}}{\partial t}$ is defined by this array.

\paragraph{\tt coef\_4\_m\_diffuse\_ctl}
\label{href_t:coef_4_m_diffuse_ctl}
\verb|[Name] [Power]| \\
Coefficient for magnetic diffusion $ -\eta \nabla \times \nabla \times \bvec{B}$ is defined by this array.

\paragraph{\tt coef\_4\_induction\_ctl}
\label{href_t:coef_4_induction_ctl}
\verb|[Name] [Power]| \\
Coefficient for magnetic induction $\nabla \times \left(\bvec{u} \times \bvec{B} \right)$ is defined by this array.

\subsubsection{\tt composition}
\label{href_t:composition}
Coefficients of each term in composition equation are defined in this block. \\
\hyperref[href_i:composition]{(Back to {\tt control\_MHD})}

\paragraph{\tt coef\_4\_composition\_ctl}
\label{href_t:coef_4_composition_ctl}
\verb|[Name] [Power]| \\
Coefficient for evolution of composition variation $\displaystyle \frac{\partial C}{\partial t}$ and advection of heat $\left(\bvec{u} \cdot \nabla \right) C$ is defined by this array.

\paragraph{\tt coef\_4\_c\_diffuse\_ctl}
\label{href_t:coef_4_c_diffuse_ctl}
\verb|[Name] [Power]| \\
Coefficient for compositional diffusion $\displaystyle \kappa_{C} \nabla^{2} C$ is defined by this array.

\paragraph{\tt coef\_4\_composition\_source\_ctll}
\label{href_t:coef_4_composition_source_ctl}
\verb|[Name] [Power]| \\
Coefficient for composition source $\displaystyle q_{C}$ is defined by this array.

% \subsection{\tt gravity\_define}
% \label{href_t:gravity_define}
% Gravity (buoyancy) vector is defined in this block \\
% \hyperref[href_i:gravity_define]{(Back to {\tt control\_MHD})} 
%
% \paragraph{\tt gravity\_type\_ctl}
% \label{href_t:gravity_type_ctl}
% \verb|[Direction]  [Value]| \\
% Gravity (buoyancy) type is defined by text. The following setting is available.
% \begin{description}
% \item{\tt radial} Gravity vector goes to center and is proportional to the radius $\bvec{g} = -\bvec{r}$. This model generally used to the geodyanmo simulations. 
% \item{\tt constant\_radial} Gravity vector goes to center and has constants amplitude  $\bvec{g} = -\bvec{r} / r$. This model generally used to the geodyanmo simulations. 
% \end{description}
%
% \subsection{\tt Coriolis\_define}
% \label{href_t:Coriolis_define}
% Rotation of the system for Coriolis force is defined in this block. \\
% \hyperref[href_i:Coriolis_define]{(Back to {\tt control\_MHD})}
%
% \paragraph{\tt array rotation\_vec}
% \label{href_t:rotation_vec}
% \verb|[Direction]  [Value]| \\
% Rotation vector of the system is defined by array. {\tt x}, {\tt y}, or {\tt z} is set in \verb|[Direction]|, and each component of the rotation vector is set in the \verb|[Value]| as real. In this program, the rotation vector does NOT normalized.

\subsection{\tt temperature\_define}
\label{href_t:temperature_define}
Reference of temperature $T_{0}$ is defined in this block. If reference of temperature is defined, perturbation of temperature $\Theta = T - T_{0}$ is used for time evolution and buoyancy. \\
\hyperref[href_i:temperature_define]{(Back to {\tt control\_MHD})}

\paragraph{\tt ref\_temp\_ctl}
\label{href_t:ref_temp_ctl}
\verb|[REFERENCE_TEMP]| \\
Type of reference temperature is defined by text. The following options are available for \verb|[REFERENCE_TEMP]|.
%
\begin{description}
\item{\tt none}   Reference of temperature is not defined. Temperature $T$ is used to time evolution and thermal buoyancy.
\item{\tt spherical\_shell} Reference of temperature is set by
\begin{eqnarray}
 T_{0} = \frac{1}{\left(r_{h}-r_{l} \right)} \left[
          r_{l}T_{l} - r_{h}T_{h} + \frac{r_{l} r_{h}}{r} \left(T_{h}-T_{l}\right) \right].
\nonumber
\end{eqnarray}
\end{description}
%

\paragraph{\tt low\_temp\_ctl}
\label{href_t:low_temp_ctl}
Amplitude of low reference temperature $T_{l}$ and its radius $r_{l}$ (Generally $r_{l} = r_{o}$) are defined in this block.

\paragraph{\tt high\_temp\_ctl}
\label{href_t:high_temp_ctl}
Amplitude of high reference temperature $T_{h}$ and its radius $r_{h}$ (Generally $r_{h} = r_{i}$) are defined in this block.

\paragraph{\tt depth}
\label{href_t:depth}
\verb|[RADIUS]| \\
Radius for reference temperature is defined by real.

\paragraph{\tt temperature}
\label{href_t:temperature}
\verb|[TEMPERATURE]| \\
Temperature for reference temperature is defined by real.


\subsection{\tt time\_step\_ctl}
\label{href_t:time_step_ctl}
Time stepping parameters are defined in this block. \\
\hyperref[href_i:time_step_ctl]{(Back to {\tt control\_MHD})} \\
\hyperref[href_i:time_step_ctl2]{(Back to {\tt control\_assemble\_sph)}}

\paragraph{\tt elapsed\_time\_ctl}
\label{href_t:elapsed_time_ctl}
\verb|[ELAPSED_TIME]| \\
Elapsed (wall clock) time (second) for simulation \verb|[ELAPSED_TIME]| is defined by real. 
This parameter varies if end step \verb|[ISTEP_FINISH]| is defined to {\tt -1}. If simulation runs for given time, program output spectrum data  \verb|[rst_prefix].elaps.[process #].fst| immediately, and finish the simulation.

\paragraph{\tt i\_step\_init\_ctl}
\label{href_t:i_step_init_ctl}
\verb|[ISTEP_START]| \\
Start step of simulation \verb|[ISTEP_START]| is defined by integer. if \verb|[ISTEP_START]| is set to {\tt -1} and \verb|[INITIAL_TYPE]| is set to \verb|start_from_rst_file|, program read spectrum data file \verb|[rst_prefix].elaps.[process #].fst| and start the simulation.

\paragraph{\tt i\_step\_finish\_ctl}
\label{href_t:i_step_finish_ctl}
\verb|[ISTEP_FINISH]| \\
End step of simulation \verb|[ISTEP_FINISH]| is defined by integer. If this value is set to  {\tt -1}, simulation stops when elapsed time reaches to \verb|[ELAPSED_TIME]|.

\paragraph{\tt i\_step\_check\_ctl}
\label{href_t:i_step_check_ctl}
\verb|[ISTEP_MONITOR]| \\
Increment of time step for monitoring data \verb|[ISTEP_MONITOR]| is defined by integer.

\paragraph{\tt i\_step\_rst\_ctl}
\label{href_t:i_step_rst_ctl}
\verb|[ISTEP_RESTART]| \\
Increment of time step to output spectrum data for restarting \verb|[ISTEP_RESTART]| is defined by integer.

\paragraph{\tt i\_step\_field\_ctl}
\label{href_t:i_step_field_ctl}
\verb|[ISTEP_FIELD]| \\
Increment of time step to output field data for visualization \verb|[ISTEP_FIELD]| is defined by integer. If \verb|[ISTEP_FIELD]| is set to be 0, no field data are written.

\paragraph{\tt i\_step\_sectioning\_ctl}
\label{href_t:i_step_sectioning_ctl}
\verb|[ISTEP_SECTION]| \\
Increment of time step to output cross section data for visualization \verb|[ISTEP_SECTION]| is defined by integer. If \verb|[ISTEP_SECTION]| is set to be 0, no cross section data are written. If \verb|[ISTEP_SECTION]| is set in the block \hyperref[href_i:visual_control]{\tt visual\_control}, The value in {\tt visual\_control} is used.

\paragraph{\tt i\_step\_isosurface\_ctl}
\label{href_t:i_step_isosurface_ctl}
\verb|[ISTEP_ISOSURFACE]| \\
Increment of time step to output isosurface data for visualization \verb|[ISTEP_ISOSURFACE]| is defined by integer. If \verb|[ISTEP_ISOSURFACE]| is set to be 0, no isosurface data are written.If \verb|[ISTEP_ISOSURFACE]| is set in the block \hyperref[href_i:visual_control]{\tt visual\_control}, The value in {\tt visual\_control} is used.

\paragraph{\tt dt\_ctl}
\label{href_t:dt_ctl}
\verb|[DELTA_TIME]| \\
Length of time step $\Delta t$ is defined by real value.

\paragraph{\tt time\_init\_ctl}
\label{href_t:time_init_ctl}
\verb|[INITIAL_TIME]| \\
Initial time $t_{0}$ is defined by real value. This value is ignored if simulation starts from restart data.

\subsection{\tt new\_time\_step\_ctl}
\label{href_t:new_time_step_ctl}
Time stepping parameters to update initial data are defined in this block. Items in this block is the same as \hyperref[href_t:i_step_field_ctl]{\tt i\_step\_field\_ctl}.
\hyperref[href_i:new_time_step_ctl]{(Back to {\tt control\_assemble\_sph)}}


\subsection{\tt restart\_file\_ctl}
\label{href_t:restart_file_ctl}
Initial field for simulation is defined in this block.\\
\hyperref[href_t:restart_file_prefix]{(Back to {\tt control\_MHD})}

\paragraph{\tt rst\_ctl}
\label{href_t:rst_ctl}
\verb|[INITIAL_TYPE]| \\
Type of Initial field is defined by text. The following parameters are available for \verb|[INITIAL_TYPE]|.
%
\begin{description}
\item{\tt No\_data}  No initial data file. Small temperature perturbation and seed magnetic field are set as an initial field.
\item{\tt start\_from\_rst\_file} Initial field is read from spectrum data file. File prefix is defined by \hyperref[href_t:restart_file_prefix]{$\mbox{\tt restart\_file\_prefix}$}.
\item{\tt Dynamo\_benchmark\_0}   Generate initial field for dynamo benchmark case 0
\item{\tt Dynamo\_benchmark\_1}   Generate initial field for dynamo benchmark case 1
\item{\tt Dynamo\_benchmark\_2}   Generate initial field for dynamo benchmark case 2
\item{\tt Pseudo\_vacuum\_benchmark} Generate initial field for pseudo vacuum dynamo benchmark
\end{description}
%

\subsection{\tt time\_loop\_ctl}
\label{href_t:time_loop_ctl}
Time evolution scheme is defined in this block. \\
\hyperref[href_i:time_loop_ctl]{(Back to {\tt control\_MHD})}

\paragraph{\tt scheme\_ctl}
\label{href_t:scheme_ctl}
\verb|[EVOLUTION_SCHEME]| \\
Time evolution scheme is defined by text. Currently, Crank-Nicolson scheme is only available for diffusion terms.
%
\begin{description}
\item{\tt Crank\_Nicolson} Crank-Nicolson scheme for diffusion terms and second order Adams-Bashforth scheme the other terms.
% \item{\tt 2nd\_Adams\_Bashforth}  Second order Adams-Bashforth scheme
% \item{\tt explicit\_Euler} First order Euler scheme.
\end{description}
%

\paragraph{\tt coef\_imp\_v\_ctl}
\label{href_t:coef_imp_v_ctl}
\verb|[COEF_INP_U]| \\
Coefficients for the implicit parts of the Crank-Nicolson scheme for viscous diffusion \verb|[COEF_INP_U]| is defined by real.

\paragraph{\tt coef\_imp\_t\_ctl}
\label{href_t:coef_imp_t_ctl}
\verb|[COEF_INP_T]| \\
Coefficients for the implicit parts of the Crank-Nicolson scheme for thermal diffusion \verb|[COEF_INP_T]| is defined by real.

\paragraph{\tt coef\_imp\_b\_ctl}
\label{href_t:coef_imp_b_ctl}
\verb|[COEF_INP_B]| \\
Coefficients for the implicit parts of the Crank-Nicolson scheme for magnetic diffusion \verb|[COEF_INP_B]| is defined by real.

\paragraph{\tt coef\_imp\_c\_ctl}
\label{href_t:coef_imp_c_ctl}
\verb|[COEF_INP_C]| \\
Coefficients for the implicit parts of the Crank-Nicolson scheme for compositional diffusion \verb|[COEF_INP_C]| is defined by real. 


\paragraph{\tt FFT\_library\_ctl}
\label{href_t:FFT_library_ctl}
\verb|[FFT_Name]| \\
FFT library name for Fourier transform is defined by text. The following libraries are available for \verb|[FFT_Name]|. 
If this flag is not defined, program searches the fastest library in the initialization process.
%
\begin{description}
\item{\tt FFTW}		Use FFTW
\item{\tt FFTPACK}	Use FFTPACK
% \item{\tt ISPACK}	Use ISPACK
\end{description}
%

\paragraph{\tt Legendre\_trans\_loop\_ctl}
\label{href_t:Legendre_trans_loop_ctl}
\verb|[FFT_Name]| \\
Loop configuration for Legendre transform is defined by text. The following settings are available for \verb|[Leg_Loop]|. 
If this flag is not defined, program searches the fastest approarch in the initialization process.
%
\begin{description}
\item{\tt Inner\_radial\_loop}	Loop for the radial grids is set as the innermost loop
\item{\tt Outer\_radial\_loop}	Loop for the radial grids is set as the outermost loop
\item{\tt Long\_loop}	        Long one-dimentional loop is used
\end{description}
%

%
\subsection{\tt sph\_monitor\_ctl}
\label{href_t:sph_monitor_ctl}
Monitoring data is defined in this block. Monitoring data output (mean square, average, Gauss coefficients, or specific components of spectrum data) are flagged by {\tt Monitor\_On} in \hyperref[href_t:nod_value_ctl]{ {\tt nod\_value\_ctl} array}. \\
\hyperref[href_i:sph_monitor_ctl]{(Back to {\tt control\_MHD})}

\paragraph{\tt volume\_average\_prefix}
\label{href_t:volume_average_prefix}
\verb|[vol_ave_prefix]| \\
File prefix for volume average data \verb|[vol_ave_prefix]| is defined by Text. Program add {\tt .dat} extension after this file prefix. If this file prefix is not defined, volume average data are not generated. 

\paragraph{\tt volume\_pwr\_spectr\_prefix}
\label{href_t:volume_pwr_spectr_prefix}
\verb|[vol_pwr_prefix]| \\
File prefix for mean square spectrum data averaged over the fluid shell \verb|[vol_pwr_prefix]| is defined by Text. 

Spectrum as a function of degree {l} is written in \verb|[vol_pwr_prefix])_l.dat|, spectrum as a function of order {m} is written in \verb|[vol_pwr_prefix]_m.dat|, and spectrum as a function of $(l-m)$ is written in \verb|[vol_pwr_prefix]_lm.dat|. This prefix is also used for the file name of the volume mean square data as \verb|[vol_pwr_prefix]_s.dat|.
If this file prefix is not defined, volume spectrum data are not generated and volume mean square data is written as \verb|sph_pwr_volume_s.dat|.

\paragraph{\tt nusselt\_number\_prefix}
\label{href_t:nusselt_number_prefix}
\verb|[nusselt_number_prefix]| \\
File prefix for Nusselt number data at ICB and CMB \verb|[nusselt_number_prefix]| is defined by Text. Program add {\tt .dat} extension after this file prefix. If this file prefix is not defined, Nusselt number data are not generated. \\
{\bf CAUTION: Nusselt number is not evaluated if heat source exsists.}

%
\subsubsection{\tt volume\_spectrum\_ctl}
\label{href_t:volume_spectrum_ctl}
Volume average of power spectrum and mean square data between any radius range are defined in this block.

\paragraph{\tt inner\_radius\_ctl}
\label{href_t:inner_radius_ctl}
\verb|[radius]| \\
Inner boundary of the volume average \verb|[radius]| is defined. The closest radial grid point is chosen as a inner boundary of averaging.

\paragraph{\tt outer\_radius\_ctl}
\label{href_t:outer_radius_ctl}
\verb|[radius]| \\
Outer boundary of the volume average \verb|[radius]| is defined. The closest radial grid point is chosen as a outer boundary of averaging.

%
\subsubsection{\tt layered\_spectrum\_ctl}
\label{href_t:layered_spectrum_ctl}
Sphere average of power spectrum and mean square data are defined in this block.

\paragraph{\tt layered\_pwr\_spectr\_prefix}
\label{href_t:layered_pwr_spectr_prefix}
\verb|[layer_pwr_prefix]| \\
File prefix for mean square spectrum data averaged over each sphere surface \verb|[layer_pwr_prefix]| is defined by Text.

Spectrum as a function of degree {l} is written in \verb|[layer_pwr_prefix]_l.dat|, spectrum as a function of order {m} is written in \verb|[layer_pwr_prefix]_m.dat|, and spectrum as a function of $(l-m)$ is written in \verb|[layer_pwr_prefix]_lm.dat|. If this file prefix is not defined, sphere averaged spectrum data are not generated. 

\paragraph{\tt array spectr\_layer\_ctl}
\label{href_t:spectr_layer_ctl}
\verb|[Layer #]|
List of radial grid point number \verb|[Layer #]| to output power spectrum data by integer. If this array is not defined, layered mean square data are written for all radial grid points.

%
\subsubsection{\tt gauss\_coefficient\_ctl}
\label{href_t:gauss_coefficient_ctl}
Gauss coefficients data at specified radius are defined in this block.

\paragraph{\tt gauss\_coefs\_prefix}
\label{href_t:gauss_coefs_prefix}
\verb|[gauss_coef_prefix]| \\
File prefix for Gauss coefficients \verb|[gauss_coef_prefix]| is defined by Text. Program add {\tt .dat} extension after this file prefix. If this file prefix is not defined, Gauss coefficients data are not generated. 

\paragraph{\tt gauss\_coefs\_radius\_ctl}
\label{href_t:gauss_coefs_radius_ctl}
\verb|[gauss_coef_radius]| \\
Normalized radius to obtain Gauss coefficients \verb|[gauss_coef_radius]| is defined by real. Gauss coefficients are evaluated from the poloidal magnetic field at CMB by assuming electrically insulated mantle. Do not set \verb|[gauss_coef_radius]| less than the outer core radius $r_{o}$.

\paragraph{\tt array pick\_gauss\_coefs\_ctl}
\label{href_t:pick_gauss_coefs_ctl}
\verb|[Degree]  [Order]| \\
List of spherical harmonics mode $l$ and $m$ of Gauss coefficients to output. \verb|[Degree]| and \verb| [Order]| are defined by integer.

\paragraph{\tt array pick\_gauss\_coef\_degree\_ctl}
\label{href_t:pick_gauss_coef_degree_ctl}
\verb|[Degree]| \\
Degrees $l$ to output Gauss coefficients are listed in \verb|[Degree]| by integer. All Gauss coefficients with listed $l$ is output in file.

\paragraph{\tt array pick\_gauss\_coef\_order\_ctl}
\label{href_t:pick_gauss_coef_order_ctl}
\verb|[Order]| \\
Orders $m$ to output Gauss coefficients are listed in \verb|[Order]| by integer. All Gauss coefficients with listed order $m$ is output in file.

%
\subsubsection{\tt pickup\_spectr\_ctl}
\label{href_t:pickup_spectr_ctl}
Spherical harmonic coefficients data output is defined in this block.

\paragraph{\tt picked\_sph\_prefix}
\label{href_t:picked_sph_prefix}
\verb|[picked_sph_prefix]| \\
File prefix for picked spectrum data \verb|[picked_sph_prefix]| is defined by Text. Program add {\tt .dat} extension after this file prefix. If this file prefix is not defined, picked spectrum data are not generated. 

\paragraph{\tt array pick\_layer\_ctl}
\label{href_t:pick_layer_ctl}
\verb|[Layer #]|
List of radial grid point number \verb|[Layer #]| to output picked spectrum data by integer. If this array is not defined, picked spectrum data are written for all radial grid points.

\paragraph{\tt array pick\_sph\_spectr\_ctl}
\label{href_t:pick_sph_spectr_ctl}
\verb|[Degree]  [Order]| \\
List of spherical harmonics mode $l$ and $m$ of spectrum data to output. \verb|[Degree]| and \verb| [Order]| are defined by integer.

\paragraph{\tt array pick\_sph\_degree\_ctl}
\label{href_t:pick_sph_degree_ctl}
\verb|[Degree]| \\
Degrees $l$ to output spectrum data are listed in \verb|[Degree]| by integer. All spectrum data with listed degree $l$ is output in file.

\paragraph{\tt array pick\_sph\_order\_ctl}
\label{href_t:pick_sph_order_ctl}
\verb|[Order]| \\
Order $m$ to output spectrum data are listed in \verb|[Order]| by integer. All spectrum data with listed order $m$ is output in file.


\subsubsection{\tt mid\_equator\_monitor\_ctl}
\label{href_t:mid_equator_monitor_ctl}
Parameters to generate data at mid-depth of euqtorial plane are defined in this block.

\paragraph{\tt nphi\_mid\_eq\_ctl}
\label{href_t:nphi_mid_eq_ctl}
\verb|[Nphi_mid_equator]| \\
Number of grid points \verb|[Nphi_mid_equator]|in longitudinal direction to evaluate mid-depth of the shell in the equatorial plane for dynamo benchmark is defined as integer. If \verb|[Nphi_mid_equator]| is not defined or less than zero, \verb|[Nphi_mid_equator]| is set set number grid as the input spherical transform data. 
%
%

\subsection{\tt visual\_control}
\label{href_t:visual_control}
Visualization modules are defined in this block. Parameters for cross sections and isosurfaces are defined in this block. \\
\hyperref[href_i:visual_control]{(Back to {\tt visual\_control})}

%
%
\subsection{\tt cross\_section\_ctl}
\label{href_t:cross_section_ctl}
Control parameters for cross sectioning are defined in this block. \\
\hyperref[href_i:cross_section_ctl]{(Back to {\tt cross\_section\_ctl)}}

\paragraph{\tt section\_file\_prefix}
\label{href_t:section_file_prefix}
\verb|[file_prefix]| \\
File prefix for cross section data is defined as character \verb|[file_prefix]|.

\paragraph{\tt psf\_output\_type}
\label{href_t:psf_output_type}
\verb|[file_format]| \\
File format for cross section data is defined as character \verb|[file_format]|. The following formats are available;
\begin{description}
\item{\tt VTK: }               VTK format
\item{\tt VTK\_gz: }           Compressed VTK format (Available if zlib library is linked)
\item{\tt PSF: }               Binary section data format
\item{\tt PSF\_gzip: }         Compressed Binary section data format (Available if zlib library is linked)
\end{description}

\subsubsection{\tt surface\_define}
\label{href_t:surface_define}
Each cross section is defined in this block. \\
\hyperref[href_i:cross_section_ctl]{(Back to {\tt cross\_section\_ctl)}} \\

\paragraph{\tt section\_method}
\label{href_t:section_method}
\verb|[METHOD]| \\
Method of the cross sectioning is defined as character \verb|[METHOD]|. Supported cross section is shown in Table \ref{table:surface_list}
%
\begin{table}[htp]
\caption{Supported cross sections}
\begin{center}
\begin{tabular}{|c|c|}
\hline
\verb|[METHOD]| & Surface type \\ \hline
\verb|equation| & Quadrature surface \\
% & $a x^2 + b y^2 + c z^2 + d y x + e z x + f x y + g x + h y + j z + k = 0$ \\
\verb|plane| & Plane surface \\
%& $a \left(x-x_{0} \right) + b \left(y-y_{0} \right) + c \left(z-z_{0} \right) = 0$ \\
\verb|sphere| & Sphere \\
%& $\left(x-x_{0} \right)^2 + \left(y-y_{0} \right)^2 + \left(z-z_{0} \right)^2 = r^2$  \\
\verb|ellipsoid| & Ellipsoid  \\
%& $\left(\frac{x-x_{0}}{a} \right)^2 + \left( \frac{y-y_{0}}{b} \right)^2 + \left( \frac{z-z_{0}}{c} \right)^2 = 1$ \\
\hline
\end{tabular}
\end{center}
\label{table:surface_list}
\end{table}
%

\paragraph{\tt coefs\_ctl}
\label{href_t:psf_coefs_ctl}
\verb|[TERM]	[COEFFICIENT]| \\
This array defines coefficients for a quadrature surface described by 
\begin{eqnarray*}
a x^2 + b y^2 + c z^2 + d y z + e z x + f x y + g x + h y + j z + k &=& 0.
\end{eqnarray*}
Each coefficient $a$ to $k$ are defined by the name of the term \verb|[TERM]| and real value \verb|[COEFFICIENT]| as shown in Table \ref{table:psf_coefs}.
%
\begin{table}[htp]
\caption{List of coefficient labels for quadrature surface}
\begin{center}
\begin{tabular}{|c|c||c|c||c|c|}
\hline
\verb|[TERM]| & Defined value & \verb|[TERM]| & Defined value & \verb|[TERM]| & Defined value \\ \hline
\verb|x2| & $a$ & \verb|y2| & $b$  & \verb|z2| & $c$ \\
\verb|yz| & $d$ & \verb|zx| & $e$  & \verb|xy| & $f$ \\
\verb|x | & $g$ & \verb|y| & $h$  & \verb|z| & $i$ \\
\verb|const| & $h$ &  &   & &  \\ \hline
\end{tabular}
\end{center}
\label{table:psf_coefs}
\end{table}
%

\paragraph{\tt radius}
\label{href_t:psf_radius}
\verb|[SIZE]| \\
\verb|[SIZE]| defines radius $r$ for a sphere surface defined by 
\begin{eqnarray*}
\left(x-x_{0} \right)^2 + \left(y-y_{0} \right)^2 + \left(z-z_{0} \right)^2 = r^2. 
\end{eqnarray*}

\paragraph{\tt normal\_vector}
\label{href_t:psf_normal_vector}
\verb|[DIRECTION]	[COMPONENT]| \\
This array defines normal vector $(a, b, c)$ for a plane surface described by 
\begin{eqnarray*}
a \left(x-x_{0} \right) + b \left(y-y_{0} \right) + c \left(z-z_{0} \right) = 0. 
\end{eqnarray*}
Each component is defined by \verb|[DIRECTION]| and real value \verb|[COMPONENT]| as shown in Table \ref{table:psf_normal}.
%
\begin{table}[htp]
\caption{List of coefficient labels for vector}
\begin{center}
\begin{tabular}{|c|c|}
\hline
\verb|[DIRECTION]| & Defined value \\ \hline
\verb|x| & $a$ \\
\verb|y| & $b$ \\
\verb|z| & $c$ \\ \hline
\end{tabular}
\end{center}
\label{table:psf_normal}
\end{table}
%

\paragraph{\tt axial\_length}
\label{href_t:psf_axial_length}
\verb|[DIRECTION]	[COMPONENT]| \\
This array defines size $(a, b, c)$ of an ellipsoid surface described by 
\begin{eqnarray*}
\left(\frac{x-x_{0}}{a} \right)^2 + \left( \frac{y-y_{0}}{b} \right)^2 + \left( \frac{z-z_{0}}{c} \right)^2 = 1. 
\end{eqnarray*}
Each component is defined by \verb|[DIRECTION]| and real value \verb|[COMPONENT]| as shown in Table \ref{table:psf_normal}.
%

\paragraph{\tt center\_position}
\label{href_t:psf_center_position}
\verb|[DIRECTION]	[COMPONENT]| \\
Position of center $(x_{0}, y_{0}, z_{0})$ of sphere or ellipsoid is defined this array. Position on a plane surface $(x_{0}, y_{0}, z_{0})$ is also defined. Each component is defined by \verb|[DIRECTION]| and real value \verb|[COMPONENT]| as shown in Table \ref{table:psf_position}.
%
\begin{table}[htp]
\caption{List of coefficient labels for vector}
\begin{center}
\begin{tabular}{|c|c|}
\hline
\verb|[DIRECTION]| & Defined value \\ \hline
\verb|x| & $x_{0}$ \\
\verb|y| & $y_{0}$ \\
\verb|z| & $z_{0}$ \\ \hline
\end{tabular}
\end{center}
\label{table:psf_position}
\end{table}
%

\paragraph{\tt section\_area\_ctl}
\label{href_t:section_area_ctl}
Areas for the cross sectioning are defined in this array. The following groups can be defined in this block.
%
\begin{description}
	\item{\tt outer\_core} Outer core.
	\item{\tt inner\_core} Inner core (If exist).
	\item{\tt external} External of the core (If exist).
	\item{\tt all} Whole simulation domain.
\end{description}

%
\subsubsection{\tt output\_field\_define}
\label{href_t:output_field_define}
Field data on the cross section are defined in this block. \\
\hyperref[href_i:cross_section_ctl]{(Back to {\tt cross\_section\_ctl)}} \\

%
\paragraph{\tt output\_field}
\label{href_t:psf_output_field}
Field informations for cross section are defined in this array. Name of the output fields is defined by \verb|[FIELD]|, and component of the fields is defined by \verb|[COMPONENT]|. Labels of the field name are listed in Table \ref{table:fields}, and labels of the component are listed in Table \ref{table:components}. \\
%
\begin{table}[htp]
\caption{List of field type for cross sectioning and isosurface module}
\label{table:components}
\begin{center} 
\begin{tabular}{|c|c|}
\hline
 \verb|[COMPONENT]| & Field type  \\ \hline \hline
 \verb|scalar| & scalar field  \\ \hline
 \verb|vector| & Cartesian vector field \\ \hline
 \verb|x| & $x$-component  \\ \hline
 \verb|y| & $y$-component  \\ \hline
 \verb|z| & $z$-component  \\ \hline
 \verb|radial| & radial ($r$-) component  \\ \hline
 \verb|theta| & $\theta$-component  \\ \hline
 \verb|phi| & $\phi$-component  \\ \hline
 \verb|cylinder_r| & cylindrical radial ($s$-) component  \\ \hline
 \verb|magnitude| & magnitude of vector  \\ \hline
\end{tabular}
\end{center}
\end{table}
%
%


\subsection{\tt isosurface\_ctl}
\label{href_t:isosurface_ctl}
Control parameters for isosurfacing are defined in this block. \\
\hyperref[href_i:isosurface_ctl]{(Back to {\tt isosurface\_ctl)}}

%
%
\paragraph{\tt isosurface\_file\_prefix}
\label{href_t:isosurface_file_prefix}
\verb|[file_prefix]| \\
File prefix for isosurface data is defined as character \verb|[file_prefix]|.

\paragraph{\tt iso\_output\_type}
\label{href_t:iso_output_type}
File format for isosurface data is defined as character \verb|[file_format]|. The following formats are available;
\begin{description}
\item{\tt VTK: }               VTK format
\item{\tt VTK\_gz: }           Compressed VTK format (Available if zlib library is linked)
\item{\tt ISO: }               Binary isosurface data format
\item{\tt ISO\_gzip: }         Compressed Binary isosurface data format (Available if zlib library is linked)
\end{description}

\subsubsection{\tt isosurf\_define}
\label{href_t:isosurf_define}
Each isosurface is defined in this block. \\
\hyperref[href_i:isosurface_ctl]{(Back to {\tt isosurface\_ctl)}}

\paragraph{\tt isosurf\_field}
\label{href_t:isosurf_field}
Field name for isosurface is defined by \verb|[FIELD]|. Labels of the field name are listed in Table \ref{table:fields}. \\
%
\paragraph{\tt isosurf\_component}
\label{href_t:isosurf_component}
Component name for isosurface is defined by \verb|[COMPONENT]|. Labels of the component are listed in Table \ref{table:components}.

%
\paragraph{\tt isosurf\_value}
\label{href_t:isosurf_value}
Isosurface value is defined as real value \verb|VALUE|.

\paragraph{\tt isosurf\_area\_ctl}
\label{href_t:isosurf_area_ctl}
Areas for the isosurfacing are defined in this array. The same groups can be defined as \hyperref[href_t:psf_output_field]{\tt section\_area\_ctl}.

%
\subsubsection{\tt field\_on\_isosurf}
\label{href_t:field_on_isosurf}
Field data on the isosurface are defined in this block. \\
\hyperref[href_i:isosurface_ctl]{(Back to {\tt isosurface\_ctl)}}

%
\paragraph{\tt result\_type}
\label{href_t:result_type}
Output data type is defined by \verb|[TYPE]|. Following types can be defined:
%
\begin{description}
	\item{\tt constant} Constant value is set as a result field. The amplitude is set by \verb|result_value|.
	\item{\tt field} field data on the isosurface are written. Fields to be written are defined by \verb|output_field| array.
\end{description}

%
\paragraph{\tt result\_value}
\label{href_t:result_value}
Isosurface value is defined as real value \verb|VALUE|.

%
\paragraph{\tt output\_field}
\label{href_t:iso_output_field}
Field informations for cross section are defined in this array. Name of the output fields is defined by \verb|[FIELD]|, and component of the fields is defined by \verb|[COMPONENT]|. Labels of the field name are listed in Table \ref{table:fields}, and labels of the component are listed in Table \ref{table:components}. \\
%
%
\subsection{\tt output\_field\_file\_fmt\_ctl  [VTK\_format]}
\label{href_t:output_field_file_fmt_ctl}
File format of field data is defined as character \verb|[VTK_format]|. THe following formats are available.
%
\begin{description}
\item{\tt single\_HDF5: }  Merged HDF5 file (Available if HDF5 library is linked)
\item{\tt single\_VTK: }   Merged VTK file (Default)
\item{\tt VTK: }           Distributed VTK file
\item{\tt single\_VTK\_gz: }   Compressed merged VTK file (Available if zlib library is linked)
\item{\tt VTK\_gz: }           Compressed distributed VTK file (Available if zlib library is linked)
\end{description}

\subsection{\tt dynamo\_vizs\_control}
\label{href_t:dynamo_vizs_control}
Visualization for zonal mean, RMS, and truncated magnetic field are defined in this block. Parameters for cross section is set for zonal mean and RMS, and spherical harmonics degree of the truncated magnetic field is also defined here. \\
\hyperref[href_i:dynamo_vizs_control]{(Back to {\tt dynamo\_vizs\_control)}}

%
%
\paragraph{\tt zonal\_mean\_section\_ctl}
\label{href_t:zonal_mean_section_ctl}
Control parameters for cross section of the zonal mean field are defined in this block. This block has the same control items as \hyperref[href_i:cross_section_ctl]{\tt cross\_section\_ctl}. In the external file {\tt [zonal\_mean\_section\_control\_file]}, con trol block starts from {\tt cross\_section\_ctl}.


\paragraph{\tt zonal\_RMS\_section\_ctl}
\label{href_t:zonal_RMS_section_ctl}
Control parameters for cross section of the zonal RMS field are defined in this block. This block has the same control items as \hyperref[href_i:cross_section_ctl]{\tt cross\_section\_ctl}. In the external file {\tt [zonal\_RMS\_section\_control\_file]}, con trol block starts from {\tt cross\_section\_ctl}.

\paragraph{\tt crustal\_filtering\_ctl}
\label{href_t:crustal_filtering_ctl}
Set the truncation degree to make the truncated magnetic field by the crustal magnetic field. The spherical harmonics degree of the truncated magnetic field is defined in {\tt truncation\_degree\_ctl}. In the external file {\tt [zonal\_mean\_section\_control\_file]}, con trol block starts from {\tt cross\_section\_ctl}.

%

\subsection{\tt new\_data\_files\_def}
\label{href_t:new_data_files_def}
File names and number of processes for new domain decomposed data are defined in this block. \\
\hyperref[href_i:new_data_files_def]{(Back to {\tt control\_assemble\_sph)}}

\paragraph{\tt delete\_original\_data\_flag}
\label{href_t:delete_original_data_flag}
\verb|[delete_original_data_flag]| \\
If this flag set to \verb|YES|, original specter data is deleted at the end of program. 

\subsection{\tt new\_time\_step\_ctl}
\label{href_t:new_time_step_ctl}
Parameters to modify time step and time data in the new restrat file. \\
\hyperref[href_i:new_time_step_ctl]{(Back to {\tt control\_assemble\_sph)}}

\paragraph{\tt magnetic\_field\_ratio\_ctl}
\label{href_t:i_step_init_ctl_a} 
\verb|[ISTEP_START]| \\
New time step \verb|[ISTEP_START]| for the restart file is defined by integer.

\paragraph{\tt i\_step\_rst\_ctl}
\label{href_t:i_step_rst_ctl_a} 
\verb|[ISTEP_RESTART]| \\
New step number of restrart file \verb|[ISTEP_RESTART]| is defined by integer.

\paragraph{\tt time\_init\_ctl}
\label{href_t:time_init_ctl_a} 
\verb|[INITIAL_TIME]| \\
New time data \verb|[INITIAL_TIME]| is defined by real.


\subsection{\tt newrst\_magne\_ctl}
\label{href_t:newrst_magne_ctl}
Parameters to modify magnetic field are defined in this block. \\
\hyperref[href_i:newrst_magne_ctl]{(Back to {\tt control\_assemble\_sph)}}

\paragraph{\tt magnetic\_field\_ratio\_ctl}
\label{href_t:magnetic_field_ratio_ctl} 
\verb|[ratio]| \\
Ratio of new magnetic field data to original magnetic field \verb|[ratio]| is defined by real.

\newpage
\section{GNU GENERAL PUBLIC LICENSE}

\begin{center}
{\parindent 0in

Copyright \copyright\ 1989, 1991 Free Software Foundation, Inc.

\bigskip

51 Franklin Street, Fifth Floor, Boston, MA  02110-1301, USA

\bigskip

Everyone is permitted to copy and distribute verbatim copies
of this license document, but changing it is not allowed.
}
\end{center}

\begin{center}
{\bf\large Preamble}
\end{center}


The licenses for most software are designed to take away your freedom to
share and change it.  By contrast, the GNU General Public License is
intended to guarantee your freedom to share and change free software---to
make sure the software is free for all its users.  This General Public
License applies to most of the Free Software Foundation's software and to
any other program whose authors commit to using it.  (Some other Free
Software Foundation software is covered by the GNU Library General Public
License instead.)  You can apply it to your programs, too.

When we speak of free software, we are referring to freedom, not price.
Our General Public Licenses are designed to make sure that you have the
freedom to distribute copies of free software (and charge for this service
if you wish), that you receive source code or can get it if you want it,
that you can change the software or use pieces of it in new free programs;
and that you know you can do these things.

To protect your rights, we need to make restrictions that forbid anyone to
deny you these rights or to ask you to surrender the rights.  These
restrictions translate to certain responsibilities for you if you
distribute copies of the software, or if you modify it.

For example, if you distribute copies of such a program, whether gratis or
for a fee, you must give the recipients all the rights that you have.  You
must make sure that they, too, receive or can get the source code.  And
you must show them these terms so they know their rights.

We protect your rights with two steps: (1) copyright the software, and (2)
offer you this license which gives you legal permission to copy,
distribute and/or modify the software.

Also, for each author's protection and ours, we want to make certain that
everyone understands that there is no warranty for this free software.  If
the software is modified by someone else and passed on, we want its
recipients to know that what they have is not the original, so that any
problems introduced by others will not reflect on the original authors'
reputations.

Finally, any free program is threatened constantly by software patents.
We wish to avoid the danger that redistributors of a free program will
individually obtain patent licenses, in effect making the program
proprietary.  To prevent this, we have made it clear that any patent must
be licensed for everyone's free use or not licensed at all.

The precise terms and conditions for copying, distribution and
modification follow.

\begin{center}
{\Large \sc Terms and Conditions For Copying, Distribution and
  Modification}
\end{center}


%\renewcommand{\theenumi}{\alpha{enumi}}
\begin{enumerate}

\addtocounter{enumi}{-1}

\item 

This License applies to any program or other work which contains a notice
placed by the copyright holder saying it may be distributed under the
terms of this General Public License.  The ``Program'', below, refers to
any such program or work, and a ``work based on the Program'' means either
the Program or any derivative work under copyright law: that is to say, a
work containing the Program or a portion of it, either verbatim or with
modifications and/or translated into another language.  (Hereinafter,
translation is included without limitation in the term ``modification''.)
Each licensee is addressed as ``you''.

Activities other than copying, distribution and modification are not
covered by this License; they are outside its scope.  The act of
running the Program is not restricted, and the output from the Program
is covered only if its contents constitute a work based on the
Program (independent of having been made by running the Program).
Whether that is true depends on what the Program does.

\item You may copy and distribute verbatim copies of the Program's source
  code as you receive it, in any medium, provided that you conspicuously
  and appropriately publish on each copy an appropriate copyright notice
  and disclaimer of warranty; keep intact all the notices that refer to
  this License and to the absence of any warranty; and give any other
  recipients of the Program a copy of this License along with the Program.

You may charge a fee for the physical act of transferring a copy, and you
may at your option offer warranty protection in exchange for a fee.

\item

You may modify your copy or copies of the Program or any portion
of it, thus forming a work based on the Program, and copy and
distribute such modifications or work under the terms of Section 1
above, provided that you also meet all of these conditions:

\begin{enumerate}

\item 

You must cause the modified files to carry prominent notices stating that
you changed the files and the date of any change.

\item

You must cause any work that you distribute or publish, that in
whole or in part contains or is derived from the Program or any
part thereof, to be licensed as a whole at no charge to all third
parties under the terms of this License.

\item
If the modified program normally reads commands interactively
when run, you must cause it, when started running for such
interactive use in the most ordinary way, to print or display an
announcement including an appropriate copyright notice and a
notice that there is no warranty (or else, saying that you provide
a warranty) and that users may redistribute the program under
these conditions, and telling the user how to view a copy of this
License.  (Exception: if the Program itself is interactive but
does not normally print such an announcement, your work based on
the Program is not required to print an announcement.)

\end{enumerate}


These requirements apply to the modified work as a whole.  If
identifiable sections of that work are not derived from the Program,
and can be reasonably considered independent and separate works in
themselves, then this License, and its terms, do not apply to those
sections when you distribute them as separate works.  But when you
distribute the same sections as part of a whole which is a work based
on the Program, the distribution of the whole must be on the terms of
this License, whose permissions for other licensees extend to the
entire whole, and thus to each and every part regardless of who wrote it.

Thus, it is not the intent of this section to claim rights or contest
your rights to work written entirely by you; rather, the intent is to
exercise the right to control the distribution of derivative or
collective works based on the Program.

In addition, mere aggregation of another work not based on the Program
with the Program (or with a work based on the Program) on a volume of
a storage or distribution medium does not bring the other work under
the scope of this License.

\item
You may copy and distribute the Program (or a work based on it,
under Section 2) in object code or executable form under the terms of
Sections 1 and 2 above provided that you also do one of the following:

\begin{enumerate}

\item

Accompany it with the complete corresponding machine-readable
source code, which must be distributed under the terms of Sections
1 and 2 above on a medium customarily used for software interchange; or,

\item

Accompany it with a written offer, valid for at least three
years, to give any third party, for a charge no more than your
cost of physically performing source distribution, a complete
machine-readable copy of the corresponding source code, to be
distributed under the terms of Sections 1 and 2 above on a medium
customarily used for software interchange; or,

\item

Accompany it with the information you received as to the offer
to distribute corresponding source code.  (This alternative is
allowed only for noncommercial distribution and only if you
received the program in object code or executable form with such
an offer, in accord with Subsection b above.)

\end{enumerate}


The source code for a work means the preferred form of the work for
making modifications to it.  For an executable work, complete source
code means all the source code for all modules it contains, plus any
associated interface definition files, plus the scripts used to
control compilation and installation of the executable.  However, as a
special exception, the source code distributed need not include
anything that is normally distributed (in either source or binary
form) with the major components (compiler, kernel, and so on) of the
operating system on which the executable runs, unless that component
itself accompanies the executable.

If distribution of executable or object code is made by offering
access to copy from a designated place, then offering equivalent
access to copy the source code from the same place counts as
distribution of the source code, even though third parties are not
compelled to copy the source along with the object code.

\item
You may not copy, modify, sublicense, or distribute the Program
except as expressly provided under this License.  Any attempt
otherwise to copy, modify, sublicense or distribute the Program is
void, and will automatically terminate your rights under this License.
However, parties who have received copies, or rights, from you under
this License will not have their licenses terminated so long as such
parties remain in full compliance.

\item
You are not required to accept this License, since you have not
signed it.  However, nothing else grants you permission to modify or
distribute the Program or its derivative works.  These actions are
prohibited by law if you do not accept this License.  Therefore, by
modifying or distributing the Program (or any work based on the
Program), you indicate your acceptance of this License to do so, and
all its terms and conditions for copying, distributing or modifying
the Program or works based on it.

\item
Each time you redistribute the Program (or any work based on the
Program), the recipient automatically receives a license from the
original licensor to copy, distribute or modify the Program subject to
these terms and conditions.  You may not impose any further
restrictions on the recipients' exercise of the rights granted herein.
You are not responsible for enforcing compliance by third parties to
this License.

\item
If, as a consequence of a court judgment or allegation of patent
infringement or for any other reason (not limited to patent issues),
conditions are imposed on you (whether by court order, agreement or
otherwise) that contradict the conditions of this License, they do not
excuse you from the conditions of this License.  If you cannot
distribute so as to satisfy simultaneously your obligations under this
License and any other pertinent obligations, then as a consequence you
may not distribute the Program at all.  For example, if a patent
license would not permit royalty-free redistribution of the Program by
all those who receive copies directly or indirectly through you, then
the only way you could satisfy both it and this License would be to
refrain entirely from distribution of the Program.

If any portion of this section is held invalid or unenforceable under
any particular circumstance, the balance of the section is intended to
apply and the section as a whole is intended to apply in other
circumstances.

It is not the purpose of this section to induce you to infringe any
patents or other property right claims or to contest validity of any
such claims; this section has the sole purpose of protecting the
integrity of the free software distribution system, which is
implemented by public license practices.  Many people have made
generous contributions to the wide range of software distributed
through that system in reliance on consistent application of that
system; it is up to the author/donor to decide if he or she is willing
to distribute software through any other system and a licensee cannot
impose that choice.

This section is intended to make thoroughly clear what is believed to
be a consequence of the rest of this License.

\item
If the distribution and/or use of the Program is restricted in
certain countries either by patents or by copyrighted interfaces, the
original copyright holder who places the Program under this License
may add an explicit geographical distribution limitation excluding
those countries, so that distribution is permitted only in or among
countries not thus excluded.  In such case, this License incorporates
the limitation as if written in the body of this License.

\item
The Free Software Foundation may publish revised and/or new versions
of the General Public License from time to time.  Such new versions will
be similar in spirit to the present version, but may differ in detail to
address new problems or concerns.

Each version is given a distinguishing version number.  If the Program
specifies a version number of this License which applies to it and ``any
later version'', you have the option of following the terms and conditions
either of that version or of any later version published by the Free
Software Foundation.  If the Program does not specify a version number of
this License, you may choose any version ever published by the Free Software
Foundation.

\item
If you wish to incorporate parts of the Program into other free
programs whose distribution conditions are different, write to the author
to ask for permission.  For software which is copyrighted by the Free
Software Foundation, write to the Free Software Foundation; we sometimes
make exceptions for this.  Our decision will be guided by the two goals
of preserving the free status of all derivatives of our free software and
of promoting the sharing and reuse of software generally.

\begin{center}
{\Large\sc
No Warranty
}
\end{center}

\item
{\sc Because the program is licensed free of charge, there is no warranty
for the program, to the extent permitted by applicable law.  Except when
otherwise stated in writing the copyright holders and/or other parties
provide the program ``as is'' without warranty of any kind, either expressed
or implied, including, but not limited to, the implied warranties of
merchantability and fitness for a particular purpose.  The entire risk as
to the quality and performance of the program is with you.  Should the
program prove defective, you assume the cost of all necessary servicing,
repair or correction.}

\item
{\sc In no event unless required by applicable law or agreed to in writing
will any copyright holder, or any other party who may modify and/or
redistribute the program as permitted above, be liable to you for damages,
including any general, special, incidental or consequential damages arising
out of the use or inability to use the program (including but not limited
to loss of data or data being rendered inaccurate or losses sustained by
you or third parties or a failure of the program to operate with any other
programs), even if such holder or other party has been advised of the
possibility of such damages.}

\end{enumerate}


\begin{center}
{\Large\sc End of Terms and Conditions}
\end{center}


\pagebreak[2]

\section*{Appendix: How to Apply These Terms to Your New Programs}

If you develop a new program, and you want it to be of the greatest
possible use to the public, the best way to achieve this is to make it
free software which everyone can redistribute and change under these
terms.

  To do so, attach the following notices to the program.  It is safest to
  attach them to the start of each source file to most effectively convey
  the exclusion of warranty; and each file should have at least the
  ``copyright'' line and a pointer to where the full notice is found.

\begin{quote}
one line to give the program's name and a brief idea of what it does. \\
Copyright (C) yyyy  name of author \\

This program is free software; you can redistribute it and/or modify
it under the terms of the GNU General Public License as published by
the Free Software Foundation; either version 2 of the License, or
(at your option) any later version.

This program is distributed in the hope that it will be useful,
but WITHOUT ANY WARRANTY; without even the implied warranty of
MERCHANTABILITY or FITNESS FOR A PARTICULAR PURPOSE.  See the
GNU General Public License for more details.

You should have received a copy of the GNU General Public License
along with this program; if not, write to the Free Software
Foundation, Inc., 51 Franklin Street, Fifth Floor, Boston, MA  02110-1301, USA.
\end{quote}

Also add information on how to contact you by electronic and paper mail.

If the program is interactive, make it output a short notice like this
when it starts in an interactive mode:

\begin{quote}
Gnomovision version 69, Copyright (C) yyyy  name of author \\
Gnomovision comes with ABSOLUTELY NO WARRANTY; for details type `show w'. \\
This is free software, and you are welcome to redistribute it
under certain conditions; type `show c' for details.
\end{quote}


The hypothetical commands {\tt show w} and {\tt show c} should show the
appropriate parts of the General Public License.  Of course, the commands
you use may be called something other than {\tt show w} and {\tt show c};
they could even be mouse-clicks or menu items---whatever suits your
program.

You should also get your employer (if you work as a programmer) or your
school, if any, to sign a ``copyright disclaimer'' for the program, if
necessary.  Here is a sample; alter the names:

\begin{quote}
Yoyodyne, Inc., hereby disclaims all copyright interest in the program \\
`Gnomovision' (which makes passes at compilers) written by James Hacker. \\

signature of Ty Coon, 1 April 1989 \\
Ty Coon, President of Vice
\end{quote}


This General Public License does not permit incorporating your program
into proprietary programs.  If your program is a subroutine library, you
may consider it more useful to permit linking proprietary applications
with the library.  If this is what you want to do, use the GNU Library
General Public License instead of this License.

\end{appendices}

\end{document}
